\chapter{Implementierung}
\section{Beispieldatensatz}\label{sec:dataset}

\begin{table}[htbp]
    \centering
    % Erste Reihe
    \begin{minipage}{0.5\textwidth}
      \centering
      \caption{Haltestellen der Buslinien}\label{tab:buslines}
      \csvautobooktabular[separator=comma]{your_content_folder/tabellen/bus-lines.csv}
    \end{minipage}\hfill
    \begin{minipage}{0.5\textwidth}
      \centering
      \caption{Touren inkl. Startzeiten}\label{tab:lines}
      \csvautobooktabular[separator=comma]{your_content_folder/tabellen/lines.csv}
    \end{minipage}
\end{table}
\begin{table}
    \centering
    \caption{Fiktive Busse inkl. Eigenschaften und Fahrerschichten/-pausen}\label{tab:busses}
    \csvautobooktabular[separator=comma]{your_content_folder/tabellen/busses.csv}
\end{table}  
\begin{table}
      \centering
      \caption{Fiktive Nachfragesituation}\label{tab:demand}
      \csvautobooktabular[separator=comma]{your_content_folder/tabellen/demand.csv}
\end{table}

\section{Code der implementierten Settings}

Siehe digitaler Anhang auf dem abgegebenen USB-Stick.