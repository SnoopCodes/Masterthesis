\chapter{Einleitung}
\label{chapter:1}
Das Bestreben nach dem effizienten Einsatz von Ressourcen ist wohl so alt wie die Nutzung von Ressourcen selbst. In Großstädten sind Busse teil eines umfassenden öffentlichen Personennahverkehr (ÖPNV)-Netzwerks und ergänzen U-Bahn und/oder Straßenbahnen, um der Nachfrage nach Mobilität ausreichend Kapazität entgegenzubringen. Im ländlicheren Bereich sind Busse für den Transport von Personengruppen hingegen oft die einzige Mobilitätsform, da schlicht keine ausreichend hohe, konstante Nachfrage für U-Bahnen oder Ähnliches besteht.

\textbf{Noch unsauber:}

Während es sich in Großstädten schwer vermeiden lässt einen gewissen Kapazitätspuffer vorhalten zu müssen, um die hohe Nachfrage zu den Stoßzeiten bedienen zu können, stellt sich hier für den ländlichen Bereich die Frage, ob eine möglichst geringe Kapazität nicht noch effizienter genutzt werden kann, um die zur Verfügung stehenden Mittel möglichst effizient zu nutzen.

Während eine feste Struktur aus linienbasierten Systemen in Großstädten sinnvoll ist, um einen geregelten Fluss von Personen und Verkehr zu gewährleisten und planbar zu machen, mangelt es im ländlichen Bereich an der Nachfrage, um ein solch aufwendig zu implementierendes System aufrecht zu erhalten / 


Heutzutage sind Menschen daran gewöhnt viele Dinge und Leistungen jederzeit durch das Internet abrufen zu können. Daher kommt der Fähigkeit auf Abruf innerhalb eines kurzen Zeitraums reagieren zu können eine wichtige Rolle zu.   


\textbf{irgendwo einordnen: Betrachtung autonomer Busse in den Settings gar incht mal mehr sooo unwahrscheinlich, da auch Forschung dazu stattfindet - Quellen}

%%%%%%%%%%%%%%%%%%%%
%%%%%%%%%%%%%%%%%%%%
%%%%%%%%%%%%%%%%%%%%
%%%%%%%%%%%%%%%%%%%%
\section{Motivation} 
\label{sec:1.1}
$\rightarrow$ warum ist das Thema relevant?


\begin{itemize}
    \item CO\textsubscript{2}-Reduktion \& Individualmobilität: aktueller Zielkonflikt im ÖPNV
    \item Herausforderungen im ländlichen Raum (geringe Auslastung, lange Taktzeiten)
    \item Technologischer Fortschritt: Autonome Busse \& digitale Nachfrageerfassung ermöglichen neue Konzepte
    \item Klassischer Linienverkehr: fixe Zeiten \& Routen, hohe Bündelung, geringe Flexibilität
    \item Ridepooling: Tür-zu-Tür, aber hohe operative Komplexität, oft ineffizient
    \item On-Demand-Linienbusse als Hybridform: planbare, aber flexible Nachfragebedienung
\end{itemize}
%%%%%%%%%%%%%%%%%%%%
%%%%%%%%%%%%%%%%%%%%
%%%%%%%%%%%%%%%%%%%%
%%%%%%%%%%%%%%%%%%%%
\section{Zielsetzung der Arbeit}
\label{sec:1.2}
Leitfrage:
\begin{enumerate}
    \item Wie geeignet ist das von Vlćek \& Schulz entwickelte Modell zur ressourceneffizienten Planung von On-Demand-Linienbusverkehren unter variierenden betrieblichen Bedingungen?
\end{enumerate}
\vspace{1em}
Dazu wird folgendes gemacht (eventuell auf so auch schon die Beschreibung was in welchem Kapitel gemacht wird, abarbeiten):

\textbf{PASST NICHT MEHR!!!!}
\begin{itemize}
    \item 2.1 Kontext des zu implementierenden Modells beschrieben
    \item 2.2 Modell wird eingeordnet
    \item 3. Mathematischer Modellaufbau wird beschrieben
    \item 4. 
    \begin{itemize}
        \item 4.1 - 4.3 Implementierung eines zu veröffentlichenden Optimierungsmodells
        \item 4.4 Validierung durch Reproduktion der Ergebnisse (fraglich)
        \item 4.5 Ergebnisvergleich \& Plausibilität
    \end{itemize} 
    \item 5. Analyse von Anwendbarkeit, Stärken \& Schwächen des Modells, Erweiterungen
\end{itemize}
