\chapter{Einleitung}
\section{Motivation}

\begin{itemize}
    \item CO\textsubscript{2}-Reduktion \& Individualmobilität: aktueller Zielkonflikt im ÖPNV
    \item Herausforderungen im ländlichen Raum (geringe Auslastung, lange Taktzeiten)
    \item Technologischer Fortschritt: Autonome Busse \& digitale Nachfrageerfassung ermöglichen neue Konzepte
    \item Klassischer Linienverkehr: fixe Zeiten \& Routen, hohe Bündelung, geringe Flexibilität
    \item Ridepooling: Tür-zu-Tür, aber hohe operative Komplexität, oft ineffizient
    \item On-Demand-Linienbusse als Hybridform: planbare, aber flexible Nachfragebedienung
\end{itemize}
\section{Zielsetzung der Arbeit}
\begin{itemize}
    \item Implementierung eines zu veröffentlichenden Optimierungsmodells
    \item Validierung durch Reproduktion der publizierten Ergebnisse
    \item Analyse von Anwendbarkeit, Stärken \& Schwächen des Modells
    \item Aufbau der Arbeit (welche Inhalte in welchen Kapiteln)
\end{itemize}
