\chapter{Einleitung}
%%%%%%%%%%%%%%%%%%%%
%%%%%%%%%%%%%%%%%%%%
%%%%%%%%%%%%%%%%%%%%
%%%%%%%%%%%%%%%%%%%%
\section{Motivation} 
\label{sec:1.1}
$\rightarrow$ warum ist das Thema relevant?


\begin{itemize}
    \item CO\textsubscript{2}-Reduktion \& Individualmobilität: aktueller Zielkonflikt im ÖPNV
    \item Herausforderungen im ländlichen Raum (geringe Auslastung, lange Taktzeiten)
    \item Technologischer Fortschritt: Autonome Busse \& digitale Nachfrageerfassung ermöglichen neue Konzepte
    \item Klassischer Linienverkehr: fixe Zeiten \& Routen, hohe Bündelung, geringe Flexibilität
    \item Ridepooling: Tür-zu-Tür, aber hohe operative Komplexität, oft ineffizient
    \item On-Demand-Linienbusse als Hybridform: planbare, aber flexible Nachfragebedienung
\end{itemize}
%%%%%%%%%%%%%%%%%%%%
%%%%%%%%%%%%%%%%%%%%
%%%%%%%%%%%%%%%%%%%%
%%%%%%%%%%%%%%%%%%%%
\section{Zielsetzung der Arbeit}
\label{sec:1.2}
Leitfrage:
\begin{enumerate}
    \item Wie geeignet ist das von Vlćek \& Schulz entwickelte Modell zur ressourceneffizienten Planung von On-Demand-Linienbusverkehren unter variierenden betrieblichen Bedingungen?
\end{enumerate}
\vspace{1em}
Dazu wird folgendes gemacht (eventuell auf so auch schon die Beschreibung was in welchem Kapitel gemacht wird, abarbeiten):
\begin{itemize}
    \item 2.1 Kontext des zu implementierenden Modells beschrieben
    \item 2.2 Modell wird eingeordnet
    \item 3. Mathematischer Modellaufbau wird beschrieben
    \item 4. 
    \begin{itemize}
        \item 4.1 - 4.3 Implementierung eines zu veröffentlichenden Optimierungsmodells
        \item 4.4 Validierung durch Reproduktion der Ergebnisse (fraglich)
        \item 4.5 Ergebnisvergleich \& Plausibilität
    \end{itemize} 
    \item 5. Analyse von Anwendbarkeit, Stärken \& Schwächen des Modells, Erweiterungen
\end{itemize}
