\chapter{Mathematisches Modell}
\label{chapter:4}
\section{Zentrale Modellannahmen}
Schulz \& Vlćek stellen foglende vereinfachenden Annahmen auf:
\begin{itemize}
    \item Es wird ein gemeinsames Depot für alle Busse angenommen
    \item Die Arbeits- und Pausenzeiten der Fahrer entsprechen den gesetzlichen Regelungen
    \item Die feste Planung einer festgelegten Pausenzeit für die Fahrer steht im Einklang mit geltendem Recht
    \item Die Zuteilung von Fahrer zu Bus erfolgt im Vorfeld des Betrachtungszeitraums und wird als gegeben angenommen. Der Bus ist somit im Rahmen der Fähigkeiten des Fahrers zwischen Beginn und  Schichtende des Fahrers verfügbar
    \item Die Busse müssen während des Betrachtungszeitraums nicht tanken/aufgeladen werden, da sie für den gesamten Betrachtungszeitraum als ausreichend betankt/aufgeladen angenommen werden
    \item Die Busse sind bis auf die Kapazität homogen
\end{itemize}

Für alle Settings ist die Menge aller relevanter Knoten $V$, sowie die Menge $A$ aller Verbindungen bzw. Kanten zwischen den Knoten aufzustellen.

\section{Erstes Setting}
Zur Erinnerung: In diesem Setting werden homogene autonome Busse eingesetzt, die keinen Einfluss auf die Lösung haben. Die Busse werden in der Modellformulierung nicht erwähnt, sind aber die Instanz, die die Verbindungen nutzt und die Knoten bedient.
\label{sec:4.1}
\subsection{Szenario 1.1: Alle Touren vollständig bedienen}
\label{sec:4.1.1}
Zunächst wird von Schulz \& Vlćek der Fall betrachtet, in dem alle Touren (Buslinien zu unterschiedlichen Startzeiten) vollständig bedient werden müssen.
%Hier evtl. Grafik, aber nicht sicher, weil zu nah an Schulz \& Vlćek dran
Die Menge aller relevanter Knoten ergibt sich in diesem ersten Szenario daher aus:

\[
V = \{ D, s^1_1, \dots, s^n_1, s^1_{m_1}, \dots, s^n_m \}
\]

Um alle Möglichkeiten für das Lösen des Problems zu erhalten, sind Verbindungen zwischen Touren möglich und werden durch die Fahrtzeit zwischen der letzten Haltestelle von Tour $l$ und der ersten Haltestelle von Nachfolgertour $l'$ mit $t_{s^l_{m_l}, s^{l'}_1}$ beschrieben. Eine Verbindung von zwei Touren darf in einer möglichen Lösung des Problems nur dann bestehen, wenn folgende Gleichung erfüllt ist: 

\begin{equation}
    \label{eq:4.1}
    \bar{t}_{s^l_{m_l}} + t_{s^l_{m_l}, s^{l'}_1} \leq \bar{t}_{s^{l'}_1}
\end{equation}

Gleichung \ref{eq:4.1} beschreibt, dass der zeitliche Abstand zwischen der letzten Haltestelle einer Tour $s^l_{m_l}$ und der ersten Haltestelle der Nachfolgertour $s^{l'}_1$ groß genug sein muss, um die Fahrtzeit zwischen den Haltestellen abdecken zu können.

Folglich ergibt sich für das erste Setting:
\[
A = \{(D, s_1^1), \ldots, (D, s_1^n), (s_{m_1}^1, D), \ldots, (s_{m_n}^n, D), (s_1^1, s_{m_1}^1), \ldots, (s_1^n, s_{m_n}^n)\}   \\   
\cup \{(s_{m_l}^l, s_1^{l'}) : \bar{t}_{s_{m_l}^l} + t_{s_{m_l}^l, s_1^{l'}} \leq \bar{t}_{s_1^{l'}}\}
\]

Die folgende Modellformulierung für das erste Setting von Schulz \& Vlćek verwendet die binäre Entscheidungsvariable $x_{ij}$, welche 1 ist, wenn die Verbindung $(i,j) \in A$ von $i$ nach $j$  genutzt wird und sonst den Wert 0 hat.

\begin{equation}
    min \sum_{j:(D,j) \in A_1} x_{Dj}
\label{eq:4.2}
\end{equation}
unter den Nebenbedingungen:
\begin{alignat}{3}
    &\sum_{j:(i,j) \in A_1} x_{ij} - \sum_{j:(j,i) \in A_1} x_{ji} &&= \quad 0 \hspace{4cm} &&\forall i \in V \label{eq:4.3}\\
    &\hspace{3cm} x_{s_1^l, s_{m_l}^l} &&= \quad 1\hspace{4cm} &&\forall l \in L\label{eq:4.4}\\
    &\hspace{3.5cm} x_{ij} &&\geq \quad 0 \hspace{4cm}&&\forall (i,j) \in A_1\label{eq:4.5}
\end{alignat}

Die Zielfunktion \ref{eq:4.2} minimiert die Anzahl der Busse die das Depot verlassen. Die Nebenbedingung \ref{eq:4.3} stellt sicher, dass die Flusserhaltung für jeden Knoten gilt, d.h. die Anzahl der eingehenden Verbindungen ist gleich der Anzahl der ausgehenden Verbindungen. Nebenbedingung \ref{eq:4.4} stellt sicher, dass jede Tour bedient wird. Nebenbedingung \ref{eq:4.5} ist die Nichtnegativitätsbedingung der Entscheidungsvariable.

\textbf{HIER NOCH CHECKEN OB KLASSIFIZIERUNG ALS EINE BESTIMMTE ART VON MODELL WIE BEI Schulz \& Vlćek}

\subsection{Szenario 1.2: Nur Touren mit Nachfrage bedienen}
In diesem Szenario werden XXXXXX bedient die ....

\textbf{HIER EVTL. GRAFIK ZUR VISUALISIERUNG}

Die Menge aller relevanter Knoten ändert sich zu

\[
\bar{V} = \{ D, s^1_1, \dots, s^1_{m_1}, s^n_1, \dots, s^n_{m_n} \},
\]

sodass nun alle Haltestellen im System, sowie das Depot aufgenommen sind. Auch die Menge der relevanten Verbindungen ändert sich und ergibt sich zu:
\[
\bar{A}= \{(D, s_1^1), \ldots, (D, s^n_{m_n}), (s_1^1, D), \ldots, (s_{m_n}^n, D), (s_1^1, s_{m_1}^1), \ldots, (s_1^n, s_{m_n}^n)\}   \\   
\cup \{(s_i^l, s_j^{l'}) : \bar{t}_{s_i^l} + t_{s_i^l, s_j^{l'}} \leq \bar{t}_{s_j^{l'}}\},
\]
mit \(i =  {1, \dots m_l}\) und \(j = {1, \dots m_{l'}}\).

\textbf{HIER DIE THEMATIK MIT ÜBERLAPPENDEN CUSTOMER TRIPS ERKLÄREN}

%%%%%%%%%%%%%%%%%%%%
%%%%%%%%%%%%%%%%%%%%
%%%%%%%%%%%%%%%%%%%%
%%%%%%%%%%%%%%%%%%%%

