\chapter{Implementierung \& Validierung des Modells}
%%%%%%%%%%%%%%%%%%%%
%%%%%%%%%%%%%%%%%%%%
%%%%%%%%%%%%%%%%%%%%
%%%%%%%%%%%%%%%%%%%%
\section{Grundlagen der Implementierung}
\begin{itemize}
    \item Hardware (Macbook Pro mit M1 Prozessor, 16 GB RAM, in VS Code)
    \item Programmiersprache (z.B. Julia/Python)
    \item Verwendete Solver (z.B. Gurobi, GLPK)
    \item Datenstrukturen: Graphenmodell, Knoten/Arc-Logik
\end{itemize}
%%%%%%%%%%%%%%%%%%%%
%%%%%%%%%%%%%%%%%%%%
%%%%%%%%%%%%%%%%%%%%
%%%%%%%%%%%%%%%%%%%%
\section{Struktur der Implementierung}
\begin{itemize}
    \item Modularer Aufbau: Dateninput, Modellkonstruktion, Lösung, Output
    \item Relevante Klassen/Methoden (z.B. für Pfadgenerierung, Kapazitätsprüfung)
    \item Beispiel für den Workflow einer Instanz: Einlesen → Modellaufbau → Lösung → Auswertung
\end{itemize}

%%%%%%%%%%%%%%%%%%%%
%%%%%%%%%%%%%%%%%%%%
%%%%%%%%%%%%%%%%%%%%
%%%%%%%%%%%%%%%%%%%%
\section{Herausforderungen bei der Umsetzung}
\begin{itemize}
    \item Setaufbau
    \item Speicher-/Performanceprobleme bei großen Instanzen
    \item Umgang mit überlappenden Trips und Duplikaten (siehe Paper)
\end{itemize}

%%%%%%%%%%%%%%%%%%%%
%%%%%%%%%%%%%%%%%%%%
%%%%%%%%%%%%%%%%%%%%
%%%%%%%%%%%%%%%%%%%%
\section{Validierung der Implementierung} 

Anhand der Originalergebnisse

Hier darauf achten, dass es 9 Szenarios gibt -> Referenz auf beschriebene Settings in Kapitel 3.3(?)

\begin{itemize}
    \item Szenarien aus Schulz/Vlćek: Fahrplan aus Mecklenburg-Vorpommern
    \item Umsetzung von Settings 1 bis 3 (jeweils relevante Details nennen)
    \item Beispiel: Setting 1 mit künstlich hoher Geschwindigkeit → Reproduzierbarkeit des theoretischen Optimums
\end{itemize}
%%%%%%%%%%%%%%%%%%%%
%%%%%%%%%%%%%%%%%%%%
%%%%%%%%%%%%%%%%%%%%
%%%%%%%%%%%%%%%%%%%%
\section{Ergebnisvergleich \& Plausibilität}
\begin{itemize}
\item Tourenanzahl, Busanzahl, Tourverläufe: Vergleich Paper vs. eigene Lösung
\item Abweichungen und deren mögliche Ursachen (z.B. Rundungsfehler, alternative Pfade)
\item Qualität \& Robustheit der eigenen Implementierung
\end{itemize}

