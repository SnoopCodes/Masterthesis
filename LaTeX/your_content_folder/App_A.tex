\chapter{Anhang A: Literaturrecherche}
\section{Searchstrings}

\textbf{Bei Optimierungsproblemen: DARAUF ACHTEN, DASS DIE JAHRESZAHLEN MIT ANGEGEBEN SIND - DADURCH ERWÄHEN WELCHEN BEI GOOGLE UND WELCHE BEI SCOPUS GENUTZT WURDE}

\section{Suchergebnisse}
\subsection{Suchergebnisse zu line-based urban}
\begin{landscape}
\begin{table}[p]
    \centering
    \caption{Forschungsergebnisse zu Line-based — urban}
    \label{tab:lb-urban}
    \scriptsize
    \setlength{\tabcolsep}{2.5pt}
    \setlength{\arrayrulewidth}{0.1pt}

    \resizebox{\textheight}{!}{%
    \begin{tabular}{
        L{2.0cm}:
        L{2.5cm}:
        L{2.0cm}:
        L{2.0cm}:
        L{2.5cm}:
        L{2.3cm}:
        L{1.8cm}:
        L{2.6cm}:
        L{7.4cm}:
    }
      \hline
      \LH{Paper} & \LH{Zielsetzung} & \LH{Region /\\Land} & \LH{Betrachtungs-\\ebene} &
      \LH{Fokus / \\ Anwendungsfeld} & \LH{Methode} & \LH{Daten} & \LH{KPI} & \LH{Zentrale Erkenntnisse} \\
      \Xhline{0.6pt}

      \textcite{fielbaum_improving_2024} & Integration Ridepooling & — & Linie & Ridepooling & Matem. Optimierungsmodell / Simulation & — & Reisezeit / Fahrzeug-km & zeigt Vorteile einer Ridepooling-Kopplung mit Linien; senkt Betriebskosten; senkt Emissionen. \\ \hline
      \textcite{filippi_exploiting_2024} & Linienoptimierung & Italy & Linie & Modularbus & Matem. Optimierungsmodell & — & Reisezeit / Auslastung & nutzt modulare Kopplung für flexible Kapazität; verkürzt Reisezeiten und verbessert Auslastung. \\ \hline
      \textcite{gal_traveling_2017} & Reisezeitprognose & ohne Bezug & Linie & Taktfahrplan & Empirie & Realdaten & Reisezeit / Prognosegüte & liefert praxistaugliche Prognosen der Reisezeiten im Linienverkehr (gute Prognosegüten). \\ \hline
      \textcite{hatzenbuhler_network_2022} & Netzdesign & ohne Bezug & Netz & Autonom & Matem. Optimierungsmodell / Simulation & — & Ø-Wartezeit / Reisezeit & zeigt, wie autonome Busnetz-Designs Nutzer- und Betreiberkosten senken können. \\ \hline
      \textcite{jimenez_urban_2016} & Flottenzuordnung & Spain & Linie & Emissionen & Matem. Optimierungsmodell & Mix & CO\(_2\) / Fahrzeug-km & weist nach, dass optimierte Flottenzuordnung Emissionen im Liniennetz deutlich reduzieren kann. \\ \hline
      \textcite{kim_integrating_2013} & Integration gemischter Flotten & USA & Linie & Flotten & Matem. Optimierungsmodell & — & Kosten/Fahrgast-km / Auslastung & kombinierte feste und flexible Dienste mit gemischten Flotten senken Kosten und erhöhen Auslastung. \\ \hline
      \textcite{rashvand_real-time_2024} & Ankunftsprognose & USA & Linie & KI & Empirie & Realdaten & Prognosegüte / Pünktlichkeit & Deep-Learning-Ansätze liefern robuste Ankunftsprognosen und verbessern die wahrgenommene Pünktlichkeit. \\ \hline
      \textcite{rosca_designing_2024} & Fahrplanoptimierung & Romania & Linie & Flotten & Empirie / Simulation & — & Pünktlichkeit / Kosten/Fahrgast-km & KI-gestützte Planung verbessert Pünktlichkeit und senkt Planungskosten in urbanen Netzen. \\ \hline
      \textcite{tang_data-driven_2021} & Fahrplanoptimierung & China & Linie & Taktfahrplan & Matem. Optimierungsmodell & Realdaten & Ø-Wartezeit / Reisezeit & zweckmäßige MOEA-Fahrpläne verkürzen Warte- und Reisezeiten auf städtischen Linien. \\ \hline
      \textcite{tian_autonomous_2021} & Flottenoptimierung & China & Linie & Autonom & Matem. Optimierungsmodell & — & Kosten/Fahrgast-km / Flottengröße & autonome Flotten mit Unsicherheiten können Betreiberkosten senken und Leistung stabilisieren. \\ \hline
      \textcite{wei_optimizing_2020} & Integration Metro-Bus & China & Netz & Integration & Matem. Optimierungsmodell / Fallstudie & Realdaten & Ø-Wartezeit / Reisezeit & koordinierte Metro-Bus-Planung senkt Nutzerzeiten und verbessert Gesamtleistung. \\ \hline
    \end{tabular}
    }% end resizebox
\end{table}
\end{landscape}


\subsection{Suchergebnisse zu on-demand urban}
\begin{landscape}

    \scriptsize
    \setlength{\tabcolsep}{2.2pt}
    \setlength{\arrayrulewidth}{0.1pt}
    \begin{xltabular}{\textwidth}{%
        L{1.8cm}:
        L{1.8cm}:
        L{1.5cm}:
        L{2.3cm}:
        L{2.3cm}:
        L{2.0cm}:
        L{1.5cm}:
        L{2.4cm}:
        L{6.0cm}:
    }
        \caption{Forschungsergebnisse zu on-demand — urban}\label{tab:od-urban}\\ 
        \hline
        \LH{Paper} & \LH{Zielsetzung} & \LH{Region /\\Land} & \LH{Betrachtungs-\\ebene} &
        \LH{Fokus / \\ Anwendungsfeld} & \LH{Methode} & \LH{Daten} & \LH{KPI} & \LH{Zentrale\\Erkenntnisse} \\
        \Xhline{0.6pt}
        \endfirsthead

        \multicolumn{9}{l}{\small\itshape Fortsetzung von Tabelle~\ref{tab:od-urban}.}\\[0.6\baselineskip]
        \hline
        \LH{Paper} & \LH{Zielsetzung} & \LH{Region /\\Land} & \LH{Betrachtungs-\\ebene} &
        \LH{Fokus / \\ Anwendungsfeld} & \LH{Methode} & \LH{Daten} & \LH{KPI} & \LH{Zentrale\\Erkenntnisse} \\
        \Xhline{0.6pt}
        \endhead

        \hline
        \multicolumn{9}{r}{\small\itshape Fortsetzung auf der nächsten Seite}
        \endfoot

        \hline
        \endlastfoot


        \textcite{abe_introducing_2019} & Autonomen DRT planen & UK & Netz & Autonom & Matem. Optimierungsmodell & — & Kosten/Fahrgast-km, Erreichbarkeit & Optimierung/Szenarien senken Warte- und Reisezeiten im urbanen DRT deutlich. \\ \hline
        
        \textcite{alonso-gonzalez_potential_2018} & DRT-Konzept bewerten & ohne Bezug & Linie & Flexible Busse & Matem. Optimierungsmodell, Empirie & — & Auslastung & Ansatz verbessert Erreichbarkeit und Servicegrad gegenüber Status quo. \\ \hline
        
        \textcite{charisis_drt_2018} & First/Last-Mile verbessern & Deutschland & Linie & First/Last-Mile & — & — & Auslastung, Erreichbarkeit & Optimierung/Szenarien senken Warte- und Reisezeiten im urbanen DRT deutlich. \\ \hline
        
        \textcite{diana_methodology_2009} & DRT-Konzept bewerten & Frankreich & Linie & Flexible Busse & Matem. Optimierungsmodell & — & Auslastung & Ansatz verbessert Erreichbarkeit und Servicegrad gegenüber Status quo. \\ \hline
        
        \textcite{giuffrida_addressing_2021} & DRT-Konzept bewerten & Italien & Region & Flexible Busse & Matem. Optimierungsmodell, Kartenbasierte Raumanalyse (GIS) & Realdaten & Auslastung, Erreichbarkeit & Studie zeigt Potenziale und Grenzen von DRT im urbanen Kontext. \\ \hline
        
        \textcite{hazan_-demand_nodate} & Urbanen DRT bewerten & ohne Bezug & Region & Flexible Busse & Matem. Optimierungsmodell & — & Kosten/Fahrgast-km, Auslastung & Ansatz verbessert Erreichbarkeit und Servicegrad gegenüber Status quo. \\ \hline
        
        \textcite{huang_flexible_2020} & DRT-Konzept bewerten & ohne Bezug & Linie & Flexible Busse & Simulation, Matem. Optimierungsmodell & — & Auslastung & Studie zeigt Potenziale und Grenzen von DRT im urbanen Kontext. \\ \hline
        
        \textcite{inturri_taxi_2021} & DRT per Simulation prüfen & Italien & Netz & Flexible Busse & Simulation & — & Auslastung & Studie zeigt Potenziale und Grenzen von DRT im urbanen Kontext. \\ \hline
        
        \textcite{jager_multi-agent_2018} & Autonomen DRT planen & ohne Bezug & Region & Autonom & Matem. Optimierungsmodell & — & Auslastung & Optimierung/Szenarien senken Warte- und Reisezeiten im urbanen DRT deutlich. \\ \hline
        
        \textcite{kim_optimization_2025} & DRT-Konzept bewerten & Schweiz & Region & Flexible Busse & Matem. Optimierungsmodell & — & Auslastung & Studie zeigt Potenziale und Grenzen von DRT im urbanen Kontext. \\ \hline
        
        \textcite{li_efficient_2024} & DRT-Konzept bewerten & Korea & Linie & Flexible Busse & — & — & Auslastung & Studie zeigt Potenziale und Grenzen von DRT im urbanen Kontext. \\ \hline
        
        \textcite{liyanage_agent-based_2020} & DRT per Simulation prüfen & Australien & Region & Flexible Busse & Simulation & — & Auslastung, CO\(_2\) & Optimierung/Szenarien senken Warte- und Reisezeiten im urbanen DRT deutlich. \\ \hline
        
        \textcite{melo_demand-responsive_2024} & First/Last-Mile verbessern & Schweiz & Netz & First/Last-Mile & — & — & Auslastung & Ansatz verbessert Erreichbarkeit und Servicegrad gegenüber Status quo. \\ \hline
        
        \textcite{meshkani_innovative_2024} & First/Last-Mile verbessern & ohne Bezug & Region & First/Last-Mile & Matem. Optimierungsmodell & — & Auslastung & Studie zeigt Potenziale und Grenzen von DRT im urbanen Kontext. \\ \hline
        
        \textcite{mortazavi_integrated_2024} & DRT-Konzept bewerten & Australien & Netz & Flexible Busse & Matem. Optimierungsmodell, Fallstudie & — & Reisezeit, Kosten/Fahrgast-km & Studie zeigt Potenziale und Grenzen von DRT im urbanen Kontext. \\ \hline
        
        \textcite{pal_smartporter_2016} & Urbanen DRT bewerten & ohne Bezug & Netz & Flexible Busse & Kartenbasierte Raumanalyse (GIS) & — & — & Optimierung/Szenarien senken Warte- und Reisezeiten im urbanen DRT deutlich. \\ \hline

        \textcite{queiroz_instance_2024} & DRT-Konzept bewerten & Frankreich & Linie & Flexible Busse & Matem. Optimierungsmodell & — & Auslastung & Studie zeigt Potenziale und Grenzen von DRT im urbanen Kontext. \\ \hline
        
        \textcite{shinoda_is_2004} & DRT per Simulation prüfen & UK & Linie & Flexible Busse & Simulation, Matem. Optimierungsmodell & — & — & Ansatz verbessert Erreichbarkeit und Servicegrad gegenüber Status quo. \\ \hline
        
        \textcite{sun_research_2025} & DRT-Konzept bewerten & China & Region & Flexible Busse & Matem. Optimierungsmodell & — & Flottengröße, Kosten/Fahrgast-km & Studie zeigt Potenziale und Grenzen von DRT im urbanen Kontext. \\ \hline
        
        \textcite{torrisi_evaluation_2025} & Urbanen DRT bewerten & Italien & Region & Flexible Busse & Matem. Optimierungsmodell & — & Auslastung & Studie zeigt Potenziale und Grenzen von DRT im urbanen Kontext. \\ \hline
        
        \textcite{yan_integrating_2019} & Urbanen DRT bewerten & USA & Netz & First/Last-Mile & Matem. Optimierungsmodell & — & Auslastung & Ansatz verbessert Erreichbarkeit und Servicegrad gegenüber Status quo. \\ \hline
        
        \textcite{yeon_real-time_2025} & DRT-Konzept bewerten & Korea & Linie & Flexible Busse & Matem. Optimierungsmodell, Kartenbasierte Raumanalyse (GIS) & — & Auslastung & Studie zeigt Potenziale und Grenzen von DRT im urbanen Kontext. \\ \hline
        
        \textcite{yi_real-time_2025} & Feeder-Service optimieren & USA & Linie & Feeder & Matem. Optimierungsmodell & — & Auslastung & Studie zeigt Potenziale und Grenzen von DRT im urbanen Kontext. \\ \hline
        
        \textcite{zhang_routing_2024} & Fahrplan/Zuteilung optimieren & China & Linie & Flexible Busse & Matem. Optimierungsmodell & — & Auslastung & Ansatz verbessert Erreichbarkeit und Servicegrad gegenüber Status quo. \\ \hline
    \end{xltabular}
\end{landscape}
\subsection{Suchergebnisse von ridepooling-urban}
\begin{landscape}

    \scriptsize
    \setlength{\tabcolsep}{2.2pt}
    \setlength{\arrayrulewidth}{0.1pt}
    \begin{xltabular}{\textwidth}{%
        L{1.8cm}:
        L{1.8cm}:
        L{1.5cm}:
        L{2.3cm}:
        L{2.3cm}:
        L{2.0cm}:
        L{1.5cm}:
        L{2.4cm}:
        L{6.0cm}:
    }
        \caption{Forschungsergebnisse zu ridepooling — urban}\label{tab:rp-urban}\\ 
        \hline
        \LH{Paper} & \LH{Zielsetzung} & \LH{Region /\\Land} & \LH{Betrachtungs-\\ebene} &
        \LH{Fokus / \\ Anwendungsfeld} & \LH{Methode} & \LH{Daten} & \LH{KPI} & \LH{Zentrale\\Erkenntnisse} \\
        \Xhline{0.6pt}
        \endfirsthead

        \multicolumn{9}{l}{\small\itshape Fortsetzung von Tabelle~\ref{tab:rp-urban}.}\\[0.6\baselineskip]
        \hline
        \LH{Paper} & \LH{Zielsetzung} & \LH{Region /\\Land} & \LH{Betrachtungs-\\ebene} &
        \LH{Fokus / \\ Anwendungsfeld} & \LH{Methode} & \LH{Daten} & \LH{KPI} & \LH{Zentrale\\Erkenntnisse} \\
        \Xhline{0.6pt}
        \endhead

        \hline
        \multicolumn{9}{r}{\small\itshape Fortsetzung auf der nächsten Seite}
        \endfoot

        \hline
        \endlastfoot

        \textcite{agatz_dynamic_2011} & Ridepooling simulieren & Niederlande & Netz & Feeder & Simulation, Matem. Optimierungsmodell & — & Kosten/Fahrgast-km, Auslastung & Pooling und Optimierung reduzieren Warte-/Reisezeiten und Fahrzeug-km im urbanen Kontext deutlich. \\ \hline
        
        \textcite{alonso-gonzalez_value_2020} & Urbanes Ridepooling bewerten & Niederlande & Region & Flexible Busse & Matem. Optimierungsmodell & — & Kosten/Fahrgast-km, Auslastung & Zielkonflikt zwischen Servicequalität und Bündelungsgrad wird sichtbar. \\ \hline
        
        \textcite{alonso-gonzalez_what_2021} & Ridepooling simulieren & ohne Bezug & Region & Pooling & Simulation & — & Kosten/Fahrgast-km, Auslastung & Ridepooling erhöht Abdeckung und Auslastung gegenüber Solo-DRT. \\ \hline
        
        \textcite{alonso-mora_-demand_2017} & Fahrplan/Zuteilung optimieren & USA & Netz & Studierende & Literatur-Review & Literatur & Auslastung & Studie zeigt Potenziale und Grenzen urbanen Ridepoolings. \\ \hline
        
        \textcite{asatryan_ridepooling_2023} & Pooling/Zuteilung optimieren & Deutschland & Haltestelle & Pooling & Simulation & synthetisch & Auslastung & Studie zeigt Potenziale und Grenzen urbanen Ridepoolings. \\ \hline
        
        \textcite{basu_automated_2018} & Ridepooling simulieren & ohne Bezug & Region & Flexible Busse & Matem. Optimierungsmodell & — & Auslastung & Studie zeigt Potenziale und Grenzen urbanen Ridepoolings. \\ \hline
        
        \textcite{bischoff_city-wide_2017} & Ridepooling simulieren & Deutschland & Netz & Pooling & Simulation, Matem. Optimierungsmodell & — & Reisezeit & Pooling und Optimierung reduzieren Warte-/Reisezeiten und Fahrzeug-km im urbanen Kontext deutlich. \\ \hline

        \textcite{chen_exploring_2021} & Ridepooling-Konzept bewerten & ohne Bezug & Region & Flexible Busse & Empirie & Mix & Fahrzeug-km, Kosten/Fahrgast-km & Ridepooling erhöht Abdeckung und Auslastung gegenüber Solo-DRT. \\ \hline
        
        \textcite{engelhardt_speed-up_2020} & Pooling/Zuteilung optimieren & Deutschland & Region & Pooling & Matem. Optimierungsmodell & — & Auslastung & Studie zeigt Potenziale und Grenzen urbanen Ridepoolings. \\ \hline
        
        \textcite{engelhardt_predictive_2023} & Pooling/Zuteilung optimieren & Deutschland & Region & Pooling & — & — & Auslastung & Ridepooling erhöht Abdeckung und Auslastung gegenüber Solo-DRT. \\ \hline
        
        \textcite{fagnant_dynamic_2018} & Autonomen DRT planen & USA & Netz & First/Last-Mile & Simulation & — & Auslastung & Studie zeigt Potenziale und Grenzen urbanen Ridepoolings. \\ \hline
        
        \textcite{fehn_ride-parcel-pooling_2023} & Pooling/Zuteilung optimieren & Deutschland & Region & Pooling & Kartenbasierte Raumanalyse (GIS) & — & Flottengröße, Auslastung & Studie zeigt Potenziale und Grenzen urbanen Ridepoolings. \\ \hline
        
        \textcite{fielbaum_improving_2024} & Pooling/Zuteilung optimieren & Niederlande & Linie & Pooling & Matem. Optimierungsmodell & — & Auslastung & Ridepooling erhöht Abdeckung und Auslastung gegenüber Solo-DRT. \\ \hline
        \midrule
        
        \textcite{garus_exploring_2024} & Urbanes Ridepooling bewerten & Italien & Region & Flexible Busse & Matem. Optimierungsmodell & — & — & Studie zeigt Potenziale und Grenzen urbanen Ridepoolings. \\ \hline
        
        \textcite{jung_dynamic_2016} & Fahrplan/Zuteilung optimieren & USA & Netz & Flexible Busse & Simulation, Matem. Optimierungsmodell & — & Ø-Wartezeit, Auslastung & Ridepooling erhöht Abdeckung und Auslastung gegenüber Solo-DRT. \\ \hline
        
        \textcite{konig_analyzing_2018} & Pooling/Zuteilung optimieren & Deutschland & Netz & Pooling & — & — & CO\(_2\) & Studie zeigt Potenziale und Grenzen urbanen Ridepoolings. \\ \hline
        
        \textcite{leich_should_2019} & Ridepooling-Konzept bewerten & ohne Bezug & Region & Autonom & Simulation, Matem. Optimierungsmodell & Literatur & — & Studie zeigt Potenziale und Grenzen urbanen Ridepoolings. \\ \hline
        
        \textcite{liang_automated_2020} & Ridepooling-Konzept bewerten & Niederlande & Netz & First/Last-Mile & Matem. Optimierungsmodell & — & Reisezeit & Studie zeigt Potenziale und Grenzen urbanen Ridepoolings. \\ \hline
        
        \textcite{liu_filtering_2024} & Autonomen DRT planen & Frankreich & Linie & Autonom & Matem. Optimierungsmodell & — & — & Studie zeigt Potenziale und Grenzen urbanen Ridepoolings. \\ \hline
        
        \textcite{liu_quantifying_2025} & Pooling/Zuteilung optimieren & China & Netz & Pooling & Fallstudie & — & Auslastung & Studie zeigt Potenziale und Grenzen urbanen Ridepoolings. \\ \hline

        \textcite{liu_analysis_2015} & Ridepooling-Konzept bewerten & China & Netz & Flexible Busse & Matem. Optimierungsmodell & — & Auslastung & Studie zeigt Potenziale und Grenzen urbanen Ridepoolings. \\ \hline
        
        \textcite{lobel_detours_2025} & Pooling/Zuteilung optimieren & USA & Netz & Pooling & Matem. Optimierungsmodell & — & — & Studie zeigt Potenziale und Grenzen urbanen Ridepoolings. \\ \hline
        
        \textcite{schatzmann_investigating_2023} & Pooling/Zuteilung optimieren & Deutschland & Netz & Pooling & Matem. Optimierungsmodell, Empirie & Mix & — & Studie zeigt Potenziale und Grenzen urbanen Ridepoolings. \\ \hline
        
        \textcite{shen_integrating_2018} & First/Last-Mile verbessern & USA & Netz & First/Last-Mile & Simulation, Matem. Optimierungsmodell & — & Auslastung & Studie zeigt Potenziale und Grenzen urbanen Ridepoolings. \\ \hline
        
        \textcite{soza-parra_shareability_2024} & Pooling/Zuteilung optimieren & UK & Region & Pooling & Kartenbasierte Raumanalyse (GIS) & — & Auslastung & Studie zeigt Potenziale und Grenzen urbanen Ridepoolings. \\ \hline
        
        \textcite{stiglic_benefits_2015} & Ridepooling-Konzept bewerten & Niederlande & Netz & Pooling & Matem. Optimierungsmodell & — & — & Studie zeigt Potenziale und Grenzen urbanen Ridepoolings. \\ \hline
        
        \textcite{stiglic_enhancing_2018} & Urbanes Ridepooling bewerten & Niederlande & Netz & Studierende & Matem. Optimierungsmodell & — & — & Ridepooling erhöht Abdeckung und Auslastung gegenüber Solo-DRT. \\ \hline

        \textcite{vosooghi_shared_2019} & Autonomen DRT planen & Frankreich & Netz & Autonom & Simulation, Matem. Optimierungsmodell & — & Auslastung & Studie zeigt Potenziale und Grenzen urbanen Ridepoolings. \\ \hline
        
        \textcite{wallar_vehicle_2018} & Urbanes Ridepooling bewerten & ohne Bezug & Linie & Flexible Busse & Matem. Optimierungsmodell & — & Auslastung & Zielkonflikt zwischen Servicequalität und Bündelungsgrad wird sichtbar. \\ \hline
        
        \textcite{wen_transit-oriented_2018} & Autonomen DRT planen & USA & Netz & Autonom & Simulation, Matem. Optimierungsmodell & — & Auslastung & Studie zeigt Potenziale und Grenzen urbanen Ridepoolings. \\ \hline
        
        \textcite{zwick_ride-pooling_2021} & Ridepooling-Konzept bewerten & Deutschland & Region & Pooling & Simulation, Literatur-Review & Literatur & — & Studie zeigt Potenziale und Grenzen urbanen Ridepoolings. \\ \hline
        
        \textcite{zwick_analysis_2020} & Pooling/Zuteilung optimieren & Schweiz & Netz & Pooling & Matem. Optimierungsmodell & synthetisch & — & Ridepooling erhöht Abdeckung und Auslastung gegenüber Solo-DRT. \\ \hline
        
        \textcite{zwick_ride-pooling_2022} & Pooling/Zuteilung optimieren & Deutschland & Region & Pooling & Simulation, Kartenbasierte Raumanalyse (GIS) & — & Flottengröße, Fahrzeug-km & Pooling und Optimierung reduzieren Warte-/Reisezeiten und Fahrzeug-km im urbanen Kontext deutlich. \\ \hline
                
    \end{xltabular}
\end{landscape}
