\chapter{Methodik \& Modellbeschreibung}
%%%%%%%%%%%%%%%%%%%%
%%%%%%%%%%%%%%%%%%%%
%%%%%%%%%%%%%%%%%%%%
%%%%%%%%%%%%%%%%%%%%
\section{Ziel des Modells}
Ziel des Modells ist die Minimierung der Anzahl benötigter Busse unter Berücksichtigung bekannter Nachfrage und Einsatz von 


Motivation: Ressourceneffizienz \& Angebotsoptimierung
%%%%%%%%%%%%%%%%%%%%
%%%%%%%%%%%%%%%%%%%%
%%%%%%%%%%%%%%%%%%%%
%%%%%%%%%%%%%%%%%%%%
\section{Erläuterung der drei Settings}
\begin{itemize}
    \item \textbf{Homogene autonome Busse:} Keine Fahrer, gleiche Kapazität. Fokus auf reine Tourenoptimierung.
    \item \textbf{Heterogene Busse:} Verschiedene Kapazitäten \(\rightarrow\) neue Zuordnungsprobleme. Selektive Tourabdeckung bei Bedarf.
    \item \textbf{Busse mit Fahrerpausen:} Zeitfenster für Einsetzbarkeit, gesetzliche Pausen. Auswirkungen auf Tourverläufe \& Zuordnung.
\end{itemize}

Verweis auf  9 Szenarien im Paper, die diese Settings abdecken.
%%%%%%%%%%%%%%%%%%%%
%%%%%%%%%%%%%%%%%%%%
%%%%%%%%%%%%%%%%%%%%
%%%%%%%%%%%%%%%%%%%%
\section{Zentrale Modellannahmen, Eingaben \& Nebenbedingungen}
\begin{itemize}
    \item Fester Fahrplan \& Linienstruktur
    \item Fahrzeugkapazität \& Depotstruktur
    \item Statischer Demand (vor Fahrtbeginn bekannt)
    \item Kein Laden/Tanken
    \item Übersicht zu relevanten Parametern (Stopps, Zeiten, Kapazität, Fahrzeiten)
\end{itemize}
