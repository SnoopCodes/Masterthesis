\chapter{Problemstellung}
\label{chapter:3}
\label{chapter:Problemstellung}
Das zu lösende Optimierungsproblem beschäftigt sich mit der Einsatzplanung von Bussen in einem, auf Abruf operierenden, linienbasierten System mehrerer Buslinien. Ziel ist es, in allen Settings, die Anzahl der eingesetzten Busse zu minimieren.
Grundsätzlich kann eine Buslinie mehrmals im zu betrachtenden Zeitraum bedient bzw. gestartet werden. 
Daher wird im weiteren Verlauf jede einzigartige Kombination aus Buslinie und Startzeit der Buslinie als Tour \( l \in L = \{1,  \dots , n\} \) bezeichnet. 
Das linienbasierte System wird durch ein Graphenmodell mit Knoten und Kanten aufgebaut. Jede Buslinie besteht aus einer Anzahl von Haltestellen (Knoten) $s_1, \dots, s_{m}$. Hier sei erwähnt, dass sich die Zusammensetzung und Reihenfolge der Haltestellen je Buslinie mit unterschiedlicher Startzeit durch die einzelnen Touren hinweg nicht verändert. Deshalb kann die Notation der Tour mit in die Notation der Haltestellen aufgenommen werden und es ergibt sich für jede Tour eine Abfolge von Haltestellen $s^l_1,  \dots , s^l_{m_l}$.
Jeder Haltestelle wird eine Zeit $\bar{t}_{s^l_j}$ zugewiesen, zu der der Bus dort hält, es gilt: $\bar{t}_{s^l_j}$, $l \in L,\ j = 1, \dots, m_l$.
In der Realität sind die einzuhaltenden Zeiten durch einen zuvor aufgestellten Fahrplan gegeben. Für ein höheres Maß an Flexibilität bei der Implementierung des Modells werden die Zeiten der einzelnen Haltestellen in dieser Arbeit ausgehend von der Startzeit der jeweiligen Tour und der gewählten Fahrgeschwindigkeit der Busse berechnet. Eine detaillierte Beschreibung der Berechnung folgt in Kapitel \textbf{XXXXXXX}.


Schulz \& Vlćek argumentieren, dass die Zeiten für Ein- und Ausstieg in Relation zur Fahrzeit zur vernachlässigen sind. Daher können die Busse die Haltestellen im Vorbeifahren bedienen.
Für die Fahrtzeit zwischen zwei beliebigen Haltestellen im System $t_{s^l_j, s^{l'}_{j'}}$ muss gelten $t_{s^l_j, s^{l'}_{j'}}> 0$, $l, l' \in L$, $j = 1, \dots, m_l$, $j' = 1, \dots, m_{l'}$, da die beiden Haltestellen sonst identisch sind.
In der Problemstellung von Schulz \& Vlćek haben alle Busse ein einzelnes Depot $D$ als Start- und Endpunkt der Schicht bzw. dem Betrachtungszeitraum. Die Verbindung vom Depot zu jeder Haltestelle im System muss möglich sein. Für die Fahrzeiten zwischen dem Depot und einer beliebigen Haltestelle der Tour $l$ gilt entsprechend $t_{D, s^l_j} > 0$ und $t_{s^l_j, D} > 0$, $l \in L$, $j = 1, \dots, m_l$. Allgemein gilt die Dreiecksungleichung für alle Fahrtzeiten.
Für die im Modell verwendeten Busse $k \in K = \{1, \dots, |K|\}$ wird eine Kapazität $Q_k$ festgelegt, die angibt wie hoch die zulässige ganzzahlige Anzahl der, gleichzeitig im Bus transportierbaren, Fahrgäste ist.
Die Nachfrage zwischen zwei Haltestellen wird durch $d_{s^l_j, s^l_{j'}}$ mit $l \in L$, sowie $j, j' = 1, \dots, m_l$ und $j' > j$ berücksichtigt.


\section{Dimensionen der Problemstellung - 9 Szenarien}
\textbf{WENN HIER NUR EIN UNTERKAPITEL, DANN KEIN UNTERKAPITEL, SONST NOCH INHALTE FÜR ANDERES UNTERKAPITEL FINDEN}
Schulz \& Vlćek stellen für die systematische Untersuchung der Problemstellung 2 Dimensionen vor, die die Problemstellung wie folgt untergliedern:

\textbf{Operative Komplexität}
\begin{enumerate}
    \item \textbf{Uneingeschränkter autonomer Betrieb}: Es werden homogene, autonome Busse mit unendlicher Kapazität eingesetzt.
    \item \textbf{Kapazitätsbeschränkung im autonomen Betrieb}: Die Busse sind weiterhin autonom, allerdings nun heterogen, durch die Berücksichtigung der individuellen Kapazitätsgrenze $Q_k$.
    \item \textbf{Fahrer- und Kapazitätsbeschränkung}: Es werden Fahrer für die Busse eingesetzt. Die Busse sind somit nicht mehr autonom. Für die Fahrer werden Arbeits- und Pausenzeiten berücksichtigt. Durch die feste Zuweisung von Fahrer zu Bus ergibt sich der Zeitpunkt $a_k$, zu dem der Bus $k \in K$ verfügbar wird, dem Arbeitszeitbeginn des Fahrers. Folglich ergibt sich auch der Zeitpunkt $b_k$, zu dem der Fahrer seine Pause mit der Länge $p$ beginnt und der Zeitpunkt $c_k$, zu dem der Fahrer seine Schicht beendet und ab dem der Bus nicht mehr verfügbar ist. 
\end{enumerate}
\textbf{Strategien zur Nachfragebedienung}
\begin{enumerate}[label=\Alph*)]
    \item \textbf{Bedienung des gesamten Netzes}: Ohne Betrachtung der Nachfrage werden alle Touren vollständig bedient.
    \item \textbf{Bedienung der nachgefragten Touren}: Es werden jene Touren vollständig bedient für die eine (Teil-)Nachfrage besteht.
    \item \textbf{Bedienung der nachgefragten Haltestellen}: Es werden nur die Teile der jeweiligen Touren bedient, für deren Haltestellen zu Beginn des Betrachtungszeitraums eine Nachfrage besteht.
\end{enumerate}

Daraus ergeben sich insgesamt 9 Szenarien 1.A) - 3.C), die durch die Implementierung zu untersuchen sind.

Ziel in jedem der 9 Szenarien ist es die Anzahl der benötigten Busse zu minimieren. Auf diese Weise soll aufgezeigt werden, welcher Einfluss auf die minimale Anzahl der Busse besteht, wenn anstelle vollständiger Touren ausschließlich nachgefragte Haltestellen einer Tour $l$ bedient werden müssen.
Weitere zukünftige Analysemöglichkeiten werden in Kapitel \ref{chapter:6} beschrieben.

%%%%%%%%%%%%%%%%%%%%¸
%%%%%%%%%%%%%%%%%%%%
%%%%%%%%%%%%%%%%%%%%
%%%%%%%%%%%%%%%%%%%%