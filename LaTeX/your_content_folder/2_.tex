%Leitfrage ist: $\rightarrow$ Was ist der Kontext des zu implementierenden Modells?

\chapter{Literaturüberblick}
\label{chapter:2}
Das von Schulz \& Vlćek vorgeschlagene System stellt eine Kombination des klassischen Linienbusverkehrs und Elementen eines, auf Abruf basierenden Ridepooling-Dienstes dar. Das präsentierte mathematische Modell ist im Bereich der OR-Probleme mit der Problemfamilie des Dial-a-Ride Problems und des Vehicle Routing Problems (VRP) verwandt. Eine ausführlichere Einordnung erfolgt in Kapitel \ref{sec:Einordnung}. Mit dem vorgeschlagenen System werden Elemente aus unterschiedlichen Systemen, Problemstellungen und Modellen verknüpft. Um ein einheitliches Verständnis zu schaffen, wird in diesem Kapitel die Definition zentraler Begriffe vorgenommen. Im Anschluss ein Überblick zur Forschung auf den gebieten der zentralen Begrife gegeben, bevor das Modell von Schulz \& Vlćek in die Menge verwandter Untersuchungen eingeordnet wird. Da Schulz \& Vlćek ein öffentliches Bus-System in einer ländlicheren Gegend untersuchen, fokussiert sich die folgende Literaturübersicht auf Forschungsarbeiten die auf den ländlichen Kontext anwendbar sind. \textbf{noch unsicher ob der vorige Satz drinne bleibt} Andere Mobilitätsformen wie zum Beispiel das klassische Taxi oder U-Bahnen werden zur kontextuellen Abgrenzung erwähnt, sind aber kein Teil des Untersuchungsbereiches dieser Arbeit.

\textbf{Methodik} Nachfolgend wird ein Überblick der Forschung auf den Gebieten der zentralen Begriffe in Bezug auf das Verkehrsmittel Bus gegeben. Dabei wird unterschieden, ob die Forschung im ländlichen oder städtischen Raum durchgeführt wurde. Kontexte wie private Bereiche (Universiäts-Campus-Shuttle, etc.) sind nicht Teil des Untersuchungsbereiches. Bei der systematischen Vorgehensweise wurden die Suchbegriffe in englischer Sprache verwendet, da die meisten Veröffentlichungen heutzutage auf englischer Sprache getätigt werden. Die Suchbegriffe wurden miteinander verknüpft und in sogenannten Searchstrings verwendet. Alle verwendeten Veröffentlichungen sind in Anhang \textbf{XXXXX} jenem Searchstring zugeordnet, mit dem sie gefunden wurden. An dieser Stelle sei erwähnt, dass dieselben Veröffentlichungen teilweise mit mehreren Searchstrings gefunden wurden, allerdings nur einem Searchstring zugeordnet wurden. Die Recherche der nachfolgend präsentierten Literatur erfolgte mittels der gebildeten Searchstrings in den wissenschaftlichen Online-Datenbanken Google Schoolar und Scopus. Es wurden nur Veröffentlichungen berücksichtigt, die in einer vollständigen PDF-Version mittels des VPN-Zugangs der Universität Hamburg oder der Helmut-Schmidt-Universität zugänglich waren. Für die, in Kapitel \ref{sec:2.1} präsentierte, Literatur wurden die Searchstrings auf der Plattform Scopus im Rahmen der Funktion \glqq TITLE-ABS-KEY()\grqq{} gesucht. Die, in Kapitel \ref{sec:2.2} genutzte, Literatur zur Einordnung / Abgrenzung des BLSP wurde auf Scopus durch Eingabe der Searchstrings innerhalb der Funktion \glqq TITLE()\grqq{} gefunden. Die Literatur wurde anhand des Titels und Abtracts bewertet und somit für die weitere Analyse und eventuelle spätere Verwendung in dieser Arbeit ausgewählt. 

EVTL. generelle Vorgehnswiese:
1. Suche mit Searchstrings aus Begriffskombinationen der zentralen Begriffe nach generellem Forschungskontext. 2. Suche nach Review-Papern zu den BLSP-verwandten Problemen DARP und VRP. 3. Suche mit Searchstrings nach Veröffentlichungen zu relevanten Varianten des DARP und VRP die mit dem BLSP verwandt sein könnten. 4. Die Herkunft aller Veröffentlichungen die ich benutzen will habe ich in csv-DAteien den jeweiligen Searchstrings mit denen ich sie gefunden habe zugeordnet. 5. Die gefunden Veröffentlichungen habe ich als, bis jetzt zuletzt erfolgten Schritt, nun den (Unter-)Kapiteln zugeordnet in denen ich sie verwenden möchte.


Die in der Arbeit genutzten zentralen Begriffe werden wie folgt definiert:

\textbf{Linienbasiert} (engl.: line-based) bedeutet für das ÖPNV-System die Einhaltung von festen Routen, festgelegten Haltestellen und Fahrplänen wie es von städtischen Bus-, Straßenbahn- und U-Bahn-Netzen bekannt ist.

\textbf{Auf Abruf} (engl.: on-demand) ist die Eigenschaft eines Systems auf die realisierte Nachfrage der Fahrgäste zu reagieren \parencite[vgl.][S.3]{vansteenwegen_survey_2022}. Dies kann in Echtzeit oder auch vorverarbeitet in einem zuvor abgegrenzten Zeitraum erfolgen. Die Nachfrage wird dann innerhalb eines festgelegten Zeitraums bedient. Zeiträume können sich überschneiden und so zum Beispiel die Form eines rollierenden Horizonts annehmen.

\textbf{Hier evtl. nochmal Klarstellung, dass on-demand zwar demand-responsive ist, aber nicht andersrum, wie fälschlicherweise von \textcite{wang_multilevel_2014} in ihrem Paper behauptet}

\textbf{Ridepooling} wird vom \textcite{verband_der_automobilindustrie_ridepooling_2025} (\textbf{Achtung online Quelle} - Schulz \& Vlćek nutzen hier \parencite{vansteenwegen_survey_2022}) als kombinierte Mobilitätsform von ÖPNV und Taxi definiert. Der Ablauf bei dieser besonderen Form des Personenverkehrswerden ist wie folgt: Der Fahrgast wird an seinem individuellen Startpunkt abgeholt. Auf dem Weg zum individuellen Ziel steigen andere Fahrgäste zu und/oder wieder aus. Je nach Zielpunkt werden die Routen so kombiniert, dass benachbarte Ziele mit einem Fahrzeug bedient werden können.   
%%%%%%%%%%%%%%%%%%%%
%%%%%%%%%%%%%%%%%%%%
%%%%%%%%%%%%%%%%%%%%
%%%%%%%%%%%%%%%%%%%%
\section{Forschung auf den Gebieten der zentrale Begriffe}
\label{sec:2.1}
\label{sec:Kontext}

\textbf{GLEIDERUNG DIESES ABSCHNITTS:}
\begin{enumerate}
    \item Fachliche Einleitung
    \begin{itemize}
        \item gnerelle relevante Statements
        \item auf andere Forschungsaspekte hinweisen, die nicht weiter betrachtet wurden
    \end{itemize}
    \item Präsentation der Forschungsergebnisse durch Tabellen
\end{enumerate}

\textbf{An dieser Stelle sei erwähnt, dass in diesem Kapitel zunächst die Breite der Forschung dargestellt wird, bevor in Kapitel \ref{sec:2.2} das Modell von Schulz \& Vlćek eingeordnet wird.}

Anwendungsbereiche \& typische Zielkonflikte darstellen (evtl. Grafik dazu anfertigen: Taxi als flexibelstes on-demand, keine festen Routen und krasses Gegenteil ist Linienbus mit festem Fahrplan und Routen)

line-based sind einfach der backbone in urban areas und sind, wenn nicht asl einziges, dann das teil des zentralen public transport networks um hohe Nachfrage kosteneffizient zu bedienen (hinweis Richtung Forschung zu feeder-networks). werden verwendet als Schulbusse, Schienenersatzverkehr, ...

on-demand häufig als "Feeder" zu den hoch-volumigen Systemen wie Metro oder Line-based buses

Generell Zielkonflikt von echten on-demand Lösungen in rural areas: geringere Kosten (durch weniger Fahrzeuge) vs. Abdeckung(?) Soll darauf hinführen, dass Schulz \& Vlćek weniger Busse nutzen bei Beibehaltung der Flexibilität.

Aussage darüber, dass viel zu "on-demand rural" mit "line-based rural" gefunden wurde -> zeigt enge verbundenheit der begriffe usw.

Es ist ersichtlich, das dieses Thema seit mehr als 20 Jahren eine große Aufmerksamkeit genießt. IN 2003 haben Quelle bereits eine Übersicht erstellt. 2014 noch eine und die eneuste 2022 von Quelle

Ridepooling: dem zugrundeliegend ist das DARP -> wofür entwickelt (Quelle), dann überführen auf Busse (Quelle) siehe Schulz \& Vlćek, \textbf{ganze wichtig: reiter et al. 2024} - das beschreiben Schulz \& Vlćek als sehr nah dran, aber Schulz \& Vlćek berücksichtigen einen vorgegebenen Fahrplan. 

Ridepooling: Synonyme Verwendung der Begriffe Ride-sharing, ridepooling usw

\begin{landscape} 
    \begin{table}[p]
        \centering
        \caption{Forschungsergebnisse zu Line-based — rural}
        \label{tab:lb-rural}
        \scriptsize
        \setlength{\tabcolsep}{2.5pt}
        \setlength{\arrayrulewidth}{0.1pt}

        \resizebox{\textheight}{!}{%
        \begin{tabular}{
          L{2.0cm}:
          L{2.5cm}:
          L{2.0cm}:
          L{2.0cm}:
          L{2.5cm}:
          L{2.3cm}:
          L{1.8cm}:
          L{2.6cm}:
          L{7.4cm}:
        }
          \hline
          \LH{Paper} & \LH{Zielsetzung} & \LH{Region /\\Land} & \LH{Betrachtungs-\\ebene} &
          \LH{Fokus / \\ Anwendungsfeld} & \LH{Methode} & \LH{Daten} & \LH{KPI} & \LH{Zentrale Erkenntnisse} \\
          \Xhline{0.6pt}

            \textcite{cirella_transport_2019} & Review: Innovationen für Ältere & ohne Bezug & Region & Ältere & Literatur-Review & Literatur & — & Ältere brauchen Innovationen bei Angebot, Infrastruktur und Fahrzeugen; Umsetzungs- und Barrierefreiheitslücken bestehen. \\ \hline
            \textcite{das_planning_2012} & Planung ländlicher Feederlinien & Indien & Linie & Feeder & Analytische Näherung / Empirie & Umfrage & Ø-Wartezeit / Gehzeit & Generalisiert gewichtete Geh-/Warte- und Komfortfaktoren liefern praktikable Linienführungen und Takte für ländliche Feeder. \\ \hline
            \textcite{guiver_buses_2007} & Bewertung touristischer Buswirkungen & UK & Region & Tourismus & Empirie & Umfrage & Modal-Shift / Erreichbarkeit & Geplante Busse bewirken moderaten Modal-Shift, ermöglichen Zugang ohne Auto und stützen lokale Ökonomien; Fördermittel bleiben problematisch. \\ \hline
            \textcite{petersen_watching_2016} & Netzansatz mit Taktfahrplan & Schweiz & Netz & Taktfahrplan & Fallstudie / Literatur-Review & Literatur & — & In ländlichen Räumen bieten integrierte Taktfahrplan-Netze verlässliche Anschlüsse und sollten vor DRT als Basisstrategie geprüft werden. \\ \hline
            \textcite{takamatsu_bus_2020} & Optimierung von Umsteigezeiten & Japan & Netz & Taktfahrplan & Matem. Optimierungsmodell / Fallstudie & Realdaten & Ø-Wartezeit / Umsteigezeit & Optimierte Takte/Anschlüsse verkürzen Wartezeiten deutlich; Tohoku-Fall bestätigt Praxistauglichkeit bei niedriger Angebotsdichte. \\ \hline
            \textcite{tsigdinos_route_2024} & Strategie flexibler Busdienste & Griechenland & Region & Flexible Busse & Empirie / Kartenbasierte Raumanalyse (GIS) & Umfrage & — & Befragung+GIS zeigen: Kosten und Zeit dominieren; morgens wird Tür-zu-Tür, nachmittags Haltestellenbetrieb bevorzugt. \\ \hline
            \textcite{zhen_feeder_2024} & CA-Design heterogener Feeder & ohne Bezug & Netz & Feeder & Analytische Näherung / Matem. Optimierungsmodell & synthetisch & Reisezeit / Kosten/Fahrgast-km & Koordinierte Zubringer-/Stammfahrpläne senken Nutzerkosten und verbessern Systemleistung bei heterogener Nachfrage. \\ \hline
        \end{tabular}
        }% end resizebox
    \end{table}

    \begin{table}[p]
        \centering
        \caption{Forschungsergebnisse zu Line-based — urban}
        \label{tab:lb-urban}
        \scriptsize
        \setlength{\tabcolsep}{2.5pt}
        \setlength{\arrayrulewidth}{0.1pt}

        \resizebox{\textheight}{!}{%
        \begin{tabular}{
            L{2.0cm}:
            L{2.5cm}:
            L{2.0cm}:
            L{2.0cm}:
            L{2.5cm}:
            L{2.3cm}:
            L{1.8cm}:
            L{2.6cm}:
            L{7.4cm}:
        }
          \hline
          \LH{Paper} & \LH{Zielsetzung} & \LH{Region /\\Land} & \LH{Betrachtungs-\\ebene} &
          \LH{Fokus / \\ Anwendungsfeld} & \LH{Methode} & \LH{Daten} & \LH{KPI} & \LH{Zentrale Erkenntnisse} \\
          \Xhline{0.6pt}

          \textcite{fielbaum_improving_2024} & Integration Ridepooling & — & Linie & Ridepooling & Matem. Optimierungsmodell / Simulation & — & Reisezeit / Fahrzeug-km & zeigt Vorteile einer Ridepooling-Kopplung mit Linien; senkt Betriebskosten; senkt Emissionen. \\ \hline
          \textcite{filippi_exploiting_2024} & Linienoptimierung & Italy & Linie & Modularbus & Matem. Optimierungsmodell & — & Reisezeit / Auslastung & nutzt modulare Kopplung für flexible Kapazität; verkürzt Reisezeiten und verbessert Auslastung. \\ \hline
          \textcite{gal_traveling_2017} & Reisezeitprognose & ohne Bezug & Linie & Taktfahrplan & Empirie & Realdaten & Reisezeit / Prognosegüte & liefert praxistaugliche Prognosen der Reisezeiten im Linienverkehr (gute Prognosegüten). \\ \hline
          \textcite{hatzenbuhler_network_2022} & Netzdesign & ohne Bezug & Netz & Autonom & Matem. Optimierungsmodell / Simulation & — & Ø-Wartezeit / Reisezeit & zeigt, wie autonome Busnetz-Designs Nutzer- und Betreiberkosten senken können. \\ \hline
          \textcite{jimenez_urban_2016} & Flottenzuordnung & Spain & Linie & Emissionen & Matem. Optimierungsmodell & Mix & CO\(_2\) / Fahrzeug-km & weist nach, dass optimierte Flottenzuordnung Emissionen im Liniennetz deutlich reduzieren kann. \\ \hline
          \textcite{kim_integrating_2013} & Integration gemischter Flotten & USA & Linie & Flotten & Matem. Optimierungsmodell & — & Kosten/Fahrgast-km / Auslastung & kombinierte feste und flexible Dienste mit gemischten Flotten senken Kosten und erhöhen Auslastung. \\ \hline
          \textcite{rashvand_real-time_2024} & Ankunftsprognose & USA & Linie & KI & Empirie & Realdaten & Prognosegüte / Pünktlichkeit & Deep-Learning-Ansätze liefern robuste Ankunftsprognosen und verbessern die wahrgenommene Pünktlichkeit. \\ \hline
          \textcite{rosca_designing_2024} & Fahrplanoptimierung & Romania & Linie & Flotten & Empirie / Simulation & — & Pünktlichkeit / Kosten/Fahrgast-km & KI-gestützte Planung verbessert Pünktlichkeit und senkt Planungskosten in urbanen Netzen. \\ \hline
          \textcite{tang_data-driven_2021} & Fahrplanoptimierung & China & Linie & Taktfahrplan & Matem. Optimierungsmodell & Realdaten & Ø-Wartezeit / Reisezeit & zweckmäßige MOEA-Fahrpläne verkürzen Warte- und Reisezeiten auf städtischen Linien. \\ \hline
          \textcite{tian_autonomous_2021} & Flottenoptimierung & China & Linie & Autonom & Matem. Optimierungsmodell & — & Kosten/Fahrgast-km / Flottengröße & autonome Flotten mit Unsicherheiten können Betreiberkosten senken und Leistung stabilisieren. \\ \hline
          \textcite{wei_optimizing_2020} & Integration Metro-Bus & China & Netz & Integration & Matem. Optimierungsmodell / Fallstudie & Realdaten & Ø-Wartezeit / Reisezeit & koordinierte Metro-Bus-Planung senkt Nutzerzeiten und verbessert Gesamtleistung. \\ \hline
        \end{tabular}
        }% end resizebox
    \end{table}
\end{landscape}


\begin{landscape}

    \scriptsize
    \setlength{\tabcolsep}{2.2pt}
    \setlength{\arrayrulewidth}{0.1pt}
    \begin{xltabular}{\textwidth}{%
        L{1.8cm}:
        L{1.8cm}:
        L{1.5cm}:
        L{2.3cm}:
        L{2.3cm}:
        L{2.0cm}:
        L{1.5cm}:
        L{2.4cm}:
        L{6.0cm}:
    }
        \caption{Forschungsergebnisse zu on-demand — rural}\label{tab:od-rural}\\ 
        \hline
        \LH{Paper} & \LH{Zielsetzung} & \LH{Region /\\Land} & \LH{Betrachtungs-\\ebene} &
        \LH{Fokus / \\ Anwendungsfeld} & \LH{Methode} & \LH{Daten} & \LH{KPI} & \LH{Zentrale\\Erkenntnisse} \\
        \Xhline{0.6pt}
        \endfirsthead

        \multicolumn{9}{l}{\small\itshape Fortsetzung von Tabelle~\ref{tab:od-rural}.}\\[0.6\baselineskip]
        \hline
        \LH{Paper} & \LH{Zielsetzung} & \LH{Region /\\Land} & \LH{Betrachtungs-\\ebene} &
        \LH{Fokus / \\ Anwendungsfeld} & \LH{Methode} & \LH{Daten} & \LH{KPI} & \LH{Zentrale\\Erkenntnisse} \\
        \Xhline{0.6pt}
        \endhead

        \hline
        \multicolumn{9}{r}{\small\itshape Fortsetzung auf der nächsten Seite}
        \endfoot

        \hline
        \endlastfoot

        \textcite{bauchinger_developing_2021} & Konnektivität verbessern & Europa & Region & Feeder, Multimodal & Fallstudie, Empirie & Mix & Erreichbarkeit, Umsteigezeit & Komplementäre Dienste (DRT, multimodal) verbessern Erreichbarkeit und ÖPNV-Anbindung in ländlich--urbanen Regionen. \\ \hline
        
        \textcite{brake_demand_2004} & DRT-Markt prüfen & UK & Region & Inklusion & Fallstudie & Realdaten & — & Britische DRT-Erfahrungen fördern soziale Inklusion und Intermodalität; Umsetzung wird durch Finanzierungs- und Betriebsfragen begrenzt. \\ \hline
        
        \textcite{calabro_designing_2023} & DRT-Bedingungen vergleichen & Italien & Region & Kleinstädte & Simulation & synthetisch & Ø-Wartezeit, Auslastung & ABM zeigt: DRT ist FRT in Kleinstädten überlegen, wenn Umwege, Warte- und Gehzeiten minimiert und Teilfahrten gebündelt werden. \\ \hline
        
        \textcite{coutinho_impacts_2020} & Linienersatz bewerten & Niederlande & Linie & Linienersatz & Empirie & Realdaten & Fahrgast-km, CO\(_2\) & Im Mokumflex-Pilotprojekt sanken Fahrgast-km, Kosten und Emissionen pro Fahrgast deutlich, jedoch ging die Nachfrage stark zurück. \\ \hline
        
        \textcite{dorso_transforming_2025} & DRT-Effekte bewerten & Italien & Region & Pilotstudie & Empirie, Simulation & Mix & Ø-Wartezeit, Reisezeit & DRT steigerte Erreichbarkeit und senkte Warte- und Fahrzeiten im suburbanen Palermo; wirtschaftliche und rechtliche Rahmenbedingungen bleiben kritisch. \\ \hline
        
        \textcite{daduna_evolution_2020} & Entwicklung skizzieren & ohne Bezug & Region & Autonom/Digital & Literatur-Review & Literatur & — & Autonome Fahrzeuge und Digitalisierung können Kostenstrukturen ländlicher ÖPNV-Angebote grundlegend verändern, ersetzen Linienverkehr jedoch nicht vollständig. \\ \hline
        
        \textcite{daniels_flexible_2012} & Barrieren identifizieren & Australien & Region & Implementierung & Empirie & Umfrage & — & Fünf Barrierefelder (Regulierung, Finanzierung, Betrieb, Einstellungen, Information) hemmen FTS-Einführung in Niedrigdichtegebieten. \\ \hline
        
        \textcite{dytckov_potential_2022} & Ruralen DRT bewerten & Dänemark & Region & Studierende & Simulation & — & Ø-Wartezeit, Auslastung & Studie zeigt Potenziale und Grenzen von DRT im ländlichen Kontext. \\ \hline
        
        \textcite{galarza_montenegro_large_2021} & Feeder-Service optimieren & ohne Bezug & Linie & Feeder & Matem. Optimierungsmodell & — & Kosten/Fahrgast-km, Auslastung & Studie zeigt Potenziale und Grenzen von DRT im ländlichen Kontext. \\ \hline
        
        \textcite{galarza_montenegro_demand-responsive_2024} & Feeder-Service optimieren & ohne Bezug & Linie & Feeder & Matem. Optimierungsmodell & — & Ø-Wartezeit, Kosten/Fahrgast-km & Ansatz verbessert Erreichbarkeit und Servicegrad gegenüber Status quo. \\ \hline

        \textcite{guo_modular_2023} & Fahrplan/Zuteilung optimieren & China & Netz & Studierende & Matem. Optimierungsmodell, Empirie & — & Auslastung, Erreichbarkeit & Ansatz verbessert Erreichbarkeit und Servicegrad gegenüber Status quo. \\ \hline
        
        \textcite{jiang_integrated_2025} & Fahrplan/Zuteilung optimieren & ohne Bezug & Netz & Studierende & Simulation, Matem. Optimierungsmodell & Realdaten & Reisezeit, Flottengröße & Optimierung/Szenarien senken Warte- und Reisezeiten im ländlichen DRT deutlich. \\ \hline

        \textcite{knierim_attitude_2021} & DRT-Konzept bewerten & Deutschland & Linie & Studierende & Empirie & Mix & Auslastung & Ansatz verbessert Erreichbarkeit und Servicegrad gegenüber Status quo. \\ \hline
        
        \textcite{lu_demand-responsive_2023} & Ruralen DRT bewerten & Deutschland & Linie & Studierende & Simulation, Empirie & Realdaten & Flottengröße, Kosten/Fahrgast-km & Studie zeigt Potenziale und Grenzen von DRT im ländlichen Kontext. \\ \hline
        
        \textcite{marti_optimization_2023} & Ruralen DRT bewerten & Schweiz & Region & Flexible Busse & Matem. Optimierungsmodell & — & Auslastung & Studie zeigt Potenziale und Grenzen von DRT im ländlichen Kontext. \\ \hline
        
        \textcite{marti_flexible_2024} & Ruralen DRT bewerten & Spanien & Netz & Schule & — & — & Auslastung & Studie zeigt Potenziale und Grenzen von DRT im ländlichen Kontext. \\ \hline
        
        \textcite{mulley_flexible_2009} & Ruralen DRT bewerten & UK & Region & Flexible Busse & Kartenbasierte Raumanalyse (GIS) & — & Auslastung, Erreichbarkeit & Studie zeigt Potenziale und Grenzen von DRT im ländlichen Kontext. \\ \hline
        
        \textcite{nourbakhsh_structured_2012} & DRT-Konzept bewerten & USA & Linie & Flexible Busse & — & — & Auslastung & Studie zeigt Potenziale und Grenzen von DRT im ländlichen Kontext. \\ \hline
        
        \textcite{papanikolaou_analytical_2021} & DRT-Konzept bewerten & ohne Bezug & Region & Flexible Busse & Analytische Näherung & — & Auslastung & Studie zeigt Potenziale und Grenzen von DRT im ländlichen Kontext. \\ \hline
        
        \textcite{poltimae_search_2022} & Ruralen DRT bewerten & ohne Bezug & Region & Flexible Busse & Literatur-Review & Literatur & — & Studie zeigt Potenziale und Grenzen von DRT im ländlichen Kontext. \\ \hline
        
        \textcite{schluter_impact_2021} & Autonomen DRT planen & Deutschland & Netz & Autonom & — & — & Auslastung & Studie zeigt Potenziale und Grenzen von DRT im ländlichen Kontext. \\ \hline
        
        \textcite{sieber_improved_2020} & Ruralen DRT bewerten & Schweiz & Netz & Autonom & Simulation & — & Auslastung & Ansatz verbessert Erreichbarkeit und Servicegrad gegenüber Status quo. \\ \hline
        
        \textcite{torrisi_evaluation_2025} & DRT-Konzept bewerten & Italien & Region & Flexible Busse & — & — & Auslastung & Studie zeigt Potenziale und Grenzen von DRT im ländlichen Kontext. \\ \hline
        
        \textcite{velaga_potential_2012} & Ruralen DRT bewerten & UK & Netz & Flexible Busse & Literatur-Review & Literatur & — & Ansatz verbessert Erreichbarkeit und Servicegrad gegenüber Status quo. \\ \hline
        
        \textcite{viergutz_demand_2019} & DRT-Konzept bewerten & Deutschland & Netz & Flexible Busse & Literatur-Review & Literatur & Auslastung & Studie zeigt Potenziale und Grenzen von DRT im ländlichen Kontext. \\ \hline
        
        \textcite{wang_multilevel_2014} & DRT-Konzept bewerten & ohne Bezug & Linie & Flexible Busse & — & — & Auslastung & Ansatz verbessert Erreichbarkeit und Servicegrad gegenüber Status quo. \\ \hline
        
        \textcite{wang_planning_2025} & DRT-Konzept bewerten & Frankreich & Netz & Flexible Busse & — & — & Auslastung, Erreichbarkeit & Optimierung/Szenarien senken Warte- und Reisezeiten im ländlichen DRT deutlich. \\ \hline
        
        \textcite{white_roles_2016} & Rollen vergleichen & UK & Region & Kostenvergleich & Literatur-Review & Literatur & Kosten/Fahrgast-km, Auslastung & Konventionelle Interurban-Linien sind oft kosteneffizienter; DRT weist häufig hohe Kosten je Fahrt auf und eignet sich eher für Zielgruppen-/Mindestbedienung. \\ \hline
    \end{xltabular}
\end{landscape}

\begin{landscape}

    \scriptsize
    \setlength{\tabcolsep}{2.2pt}
    \setlength{\arrayrulewidth}{0.1pt}
    \begin{xltabular}{\textwidth}{%
        L{1.8cm}:
        L{1.8cm}:
        L{1.5cm}:
        L{2.3cm}:
        L{2.3cm}:
        L{2.0cm}:
        L{1.5cm}:
        L{2.4cm}:
        L{6.0cm}:
    }
        \caption{Forschungsergebnisse zu on-demand — urban}\label{tab:od-urban}\\ 
        \hline
        \LH{Paper} & \LH{Zielsetzung} & \LH{Region /\\Land} & \LH{Betrachtungs-\\ebene} &
        \LH{Fokus / \\ Anwendungsfeld} & \LH{Methode} & \LH{Daten} & \LH{KPI} & \LH{Zentrale\\Erkenntnisse} \\
        \Xhline{0.6pt}
        \endfirsthead

        \multicolumn{9}{l}{\small\itshape Fortsetzung von Tabelle~\ref{tab:od-urban}.}\\[0.6\baselineskip]
        \hline
        \LH{Paper} & \LH{Zielsetzung} & \LH{Region /\\Land} & \LH{Betrachtungs-\\ebene} &
        \LH{Fokus / \\ Anwendungsfeld} & \LH{Methode} & \LH{Daten} & \LH{KPI} & \LH{Zentrale\\Erkenntnisse} \\
        \Xhline{0.6pt}
        \endhead

        \hline
        \multicolumn{9}{r}{\small\itshape Fortsetzung auf der nächsten Seite}
        \endfoot

        \hline
        \endlastfoot


        \textcite{abe_introducing_2019} & Autonomen DRT planen & UK & Netz & Autonom & Matem. Optimierungsmodell & — & Kosten/Fahrgast-km, Erreichbarkeit & Optimierung/Szenarien senken Warte- und Reisezeiten im urbanen DRT deutlich. \\ \hline
        
        \textcite{alonso-gonzalez_potential_2018} & DRT-Konzept bewerten & ohne Bezug & Linie & Flexible Busse & Matem. Optimierungsmodell, Empirie & — & Auslastung & Ansatz verbessert Erreichbarkeit und Servicegrad gegenüber Status quo. \\ \hline
        
        \textcite{charisis_drt_2018} & First/Last-Mile verbessern & Deutschland & Linie & First/Last-Mile & — & — & Auslastung, Erreichbarkeit & Optimierung/Szenarien senken Warte- und Reisezeiten im urbanen DRT deutlich. \\ \hline
        
        \textcite{diana_methodology_2009} & DRT-Konzept bewerten & Frankreich & Linie & Flexible Busse & Matem. Optimierungsmodell & — & Auslastung & Ansatz verbessert Erreichbarkeit und Servicegrad gegenüber Status quo. \\ \hline
        
        \textcite{giuffrida_addressing_2021} & DRT-Konzept bewerten & Italien & Region & Flexible Busse & Matem. Optimierungsmodell, Kartenbasierte Raumanalyse (GIS) & Realdaten & Auslastung, Erreichbarkeit & Studie zeigt Potenziale und Grenzen von DRT im urbanen Kontext. \\ \hline
        
        \textcite{hazan_-demand_nodate} & Urbanen DRT bewerten & ohne Bezug & Region & Flexible Busse & Matem. Optimierungsmodell & — & Kosten/Fahrgast-km, Auslastung & Ansatz verbessert Erreichbarkeit und Servicegrad gegenüber Status quo. \\ \hline
        
        \textcite{huang_flexible_2020} & DRT-Konzept bewerten & ohne Bezug & Linie & Flexible Busse & Simulation, Matem. Optimierungsmodell & — & Auslastung & Studie zeigt Potenziale und Grenzen von DRT im urbanen Kontext. \\ \hline
        
        \textcite{inturri_taxi_2021} & DRT per Simulation prüfen & Italien & Netz & Flexible Busse & Simulation & — & Auslastung & Studie zeigt Potenziale und Grenzen von DRT im urbanen Kontext. \\ \hline
        
        \textcite{jager_multi-agent_2018} & Autonomen DRT planen & ohne Bezug & Region & Autonom & Matem. Optimierungsmodell & — & Auslastung & Optimierung/Szenarien senken Warte- und Reisezeiten im urbanen DRT deutlich. \\ \hline
        
        \textcite{kim_optimization_2025} & DRT-Konzept bewerten & Schweiz & Region & Flexible Busse & Matem. Optimierungsmodell & — & Auslastung & Studie zeigt Potenziale und Grenzen von DRT im urbanen Kontext. \\ \hline
        
        \textcite{li_efficient_2024} & DRT-Konzept bewerten & Korea & Linie & Flexible Busse & — & — & Auslastung & Studie zeigt Potenziale und Grenzen von DRT im urbanen Kontext. \\ \hline
        
        \textcite{liyanage_agent-based_2020} & DRT per Simulation prüfen & Australien & Region & Flexible Busse & Simulation & — & Auslastung, CO\(_2\) & Optimierung/Szenarien senken Warte- und Reisezeiten im urbanen DRT deutlich. \\ \hline
        
        \textcite{melo_demand-responsive_2024} & First/Last-Mile verbessern & Schweiz & Netz & First/Last-Mile & — & — & Auslastung & Ansatz verbessert Erreichbarkeit und Servicegrad gegenüber Status quo. \\ \hline
        
        \textcite{meshkani_innovative_2024} & First/Last-Mile verbessern & ohne Bezug & Region & First/Last-Mile & Matem. Optimierungsmodell & — & Auslastung & Studie zeigt Potenziale und Grenzen von DRT im urbanen Kontext. \\ \hline
        
        \textcite{mortazavi_integrated_2024} & DRT-Konzept bewerten & Australien & Netz & Flexible Busse & Matem. Optimierungsmodell, Fallstudie & — & Reisezeit, Kosten/Fahrgast-km & Studie zeigt Potenziale und Grenzen von DRT im urbanen Kontext. \\ \hline
        
        \textcite{pal_smartporter_2016} & Urbanen DRT bewerten & ohne Bezug & Netz & Flexible Busse & Kartenbasierte Raumanalyse (GIS) & — & — & Optimierung/Szenarien senken Warte- und Reisezeiten im urbanen DRT deutlich. \\ \hline

        \textcite{queiroz_instance_2024} & DRT-Konzept bewerten & Frankreich & Linie & Flexible Busse & Matem. Optimierungsmodell & — & Auslastung & Studie zeigt Potenziale und Grenzen von DRT im urbanen Kontext. \\ \hline
        
        \textcite{shinoda_is_2004} & DRT per Simulation prüfen & UK & Linie & Flexible Busse & Simulation, Matem. Optimierungsmodell & — & — & Ansatz verbessert Erreichbarkeit und Servicegrad gegenüber Status quo. \\ \hline
        
        \textcite{sun_research_2025} & DRT-Konzept bewerten & China & Region & Flexible Busse & Matem. Optimierungsmodell & — & Flottengröße, Kosten/Fahrgast-km & Studie zeigt Potenziale und Grenzen von DRT im urbanen Kontext. \\ \hline
        
        \textcite{torrisi_evaluation_2025} & Urbanen DRT bewerten & Italien & Region & Flexible Busse & Matem. Optimierungsmodell & — & Auslastung & Studie zeigt Potenziale und Grenzen von DRT im urbanen Kontext. \\ \hline
        
        \textcite{yan_integrating_2019} & Urbanen DRT bewerten & USA & Netz & First/Last-Mile & Matem. Optimierungsmodell & — & Auslastung & Ansatz verbessert Erreichbarkeit und Servicegrad gegenüber Status quo. \\ \hline
        
        \textcite{yeon_real-time_2025} & DRT-Konzept bewerten & Korea & Linie & Flexible Busse & Matem. Optimierungsmodell, Kartenbasierte Raumanalyse (GIS) & — & Auslastung & Studie zeigt Potenziale und Grenzen von DRT im urbanen Kontext. \\ \hline
        
        \textcite{yi_real-time_2025} & Feeder-Service optimieren & USA & Linie & Feeder & Matem. Optimierungsmodell & — & Auslastung & Studie zeigt Potenziale und Grenzen von DRT im urbanen Kontext. \\ \hline
        
        \textcite{zhang_routing_2024} & Fahrplan/Zuteilung optimieren & China & Linie & Flexible Busse & Matem. Optimierungsmodell & — & Auslastung & Ansatz verbessert Erreichbarkeit und Servicegrad gegenüber Status quo. \\ \hline
    \end{xltabular}
\end{landscape}

\begin{landscape}

    \scriptsize
    \setlength{\tabcolsep}{2.2pt}
    \setlength{\arrayrulewidth}{0.1pt}
    \begin{xltabular}{\textwidth}{%
        L{1.8cm}:
        L{1.8cm}:
        L{1.5cm}:
        L{2.3cm}:
        L{2.3cm}:
        L{2.0cm}:
        L{1.5cm}:
        L{2.4cm}:
        L{6.0cm}:
    }
        \caption{Forschungsergebnisse zu ridepooling — rural}\label{tab:rp-rural}\\ 
        \hline
        \LH{Paper} & \LH{Zielsetzung} & \LH{Region /\\Land} & \LH{Betrachtungs-\\ebene} &
        \LH{Fokus / \\ Anwendungsfeld} & \LH{Methode} & \LH{Daten} & \LH{KPI} & \LH{Zentrale\\Erkenntnisse} \\
        \Xhline{0.6pt}
        \endfirsthead

        \multicolumn{9}{l}{\small\itshape Fortsetzung von Tabelle~\ref{tab:rp-rural}.}\\[0.6\baselineskip]
        \hline
        \LH{Paper} & \LH{Zielsetzung} & \LH{Region /\\Land} & \LH{Betrachtungs-\\ebene} &
        \LH{Fokus / \\ Anwendungsfeld} & \LH{Methode} & \LH{Daten} & \LH{KPI} & \LH{Zentrale\\Erkenntnisse} \\
        \Xhline{0.6pt}
        \endhead

        \hline
        \multicolumn{9}{r}{\small\itshape Fortsetzung auf der nächsten Seite}
        \endfoot

        \hline
        \endlastfoot

        \textcite{elting_potential_2021} & Ridepooling-Konzept bewerten & ohne Bezug & Region & Flexible Busse & Matem. Optimierungsmodell, Literatur-Review & Literatur & — & Studie zeigt Potenziale und Grenzen von Ridepooling im ländlichen Kontext. \\ \hline
        
        \textcite{heinitz_flexible_2022} & Rurales Ridepooling bewerten & Deutschland & Linie & Flexible Busse & Simulation, Kartenbasierte Raumanalyse (GIS) & synthetisch & Auslastung & Pooling und Optimierung senken Warte- und Reisezeiten sowie Fahrzeug-km im ländlichen Raum. \\ \hline
        
        \textcite{liu_filtering_2024} & Autonomen DRT planen & Frankreich & Linie & Autonom & Matem. Optimierungsmodell & — & — & Studie zeigt Potenziale und Grenzen von Ridepooling im ländlichen Kontext. \\ \hline
        
        \textcite{patricio_assessing_2025} & Ridepooling simulieren & ohne Bezug & Netz & Flexible Busse & Simulation & — & Kosten/Fahrgast-km, Auslastung & Pooling und Optimierung senken Warte- und Reisezeiten sowie Fahrzeug-km im ländlichen Raum. \\ \hline
        
        \textcite{schaefer_acceptance_2022} & Rurales Ridepooling bewerten & Schweiz & Netz & Flexible Busse & — & — & — & Studie zeigt Potenziale und Grenzen von Ridepooling im ländlichen Kontext. \\ \hline
        
        \textcite{si_vehicle_2024} & Pooling/Zuteilung optimieren & USA & Linie & Pooling & Matem. Optimierungsmodell & — & Auslastung & Studie zeigt Potenziale und Grenzen von Ridepooling im ländlichen Kontext. \\ \hline

        \textcite{sorensen_how_2021} & Rurales Ridepooling bewerten & Deutschland & Netz & Flexible Busse & — & — & — & Studie zeigt Potenziale und Grenzen von Ridepooling im ländlichen Kontext. \\ \hline
        
        \textcite{truden_analysis_2021} & First/Last-Mile verbessern & Österreich & Region & First/Last-Mile & Simulation, Matem. Optimierungsmodell & synthetisch & Kosten/Fahrgast-km, Auslastung & Studie zeigt Potenziale und Grenzen von Ridepooling im ländlichen Kontext. \\ \hline
        
        \textcite{yu_optimal_2021} & Fahrplan/Zuteilung optimieren & USA & Linie & Flexible Busse & Literatur-Review & Literatur & Auslastung & Studie zeigt Potenziale und Grenzen von Ridepooling im ländlichen Kontext. \\ \hline
        
        \textcite{zhou_bus-pooling_2025} & Pooling/Zuteilung optimieren & ohne Bezug & Linie & Pooling & Matem. Optimierungsmodell & Realdaten & Ø-Wartezeit, Kosten/Fahrgast-km & Pooling und Optimierung senken Warte- und Reisezeiten sowie Fahrzeug-km im ländlichen Raum. \\ \hline
        
        \textcite{zwick_ride-pooling_2021} & Ridepooling-Konzept bewerten & Polen & Region & Pooling & Simulation, Literatur-Review & Literatur & — & Studie zeigt Potenziale und Grenzen von Ridepooling im ländlichen Kontext. \\ \hline
    \end{xltabular}
\end{landscape}

\begin{landscape}

    \scriptsize
    \setlength{\tabcolsep}{2.2pt}
    \setlength{\arrayrulewidth}{0.1pt}
    \begin{xltabular}{\textwidth}{%
        L{1.8cm}:
        L{1.8cm}:
        L{1.5cm}:
        L{2.3cm}:
        L{2.3cm}:
        L{2.0cm}:
        L{1.5cm}:
        L{2.4cm}:
        L{6.0cm}:
    }
        \caption{Forschungsergebnisse zu ridepooling — urban}\label{tab:rp-urban}\\ 
        \hline
        \LH{Paper} & \LH{Zielsetzung} & \LH{Region /\\Land} & \LH{Betrachtungs-\\ebene} &
        \LH{Fokus / \\ Anwendungsfeld} & \LH{Methode} & \LH{Daten} & \LH{KPI} & \LH{Zentrale\\Erkenntnisse} \\
        \Xhline{0.6pt}
        \endfirsthead

        \multicolumn{9}{l}{\small\itshape Fortsetzung von Tabelle~\ref{tab:rp-urban}.}\\[0.6\baselineskip]
        \hline
        \LH{Paper} & \LH{Zielsetzung} & \LH{Region /\\Land} & \LH{Betrachtungs-\\ebene} &
        \LH{Fokus / \\ Anwendungsfeld} & \LH{Methode} & \LH{Daten} & \LH{KPI} & \LH{Zentrale\\Erkenntnisse} \\
        \Xhline{0.6pt}
        \endhead

        \hline
        \multicolumn{9}{r}{\small\itshape Fortsetzung auf der nächsten Seite}
        \endfoot

        \hline
        \endlastfoot

        \textcite{agatz_dynamic_2011} & Ridepooling simulieren & Niederlande & Netz & Feeder & Simulation, Matem. Optimierungsmodell & — & Kosten/Fahrgast-km, Auslastung & Pooling und Optimierung reduzieren Warte-/Reisezeiten und Fahrzeug-km im urbanen Kontext deutlich. \\ \hline
        
        \textcite{alonso-gonzalez_value_2020} & Urbanes Ridepooling bewerten & Niederlande & Region & Flexible Busse & Matem. Optimierungsmodell & — & Kosten/Fahrgast-km, Auslastung & Zielkonflikt zwischen Servicequalität und Bündelungsgrad wird sichtbar. \\ \hline
        
        \textcite{alonso-gonzalez_what_2021} & Ridepooling simulieren & ohne Bezug & Region & Pooling & Simulation & — & Kosten/Fahrgast-km, Auslastung & Ridepooling erhöht Abdeckung und Auslastung gegenüber Solo-DRT. \\ \hline
        
        \textcite{alonso-mora_-demand_2017} & Fahrplan/Zuteilung optimieren & USA & Netz & Studierende & Literatur-Review & Literatur & Auslastung & Studie zeigt Potenziale und Grenzen urbanen Ridepoolings. \\ \hline
        
        \textcite{asatryan_ridepooling_2023} & Pooling/Zuteilung optimieren & Deutschland & Haltestelle & Pooling & Simulation & synthetisch & Auslastung & Studie zeigt Potenziale und Grenzen urbanen Ridepoolings. \\ \hline
        
        \textcite{basu_automated_2018} & Ridepooling simulieren & ohne Bezug & Region & Flexible Busse & Matem. Optimierungsmodell & — & Auslastung & Studie zeigt Potenziale und Grenzen urbanen Ridepoolings. \\ \hline
        
        \textcite{bischoff_city-wide_2017} & Ridepooling simulieren & Deutschland & Netz & Pooling & Simulation, Matem. Optimierungsmodell & — & Reisezeit & Pooling und Optimierung reduzieren Warte-/Reisezeiten und Fahrzeug-km im urbanen Kontext deutlich. \\ \hline

        \textcite{chen_exploring_2021} & Ridepooling-Konzept bewerten & ohne Bezug & Region & Flexible Busse & Empirie & Mix & Fahrzeug-km, Kosten/Fahrgast-km & Ridepooling erhöht Abdeckung und Auslastung gegenüber Solo-DRT. \\ \hline
        
        \textcite{engelhardt_speed-up_2020} & Pooling/Zuteilung optimieren & Deutschland & Region & Pooling & Matem. Optimierungsmodell & — & Auslastung & Studie zeigt Potenziale und Grenzen urbanen Ridepoolings. \\ \hline
        
        \textcite{engelhardt_predictive_2023} & Pooling/Zuteilung optimieren & Deutschland & Region & Pooling & — & — & Auslastung & Ridepooling erhöht Abdeckung und Auslastung gegenüber Solo-DRT. \\ \hline
        
        \textcite{fagnant_dynamic_2018} & Autonomen DRT planen & USA & Netz & First/Last-Mile & Simulation & — & Auslastung & Studie zeigt Potenziale und Grenzen urbanen Ridepoolings. \\ \hline
        
        \textcite{fehn_ride-parcel-pooling_2023} & Pooling/Zuteilung optimieren & Deutschland & Region & Pooling & Kartenbasierte Raumanalyse (GIS) & — & Flottengröße, Auslastung & Studie zeigt Potenziale und Grenzen urbanen Ridepoolings. \\ \hline
        
        \textcite{fielbaum_improving_2024} & Pooling/Zuteilung optimieren & Niederlande & Linie & Pooling & Matem. Optimierungsmodell & — & Auslastung & Ridepooling erhöht Abdeckung und Auslastung gegenüber Solo-DRT. \\ \hline
        \midrule
        
        \textcite{garus_exploring_2024} & Urbanes Ridepooling bewerten & Italien & Region & Flexible Busse & Matem. Optimierungsmodell & — & — & Studie zeigt Potenziale und Grenzen urbanen Ridepoolings. \\ \hline
        
        \textcite{jung_dynamic_2016} & Fahrplan/Zuteilung optimieren & USA & Netz & Flexible Busse & Simulation, Matem. Optimierungsmodell & — & Ø-Wartezeit, Auslastung & Ridepooling erhöht Abdeckung und Auslastung gegenüber Solo-DRT. \\ \hline
        
        \textcite{konig_analyzing_2018} & Pooling/Zuteilung optimieren & Deutschland & Netz & Pooling & — & — & CO\(_2\) & Studie zeigt Potenziale und Grenzen urbanen Ridepoolings. \\ \hline
        
        \textcite{leich_should_2019} & Ridepooling-Konzept bewerten & ohne Bezug & Region & Autonom & Simulation, Matem. Optimierungsmodell & Literatur & — & Studie zeigt Potenziale und Grenzen urbanen Ridepoolings. \\ \hline
        
        \textcite{liang_automated_2020} & Ridepooling-Konzept bewerten & Niederlande & Netz & First/Last-Mile & Matem. Optimierungsmodell & — & Reisezeit & Studie zeigt Potenziale und Grenzen urbanen Ridepoolings. \\ \hline
        
        \textcite{liu_filtering_2024} & Autonomen DRT planen & Frankreich & Linie & Autonom & Matem. Optimierungsmodell & — & — & Studie zeigt Potenziale und Grenzen urbanen Ridepoolings. \\ \hline
        
        \textcite{liu_quantifying_2025} & Pooling/Zuteilung optimieren & China & Netz & Pooling & Fallstudie & — & Auslastung & Studie zeigt Potenziale und Grenzen urbanen Ridepoolings. \\ \hline

        \textcite{liu_analysis_2015} & Ridepooling-Konzept bewerten & China & Netz & Flexible Busse & Matem. Optimierungsmodell & — & Auslastung & Studie zeigt Potenziale und Grenzen urbanen Ridepoolings. \\ \hline
        
        \textcite{lobel_detours_2025} & Pooling/Zuteilung optimieren & USA & Netz & Pooling & Matem. Optimierungsmodell & — & — & Studie zeigt Potenziale und Grenzen urbanen Ridepoolings. \\ \hline
        
        \textcite{schatzmann_investigating_2023} & Pooling/Zuteilung optimieren & Deutschland & Netz & Pooling & Matem. Optimierungsmodell, Empirie & Mix & — & Studie zeigt Potenziale und Grenzen urbanen Ridepoolings. \\ \hline
        
        \textcite{shen_integrating_2018} & First/Last-Mile verbessern & USA & Netz & First/Last-Mile & Simulation, Matem. Optimierungsmodell & — & Auslastung & Studie zeigt Potenziale und Grenzen urbanen Ridepoolings. \\ \hline
        
        \textcite{soza-parra_shareability_2024} & Pooling/Zuteilung optimieren & UK & Region & Pooling & Kartenbasierte Raumanalyse (GIS) & — & Auslastung & Studie zeigt Potenziale und Grenzen urbanen Ridepoolings. \\ \hline
        
        \textcite{stiglic_benefits_2015} & Ridepooling-Konzept bewerten & Niederlande & Netz & Pooling & Matem. Optimierungsmodell & — & — & Studie zeigt Potenziale und Grenzen urbanen Ridepoolings. \\ \hline
        
        \textcite{stiglic_enhancing_2018} & Urbanes Ridepooling bewerten & Niederlande & Netz & Studierende & Matem. Optimierungsmodell & — & — & Ridepooling erhöht Abdeckung und Auslastung gegenüber Solo-DRT. \\ \hline

        \textcite{vosooghi_shared_2019} & Autonomen DRT planen & Frankreich & Netz & Autonom & Simulation, Matem. Optimierungsmodell & — & Auslastung & Studie zeigt Potenziale und Grenzen urbanen Ridepoolings. \\ \hline
        
        \textcite{wallar_vehicle_2018} & Urbanes Ridepooling bewerten & ohne Bezug & Linie & Flexible Busse & Matem. Optimierungsmodell & — & Auslastung & Zielkonflikt zwischen Servicequalität und Bündelungsgrad wird sichtbar. \\ \hline
        
        \textcite{wen_transit-oriented_2018} & Autonomen DRT planen & USA & Netz & Autonom & Simulation, Matem. Optimierungsmodell & — & Auslastung & Studie zeigt Potenziale und Grenzen urbanen Ridepoolings. \\ \hline
        
        \textcite{zwick_ride-pooling_2021} & Ridepooling-Konzept bewerten & Deutschland & Region & Pooling & Simulation, Literatur-Review & Literatur & — & Studie zeigt Potenziale und Grenzen urbanen Ridepoolings. \\ \hline
        
        \textcite{zwick_analysis_2020} & Pooling/Zuteilung optimieren & Schweiz & Netz & Pooling & Matem. Optimierungsmodell & synthetisch & — & Ridepooling erhöht Abdeckung und Auslastung gegenüber Solo-DRT. \\ \hline
        
        \textcite{zwick_ride-pooling_2022} & Pooling/Zuteilung optimieren & Deutschland & Region & Pooling & Simulation, Kartenbasierte Raumanalyse (GIS) & — & Flottengröße, Fahrzeug-km & Pooling und Optimierung reduzieren Warte-/Reisezeiten und Fahrzeug-km im urbanen Kontext deutlich. \\ \hline
                
    \end{xltabular}
\end{landscape}
Nachdem die Forschungsgebiete der zentralen Begriffe nun beleuchtet wurden, werden im Folgenden Kapitel Arbeiten aufgezeigt, die mit dem von Schulz \& Vlćek vorgestellten Modell verwandt sind, um das Modell von Schulz \& Vlćek einzuordnen und abzugrenzen.

Es ist zwar nicht Teil des Untersuchungsgebietes dieser Arbeit, da Schulz \& Vlćek einen fest vorgegebenen Fahrplan für ihr Modell nutzen, aber es sei an dieser Stelle auf die Forschung im Bereich für die Vorhersage der Ankunftszeiten, etc. mit Quelle1, Quelle 2 usw. verwiesen

\textbf{DAS HIER IST ÜBERLEITUNG MIT DEM "REINEN" DARP UND VRP ZU DEN SEMI-FLEXIBLEN SYSTEMEN}
Das, in dieser Arbeit zu testende, Modell von Schulz \& Vlćek kombiniert Elemente eines klassischen Linienbus-Services mit der Flexibilität die Routen anhand der gestellten Nachfrage zu optimieren. Die Problemstellung von Schulz \& Vlćek ist mit denen des Dial-a-Ride Problems (DARP) und des Vehicle Routing Problems (VRP) verwandt, \textbf{lässt sich aber nicht klar einer der beiden Problemfamilien zuordnen ???}. Das DARP wird bereits seit Jahrzenten untersucht \parencite{psaraftis_dynamic_1980}, die Forschung zu diesem Problem ist dementsprechend sehr umfassend. Den wohl aktuellsten Überblick zum DARP und seinen Vairanten geben \textcite{molenbruch_typology_2017} und \textcite{ho_survey_2018}. 

Auch das VRP ist ein seit Jahrzehnten erforschtes Problem \parencite{orloff_fundamental_1974}. Das VRP allein ist ein so breit und intensiv erforschtes Problem, dass es dazu über 150 Review-Paper für spezifische Varianten oder Aspekte gibt. Es wird mit dem Searchstring \textbf{XXXX (siehe Anhang XXX)} auf die Literatur dieser Reviews \textbf{und damit indirekt auf die einzelnen Veröffentlichungen ???} verwiesen. Reviews des Problems haben unter anderem \textcite{braekers_vehicle_2016} und \textcite{vidal_concise_2020} gegeben. 

\textbf{Irgendwie noch erwähnen, dass auch die CVRP Variante ein breit erforschtes Thema ist, daher nicht tiefer im Detail betrachtet}

Überleitung: Da Mdoell von Schulz \& Vlćek semiflexibel und nicht so richtig ein reines DARP pder VRP wird anschließend auf die verwandten Problemstellungen verwiesen und eingeordnet.

%%%%%%%%%%%%%%%%%%%%
%%%%%%%%%%%%%%%%%%%%
%%%%%%%%%%%%%%%%%%%%
%%%%%%%%%%%%%%%%%%%%
\newpage
\section{Semi-flexible Systeme - Einordnung des zu betrachtenden Modells}
\label{sec:Einordnung}
\label{sec:2.2}

\textbf{GLIEDERUNG DIESES ABSCHNITTS:}
\begin{enumerate}
    \item Systeme die zwar semi-flexible aber nicht mit BLSP verwandt sind
    \item Verwandte Systeme
\end{enumerate}


\textbf{DIE TABELLE ZU DEN NICHT VERWANDTEN SYSTEMEN EVTL. NICHT AUF EXTRA SEITE}
\begin{landscape}

    \scriptsize
    \setlength{\tabcolsep}{2.2pt}
    \setlength{\arrayrulewidth}{0.1pt}
    \begin{xltabular}{\textwidth}{%
        L{1.8cm}:
        L{1.8cm}:
        L{2.2cm}:
        L{2.3cm}:
        L{1.2cm}:
        L{2.0cm}:
        L{2.0cm}:
        L{2.4cm}:
        L{6.0cm}:
    }
        \caption{Forschungsergebnisse zu nicht BLSP-verwandten Systemen}\label{tab:nicht_verwandt}\\ 
        \hline
        \LH{Paper} & \LH{Zielsetzung} & \LH{Semi-Flexible\\Mechanik} & \LH{Bedienmodus} &
        \LH{Stopps} & \LH{Raumbezug} & \LH{Nachfrage-\\berücksichtigung} & \LH{Unterscheidung \\zum BLSP} & \LH{Zentrale Erkenntnisse} \\
        \Xhline{0.6pt}
        \endfirsthead

        \multicolumn{9}{l}{\small\itshape Fortsetzung von Tabelle~\ref{tab:nicht_verwandt}.}\\[0.6\baselineskip]
        \hline
        \LH{Paper} & \LH{Zielsetzung} & \LH{Semi-Flexible\\Mechanik} & \LH{Bedienmodus} &
        \LH{Stopps} & \LH{Raumbezug} & \LH{Nachfrage-\\berücksichtigung} & \LH{Unterscheidung \\zum BLSP} & \LH{Zentrale Erkenntnisse} \\
        \Xhline{0.6pt}
        \endhead

        \hline
        \multicolumn{9}{r}{\small\itshape Fortsetzung auf der nächsten Seite}
        \endfoot

        \hline
        \endlastfoot

        \textcite{edward_kim_optimal_2019} & Konzept evaluieren & Zonen-Flex-Route & Headway-basiert & flexibel & Zone & stochastisch/unsicher & reine First/Last-Mile & Ansätze reduzieren Wartezeiten und Betriebskosten bei moderater Flotte deutlich. \\ \hline
        
        \textcite{kim_optimal_2021} & Planung/Dimensionierung & Zonen-Flex-Route & Headway-basiert & flexibel & Zone & stationär & reine First/Last-Mile & Ansätze reduzieren Wartezeiten und Betriebskosten bei moderater Flotte deutlich. \\ \hline
        
        \textcite{kim_conventional_2012} & Konzept evaluieren & Zonen-Flex-Route & Headway-basiert & teils-fix & Zone & mehrperiodig & Zonen/Korridor-basiert & Ansätze reduzieren Wartezeiten und Betriebskosten bei moderater Flotte deutlich. \\ \hline
        
        \textcite{lu_flexible_2016} & Verbesserung der Erreichbarkeit & Flex-Feeder & Headway-basiert & teils-fix & Feeder→Hub & stationär & reine First/Last-Mile & Ansätze reduzieren Wartezeiten und Betriebskosten bei moderater Flotte deutlich. \\ \hline
        
        \textcite{mehran_analytical_2020} & Verbesserung der Servicequalität & Variabler Servicetyp & bedarfsorientiert & fix & Netz & mehrperiodig & Zonen/Korridor-basiert & Ansätze reduzieren Wartezeiten und Betriebskosten bei moderater Flotte deutlich. \\ \hline
        
        \textcite{mishra_optimal_2023} & Planung/Dimensionierung & Zonen-Flex-Route & fester Takt & fix & Zone & stochastisch/unsicher & reine First/Last-Mile & Ansätze reduzieren Wartezeiten und Betriebskosten bei moderater Flotte deutlich. \\ \hline
        
        \textcite{mishra_cost_2024} & Konzept evaluieren & Flex-Feeder & Headway-basiert & teils-fix & Feeder→Hub & mehrperiodig & reine First/Last-Mile & Ansätze reduzieren Wartezeiten und Betriebskosten bei moderater Flotte deutlich. \\ \hline

        \textcite{ng_autonomous_2023} & Konzept evaluieren & Zonen-Flex-Route & bedarfsorientiert & teils-fix & Zone & stationär & Zonen/Korridor-basiert & Semi-flexible Varianten zeigen Vorteile gegenüber starren Konzepten in ausgewählten Szenarien. \\ \hline
        
        \textcite{qiu_demi-flexible_2015} & Konzept evaluieren & Zonen-Flex-Route & bedarfsorientiert & teils-fix & Netz & stochastisch/unsicher & Zonen/Korridor-basiert & Semi-flexible Varianten zeigen Vorteile gegenüber starren Konzepten in ausgewählten Szenarien. \\ \hline
        
        \textcite{sadrani_vehicle_2022} & Minimierung der Wartezeit & Korridor-Checkpoints & hybrid & teils-fix & Korridor & stochastisch/unsicher & Zonen/Korridor-basiert & Ansätze reduzieren Wartezeiten und Betriebskosten bei moderater Flotte deutlich. \\ \hline
        
        \textcite{stiglic_enhancing_2018} & Konzept evaluieren & Ride-Sharing-Kopplung & bedarfsorientiert & teils-fix & Netz & stationär & Zonen/Korridor-basiert & Erreichbarkeit und Servicequalität steigen gegenüber rein festen Linien merklich an. \\ \hline
        
        \textcite{wang_joint_2021} & Konzept evaluieren & Zonen-Flex-Route & bedarfsorientiert & flexibel & Zone & stationär & reine First/Last-Mile & Semi-flexible Varianten zeigen Vorteile gegenüber starren Konzepten in ausgewählten Szenarien. \\ \hline
        
        \textcite{zheng_slack_2018} & Konzept evaluieren & Korridor-Checkpoints & bedarfsorientiert & teils-fix & Netz & stochastisch/unsicher & Zonen/Korridor-basiert & Ansätze reduzieren Wartezeiten und Betriebskosten bei moderater Flotte deutlich. \\ \hline
        
                
    \end{xltabular}
\end{landscape}

Irgendwie Aussage darüber, dass in der Forschung oft gesagt wird, dass durch autonome vehicle erst so richtig ermöglichen demand responsive zu agiere, um fixed-schedule abzulösen (sieh)

Modell von Schulz \& Vlćek ist Kombination aus Linienverkehr \& on-demand, daher: semi-flexible Systeme --> \textbf{HIER EVENTUELL mehrere Quellen die Felxibilitt definieren?}

\textbf{EINORDNUNG ÜBER EINEN TABELARISCHEN VERGLEICH IN HINBLICK AUF OPTIMIERUNGSZIELE, WIE: total rider time, total number of vehicles --> UNterscheidung auch durch: "aus sicht des betreibers" oder aus sicht des fahrgastes"}


 

%%%%%%%%%%%%%%%%%%%%
%%%%%%%%%%%%%%%%%%%%
%%%%%%%%%%%%%%%%%%%%
%%%%%%%%%%%%%%%%%%%%
\section{Offene Forschungsfragen}
\label{sec:2.3}
\label{sec:OffeneForschungsfragen}
\begin{itemize}
    \item welche offenen Punkte aus anderen Papern greifen Schulz \& Vlćek eventuell auf?
    \item Kapazitätsfragen, Depotstruktur, Echtzeitfähigkeit
    \item kombination von on-demand und line-based nochmal evtl. aufgreifen nachdem entsprechend mit searchstring bewiesen oder nicht bewiesen ist, dass es diese kombination so noch nicht extensiv gibt
    \item Zukunftsperspektiven: adaptive Fahrpläne, Realtime-Demand
    \item Bewertung der Robustheit und Praktikabilität in Realanwendungen
    
    $\rightarrow$ Enden mit Rechtfertigung dafür, dass es sich lohnt die Kombination, die Vlćek und Schulz gemacht haben, weiter zu untersuchen
\end{itemize}

\textbf{--> Aussage on-demand senkt kosten in rural areas, aber ...}

%Hier Überleitung und EIngrenzung dessen was in der Arbeit behandelt wird