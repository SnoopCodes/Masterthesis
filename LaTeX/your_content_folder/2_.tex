\chapter{Stand der Forschung}
%%%%%%%%%%%%%%%%%%%%
%%%%%%%%%%%%%%%%%%%%
%%%%%%%%%%%%%%%%%%%%
%%%%%%%%%%%%%%%%%%%%
\section{Linienverkehr vs. Ridepooling vs. on-demand}

$\rightarrow$ warum ist das Thema relevant?
\begin{itemize}
    \item hier mit Verweis auf Literaturtabelle argumentieren, dass DRT in rural areas noch nicht so viel erforscht ist
    \item 
\end{itemize}

Je Thema:
\begin{itemize}
    \item Definitionen \& Merkmale
    \item Anwendungsbereiche \& typische Zielkonflikte
    \item Literaturübersicht zu semi-flexiblen Systemen (z.B. MAST, Schulbusse)
    \item Suchstruktur darstellen als:
    \begin{itemize}
        \item generell zum Thema on-demand in Relation zu Bussen: Searchstrings: (on-demand bus), ...
        \item klassisches bus system in rural areas
        \item on-demand in rural areas
        \item dann Kombination von on-demand und line-based
        
        
        \item 
    \end{itemize}
\end{itemize}


%%%%%%%%%%%%%%%%%%%%
%%%%%%%%%%%%%%%%%%%%
%%%%%%%%%%%%%%%%%%%%
%%%%%%%%%%%%%%%%%%%%

\section{Einordnung des zu betrachtenden Modells}



BESCHREIBUNG DER VORGEHENSWEISE:

\begin{enumerate}
    \item SUCHEN MITTELS REVIEW PAPER (SEARCHSTRING)
    \item ...
\end{enumerate}


$\rightarrow$ was gibt es schon?

\begin{itemize}
    \item Modell von Schulz \& Vlćek: Kombination aus Linienverkehr \& Bedarfssteuerung
    \item Überleitung zum Beispiel:
    \begin{itemize}
        \item on-demand wird in urban areas viel geforscht, aber nicht in rural areas, da sich dort on-demand eigentlich immer lohnt, aufgrund der geringen Nachfrage
        \item %kombination von on-demand und line base nochmal evtl. aufgreifen nachdem entsprechend mit searchstring bewiesen oder nicht bewiesen ist, dass es diese kombination so noch nicht extensiv gibt
    \end{itemize}
    \item Beitrag: Reduktion der Fahrtenzahl durch On-Demand-Verkürzung
    \item Einordnung in die Forschung zur flexiblen Linienplanung
\end{itemize}
%%%%%%%%%%%%%%%%%%%%
%%%%%%%%%%%%%%%%%%%%
%%%%%%%%%%%%%%%%%%%%
%%%%%%%%%%%%%%%%%%%%
\section{Relevante Modelle \& Literatur}
\textbf{$\rightarrow$ evtl. 2.2 und 2.3 zusammenlegen....}
\begin{itemize}
    \item Überblick über verwandte Optimierungsansätze (z.B. DARP, MIP, Netzwerkflussmodelle)
    \item Besonderheiten des gewählten Modells (Netzwerkstruktur, einfache Erweiterbarkeit)
    \item Überblick zur Methodik: LP/IP, Flow-Modelle, Erweiterbarkeit für verschiedene Szenarien
\end{itemize}

Hier evtl. dann die tabellarische Übersicht der gefundenen Literatur
%%%%%%%%%%%%%%%%%%%%
%%%%%%%%%%%%%%%%%%%%
%%%%%%%%%%%%%%%%%%%%
%%%%%%%%%%%%%%%%%%%%
\section{Offene Forschungsfragen}
\begin{itemize}
    \item Kapazitätsfragen, Depotstruktur, Echtzeitfähigkeit
    \item Zukunftsperspektiven: adaptive Fahrpläne, Realtime-Demand
    \item Bewertung der Robustheit und Praktikabilität in Realanwendungen
    
    $\rightarrow$ Enden mit Rechtfertigung dafür, dass es sich lohnt die Kombination, die Vlćek und Schulz gemacht haben, weiter zu untersuchen
\end{itemize}
%Hier Überleitung und EIngrenzung dessen was in der Arbeit behandelt wird