%Leitfrage ist: $\rightarrow$ Was ist der Kontext des zu implementierenden Modells?

\chapter{Literaturüberblick}
\label{chapter:2}
Das von Schulz \& Vlćek vorgeschlagene System stellt eine Kombination des klassischen Linienbusverkehrs und Elementen eines, auf Abruf basierenden Ridepooling-Dienstes dar. Das präsentierte mathematische Modell ist im Bereich der OR-Probleme mit der Problemfamilie des Dial-a-Ride Problems und des Vehicle Routing Problems (VRP) verwandt. Eine ausführlichere Einordnung des BLSP erfolgt in Kapitel \ref{sec:Einordnung}. Mit dem, von Schulz \& Vlćek, vorgeschlagenen System werden Elemente aus unterschiedlichen Systemen, Problemstellungen und Modellen verknüpft. Um ein einheitliches Verständnis zu schaffen, wird in diesem Kapitel die Definition zentraler Begriffe vorgenommen. Im Anschluss ein Überblick zur Forschung auf den Gebieten der zentralen Begriffe in Bezug auf das Verkehrsmittel Bus gegeben, bevor das Modell von Schulz \& Vlćek in die Menge verwandter Untersuchungen eingeordnet wird. 

Die in der Arbeit genutzten zentralen Begriffe werden wie folgt definiert:

\textbf{Linienbasiert} (engl.: line-based) bedeutet für das ÖPNV-System die Einhaltung von festen Routen, festgelegten Haltestellen und Fahrplänen wie es von städtischen Bus-, Straßenbahn- und U-Bahn-Netzen bekannt ist.

\textbf{Auf Abruf} (engl.: on-demand) ist die Eigenschaft eines Systems auf die realisierte Nachfrage der Fahrgäste zu reagieren \parencite[vgl.][S.3]{vansteenwegen_survey_2022}. Dies kann in Echtzeit oder auch vorverarbeitet in einem zuvor abgegrenzten Zeitraum erfolgen. Die Nachfrage wird dann innerhalb eines festgelegten Zeitraums bedient. Zeiträume können sich überschneiden und so zum Beispiel die Form eines rollierenden Horizonts annehmen.

\textbf{Hier evtl. nochmal Klarstellung, dass on-demand zwar demand-responsive ist, aber nicht andersrum, wie fälschlicherweise von \textcite{wang_multilevel_2014} in ihrem Paper behauptet}

\textbf{Ridepooling} wird vom \textcite{verband_der_automobilindustrie_ridepooling_2025} (\textbf{Achtung online Quelle} - Schulz \& Vlćek nutzen hier \parencite{vansteenwegen_survey_2022}) als kombinierte Mobilitätsform von ÖPNV und Taxi definiert. Der Ablauf bei dieser besonderen Form des Personenverkehrswerden ist wie folgt: Der Fahrgast wird an seinem individuellen Startpunkt abgeholt. Auf dem Weg zum individuellen Ziel steigen andere Fahrgäste zu und/oder wieder aus. Je nach Zielpunkt werden die Routen so kombiniert, dass benachbarte Ziele mit einem Fahrzeug bedient werden können.   


\textbf{Methodik} Bei der systematischen Vorgehensweise wurden die Suchbegriffe in englischer Sprache verwendet, da die meisten Veröffentlichungen heutzutage auf englischer Sprache getätigt werden. Die Suchbegriffe wurden miteinander verknüpft und in sogenannten Searchstrings verwendet. In Anhang \textbf{XXXXX}sind allen verwendeten Veröffentlichungen der Searchstring und die Datenbank zugeordnet auf der sie bzw. durch den sie gefunden wurden. An dieser Stelle sei erwähnt, dass dieselben Veröffentlichungen teilweise mit mehreren Searchstrings gefunden wurden, allerdings nur einem Searchstring zugeordnet wurden. Die Recherche der nachfolgend präsentierten Literatur erfolgte mittels der gebildeten Searchstrings in den wissenschaftlichen Online-Datenbanken Google Schoolar und Scopus. Reihenfolge bei der Recherche: 1. Zuerst wurde die Forschung auf den Gebieten der zentralen Begriffe gesichtet und geordnet. 2. Anschließend wurde eine Suche nach Veröffentlichungen zu den BLSP-verwandten Problemen DARP und VRP mit Review- oder State-of-the-Art Charakter durchgeführt, um einen Aufschlagspunkt für Lesende zu geben, wenn sich intensiver mit den verwandte Problemen beschäftigen werden möchte, da die Probleme zwar für die Einordnung des BLSP relevant, aber selbst nicht Teil des Betrachtungsraums dieser Arbeit sind. 3. Zu relevanten Varianten des DARP und VRP, die mit dem BLSP verwandt sein könnten, wurde eine Suche mit Searchstrings ausschließlich in der Scopus Datenbank durchgeführt. In allen Schritten wurden nur Veröffentlichungen berücksichtigt, die in einer vollständigen PDF-Version mittels des VPN-Zugangs der Universität Hamburg oder der Helmut-Schmidt-Universität zugänglich waren. Für die, in Kapitel \ref{sec:2.1} präsentierte, Literatur wurden die Searchstrings auf der Plattform Scopus im Rahmen der Funktion \glqq TITLE-ABS-KEY()\grqq{} gesucht. Bei der Recherche von Literatur in Kapitel \ref{sec:2.1} wurde bereits in den Searchstrings unterschieden, ob die Forschung im ländlichen oder städtischen Raum durchgeführt wurde. Umgebungen wie private Bereiche (Universiäts-Campus-Shuttle, etc.) sind nicht Teil des Untersuchungsbereiches. Andere Mobilitätsformen wie zum Beispiel das klassische Taxi oder U-Bahnen werden zur kontextuellen Abgrenzung erwähnt, sind aber kein Teil des Untersuchungsbereiches dieser Arbeit. Die verwandten OR-Probleme des DARP und des VRP inkl. ihrer Varianten werden zur Abgrenzung genutzt, werden aber als bekannt vorausgesetzt. Die, in Kapitel \ref{sec:2.2} genutzte, Literatur zur Einordnung / Abgrenzung des BLSP wurde auf Scopus durch Eingabe der Searchstrings innerhalb der Funktion \glqq TITLE()\grqq{} gefunden. Die Literatur wurde anhand des Titels und Abstract bewertet und somit für die weitere Analyse und eventuelle spätere Verwendung in dieser Arbeit ausgewählt. 

%%%%%%%%%%%%%%%%%%%%
%%%%%%%%%%%%%%%%%%%%
%%%%%%%%%%%%%%%%%%%%
%%%%%%%%%%%%%%%%%%%%
\section{Forschung auf den Gebieten der zentralen Begriffe}
\label{sec:2.1}
\label{sec:Kontext}

\subsection{ BLEIBT NICHT: generelle relevante Statements}
\textbf{ MEINE EIGENEN ERSTEN IDEEN:}

Es ist ersichtlich, das dieses Thema seit mehr als 20 Jahren eine große Aufmerksamkeit genießt. IN 2003 haben Quelle bereits eine Übersicht erstellt. 2014 noch eine und die eneuste 2022 von Quelle


line-based sind einfach der backbone in urban areas und sind, wenn nicht asl einziges, dann das teil des zentralen public transport networks um hohe Nachfrage kosteneffizient zu bedienen (hinweis Richtung Forschung zu feeder-networks). werden verwendet als Schulbusse, Schienenersatzverkehr, ...

on-demand häufig als "Feeder" zu den hoch-volumigen Systemen wie Metro oder Line-based buses

Generell Zielkonflikt von echten on-demand Lösungen in rural areas: geringere Kosten (durch weniger Fahrzeuge) vs. Abdeckung(?) Soll darauf hinführen, dass Schulz \& Vlćek weniger Busse nutzen bei Beibehaltung der Flexibilität.

\textbf{ACHTUNG GPT:}


1. On-Demand/DRT ist besonders geeignet für gering verdichtete bzw. ländliche Räume, in denen Linienbusse oft hohe Kosten je Fahrgast aufweisen und die Abdeckung sinkt. Belege: Berrada \& Poulhès (2021); Al-Suleiman et al. (2023).3. Flexible, integrierte Bedienformen (geplant + on-demand) sind für ländliche Räume besonders geeignet, weil sie spezielle Zeitfenster/Relaisaufgaben abdecken und die Erreichbarkeit erhöhen.(Porru et al., 2020, S. 2; S. 6).
\textbf{--> Erhöhen ERreichbarkeit, aber zu welchem Preis und ab welcher Nachfrage?}
\textbf{--> hier nochmal die ergebnisse von schasché 2022 angucken}

3. Ride-Pooling kann die Gesamt-Fahrleistung (km) substanziell senken, sofern Nachfrage, Heterogenität der Nutzerpräferenzen und Preissetzung zusammenpassen; Servicequalität bleibt dabei handhabbar. Beleg: de Ruijter et al. (2023).1. Pooling senkt Systemaufwand und dämpft Fahrzeug-km, wenn Nachfrage ausreichend gebündelt werden kann; Effizienz steigt mit Bevölkerungs-/Anfragedichte.(Zwick et al., 2021).2. Ride-sharing kann erhebliche Effizienzgewinne erzielen, ohne das Servicelevel zu verschlechtern; allerdings bestehen Trade-offs (Privatsphäre, Gesamtreisezeit).(Ruch et al., 2021, S. 4). Smartphone-basierte Buchung erleichtert die Nutzung.(Shaheen \& Cohen, 2019, S. 2, 5).* Ride-Pooling-Effizienz \& Dichteeffekte: Zwick et al., 2021; Ruch et al., 2021
\textbf{-->> Entscheident für Effizienz von Ridepooling ist die Tripdichte, sonst zuviele vehicles beötigt, um die Nachfrage zu bedienen. ICT trägt zu eienr besseren Nutzbarkeit bei}

4. Entscheidungsrelevante Attribute für ländliche Nutzer*innen bei Bus-/On-Demand-Alternativen sind u. a. Wartezeit, Verlässlichkeit, Preis und Bedienform. Belege: Bronsvoort et al. (2021); Ceder (2021).

5. Systemverfügbarkeit/„Availability“ wird primär durch räumliche und zeitliche Abdeckung bestimmt (Haltestellen-Nähe, Headway/Takt); diese Größen sind mess- und steuerbar. Beleg: Eboli et al. (2014).

1. Trip-Dichte ist ein zentraler Hebel für Flottengröße und Pooling-Rate bei ride-hailing/-pooling, selbst bei fixen Wartezeit-Zielen (z. B. 90\% < 10 min). Beleg: Kaddoura \& Schlenther (2021).

2. GTFS-basierte Analysen ermöglichen fahrplangetreue Reisezeiten inkl. Umstiegs-/Wartezeiten; ländliche Regionen weisen deutlich schlechtere Erreichbarkeit auf. Belege: Kotavaara et al. (2021, S. 5, S. 9).

4. Feeder-Systeme (z. B. Shuttle vs. Fußweg) besitzen kritische Nachfrageschwellen, ab denen sie systemisch vorteilhaft werden; Parameter wie Gehgeschwindigkeit und Dichte sind ausschlaggebend. Beleg: Li \& Luo (2025, S. 6–7).4. Feeder-Dienste können formale Netze wirksam ergänzen; die Entscheidung zwischen fixed und demand-responsive hängt vom Kontext und Netzdesign (Transfers, Tarife) ab.(McLeod et al., 2017, S. 4–5; S. 15).
\textbf{--> Feeder}

7. Netz-Topologie (Ring, Grid, Stern) beeinflusst die Pooling-Effizienz und Backlog-Skalierung. Beleg: Manik \& Molkenthin (2020, S. 7).




%%%%%%%%%%%%%%%%%%%%
%%%%%%%%%%%%%%%%%%%%
%%%%%%%%%%%%%%%%%%%%
\subsection{ BLEIBT NICHT: auf andere Forschungsaspekte hinweisen, die nicht weiter betrachtet wurden}
* Pricing/Fares → de Ruijter (2023); Bronsvoort (2021); Ceder (2021); Al-Suleiman (2023); Fernández L. (2008).
* Pricing / Marktgleichgewicht / Tarife:Estrada (2021); Fielbaum (2023); Hansen \& Sener (2023); Hansson (2019); Jin (2016)
* Pricing/Ökonomie (Randbezug über DRT-Vergleiche): Mageean \& Nelson, 2003.



* Accessibility/Equity/Inclusion → Eboli (2014); Ceder (2021); Berrada \& Poulhès (2021); Bronsvoort (2021); de Ruijter (2023); Al-Suleiman (2023); Fernández L. (2008).
* Accessibility/Equity (räumlich/zeitlich): Kotavaara et al., 2021; Mageean \& Nelson, 2003.




* Energie/Charging/Emissionen → Al-Suleiman (2023); Berrada \& Poulhès (2021); Bronsvoort (2021); Ceder (2021); de Ruijter (2023); Eboli (2014); Fernández L. (2008).





* Simulation/Agent-based / Realtime-Dispatching → Bronsvoort (2021); Ceder (2021); de Ruijter (2023); Fernández L. (2008).
* Simulation / Realtime-Dispatching /  Online-Steuerung:Hansen \& Sener (2023); Jin (2016); (teilweise) Estrada (2021)
* Realtime/Online-Dispatching \& Simulation: Kaddoura \& Schlenther, 2021; Liu \& Wang, 2022; Li et al., 2024.





* Shift Scheduling / Crew Rostering → Eboli (2014) (Randbezug/Erwähnung); Bronsvoort (2021) (Randbezug).
* Shift Scheduling / Personalplanung: Tóth \& Krész, 2013



* Multiobjective/Robustness → Berrada \& Poulhès (2021); Fernández L. (2008).



* KI/ML \& Prognose (z. B. Reisezeiten, Nachfrage):Gal (2017); Hansen \& Sener (2023); (teilweise) Gorev (2020)
* KI/ML/Prognose \& RL-Planung: Li et al., 2024; Liu \& Wang, 2022 (datengetrieben); Kotavaara et al., 2021 (GTFS-gestützte Analytik).




* Feeder/Connector-Design \& Schwellenwerte: Li \& Luo, 2025.
* Feeder/Connector \& Netzdesign: McLeod et al., 2017




* Disruption Management / Resilienz (Betriebsstörungen):Jin (2016)
* Disruption/Resilienz (Betriebsstörungen): Liu \& Wang, 2022.




* Skaleneffekte \& Flottendimensionierung:Fielbaum (2023)



* Fixed-route mit Request-Stops (Deviations/Connector):Gorev (2020); Estrada (2021)



* Netz-/Topologieeffekte: Manik \& Molkenthin, 2020.

* Präferenzen \& Zahlungsbereitschaft (WTP) für Service-Level:Hansson (2019)
* Nutzerakzeptanz / Erwartungsmanagement (rural DRT): Schaschü et al., 2022

* Smart-Mobility-Integration \& Realtime-Information: Porru et al., 2020; Shaheen \& Cohen, 2019; McLeod et al., 2017



* Emissionen/Stau (Policy-Motivation): Shaheen \& Cohen, 2019


Anwendungsbereiche \& typische Zielkonflikte darstellen (evtl. Grafik dazu anfertigen: Taxi als flexibelstes on-demand, keine festen Routen und krasses Gegenteil ist Linienbus mit festem Fahrplan und Routen)

%%%%%%%%%%%%%%%%%%%%
%%%%%%%%%%%%%%%%%%%%
%%%%%%%%%%%%%%%%%%%%
\subsection{BLEIBT NICHT: Forschungsergebnisse auf den Gebieten der zentralen Begriffe}


\textbf{An dieser Stelle sei erwähnt, dass in diesem Kapitel zunächst die Breite der Forschung dargestellt wird, bevor in Kapitel \ref{sec:2.2} das Modell von Schulz \& Vlćek eingeordnet wird.}

Aussage darüber, dass viel zu "on-demand rural" mit "line-based rural" gefunden wurde -> zeigt enge verbundenheit der begriffe usw.


Ridepooling: dem zugrundeliegend ist das DARP -> wofür entwickelt (Quelle), dann überführen auf Busse (Quelle) siehe Schulz \& Vlćek, \textbf{ganze wichtig: reiter et al. 2024} - das beschreiben Schulz \& Vlćek als sehr nah dran, aber Schulz \& Vlćek berücksichtigen einen vorgegebenen Fahrplan. 

Ridepooling: Synonyme Verwendung der Begriffe Ride-sharing, ridepooling usw


\begin{landscape} 
    \begin{table}[p]
        \centering
        \caption{Forschungsergebnisse zu Line-based — rural}
        \label{tab:lb-rural}
        \scriptsize
        \setlength{\tabcolsep}{2.5pt}
        \setlength{\arrayrulewidth}{0.1pt}

        \resizebox{\textheight}{!}{%
        \begin{tabular}{
          L{2.0cm}:
          L{2.5cm}:
          L{2.0cm}:
          L{2.0cm}:
          L{2.5cm}:
          L{2.3cm}:
          L{1.8cm}:
          L{2.6cm}:
          L{7.4cm}:
        }
          \hline
          \LH{Paper} & \LH{Zielsetzung} & \LH{Region /\\Land} & \LH{Betrachtungs-\\ebene} &
          \LH{Fokus / \\ Anwendungsfeld} & \LH{Methode} & \LH{Daten} & \LH{KPI} & \LH{Zentrale Erkenntnisse} \\
          \Xhline{0.6pt}

            \textcite{cirella_transport_2019} & Review: Innovationen für Ältere & ohne Bezug & Region & Ältere & Literatur-Review & Literatur & — & Ältere brauchen Innovationen bei Angebot, Infrastruktur und Fahrzeugen; Umsetzungs- und Barrierefreiheitslücken bestehen. \\ \hline
            \textcite{das_planning_2012} & Planung ländlicher Feederlinien & Indien & Linie & Feeder & Analytische Näherung / Empirie & Umfrage & Ø-Wartezeit / Gehzeit & Generalisiert gewichtete Geh-/Warte- und Komfortfaktoren liefern praktikable Linienführungen und Takte für ländliche Feeder. \\ \hline
            \textcite{guiver_buses_2007} & Bewertung touristischer Buswirkungen & UK & Region & Tourismus & Empirie & Umfrage & Modal-Shift / Erreichbarkeit & Geplante Busse bewirken moderaten Modal-Shift, ermöglichen Zugang ohne Auto und stützen lokale Ökonomien; Fördermittel bleiben problematisch. \\ \hline
            \textcite{petersen_watching_2016} & Netzansatz mit Taktfahrplan & Schweiz & Netz & Taktfahrplan & Fallstudie / Literatur-Review & Literatur & — & In ländlichen Räumen bieten integrierte Taktfahrplan-Netze verlässliche Anschlüsse und sollten vor DRT als Basisstrategie geprüft werden. \\ \hline
            \textcite{takamatsu_bus_2020} & Optimierung von Umsteigezeiten & Japan & Netz & Taktfahrplan & Matem. Optimierungsmodell / Fallstudie & Realdaten & Ø-Wartezeit / Umsteigezeit & Optimierte Takte/Anschlüsse verkürzen Wartezeiten deutlich; Tohoku-Fall bestätigt Praxistauglichkeit bei niedriger Angebotsdichte. \\ \hline
            \textcite{tsigdinos_route_2024} & Strategie flexibler Busdienste & Griechenland & Region & Flexible Busse & Empirie / Kartenbasierte Raumanalyse (GIS) & Umfrage & — & Befragung+GIS zeigen: Kosten und Zeit dominieren; morgens wird Tür-zu-Tür, nachmittags Haltestellenbetrieb bevorzugt. \\ \hline
            \textcite{zhen_feeder_2024} & CA-Design heterogener Feeder & ohne Bezug & Netz & Feeder & Analytische Näherung / Matem. Optimierungsmodell & synthetisch & Reisezeit / Kosten/Fahrgast-km & Koordinierte Zubringer-/Stammfahrpläne senken Nutzerkosten und verbessern Systemleistung bei heterogener Nachfrage. \\ \hline
        \end{tabular}
        }% end resizebox
    \end{table}

    \begin{table}[p]
        \centering
        \caption{Forschungsergebnisse zu Line-based — urban}
        \label{tab:lb-urban}
        \scriptsize
        \setlength{\tabcolsep}{2.5pt}
        \setlength{\arrayrulewidth}{0.1pt}

        \resizebox{\textheight}{!}{%
        \begin{tabular}{
            L{2.0cm}:
            L{2.5cm}:
            L{2.0cm}:
            L{2.0cm}:
            L{2.5cm}:
            L{2.3cm}:
            L{1.8cm}:
            L{2.6cm}:
            L{7.4cm}:
        }
          \hline
          \LH{Paper} & \LH{Zielsetzung} & \LH{Region /\\Land} & \LH{Betrachtungs-\\ebene} &
          \LH{Fokus / \\ Anwendungsfeld} & \LH{Methode} & \LH{Daten} & \LH{KPI} & \LH{Zentrale Erkenntnisse} \\
          \Xhline{0.6pt}

          \textcite{fielbaum_improving_2024} & Integration Ridepooling & — & Linie & Ridepooling & Matem. Optimierungsmodell / Simulation & — & Reisezeit / Fahrzeug-km & zeigt Vorteile einer Ridepooling-Kopplung mit Linien; senkt Betriebskosten; senkt Emissionen. \\ \hline
          \textcite{filippi_exploiting_2024} & Linienoptimierung & Italy & Linie & Modularbus & Matem. Optimierungsmodell & — & Reisezeit / Auslastung & nutzt modulare Kopplung für flexible Kapazität; verkürzt Reisezeiten und verbessert Auslastung. \\ \hline
          \textcite{gal_traveling_2017} & Reisezeitprognose & ohne Bezug & Linie & Taktfahrplan & Empirie & Realdaten & Reisezeit / Prognosegüte & liefert praxistaugliche Prognosen der Reisezeiten im Linienverkehr (gute Prognosegüten). \\ \hline
          \textcite{hatzenbuhler_network_2022} & Netzdesign & ohne Bezug & Netz & Autonom & Matem. Optimierungsmodell / Simulation & — & Ø-Wartezeit / Reisezeit & zeigt, wie autonome Busnetz-Designs Nutzer- und Betreiberkosten senken können. \\ \hline
          \textcite{jimenez_urban_2016} & Flottenzuordnung & Spain & Linie & Emissionen & Matem. Optimierungsmodell & Mix & CO\(_2\) / Fahrzeug-km & weist nach, dass optimierte Flottenzuordnung Emissionen im Liniennetz deutlich reduzieren kann. \\ \hline
          \textcite{kim_integrating_2013} & Integration gemischter Flotten & USA & Linie & Flotten & Matem. Optimierungsmodell & — & Kosten/Fahrgast-km / Auslastung & kombinierte feste und flexible Dienste mit gemischten Flotten senken Kosten und erhöhen Auslastung. \\ \hline
          \textcite{rashvand_real-time_2024} & Ankunftsprognose & USA & Linie & KI & Empirie & Realdaten & Prognosegüte / Pünktlichkeit & Deep-Learning-Ansätze liefern robuste Ankunftsprognosen und verbessern die wahrgenommene Pünktlichkeit. \\ \hline
          \textcite{rosca_designing_2024} & Fahrplanoptimierung & Romania & Linie & Flotten & Empirie / Simulation & — & Pünktlichkeit / Kosten/Fahrgast-km & KI-gestützte Planung verbessert Pünktlichkeit und senkt Planungskosten in urbanen Netzen. \\ \hline
          \textcite{tang_data-driven_2021} & Fahrplanoptimierung & China & Linie & Taktfahrplan & Matem. Optimierungsmodell & Realdaten & Ø-Wartezeit / Reisezeit & zweckmäßige MOEA-Fahrpläne verkürzen Warte- und Reisezeiten auf städtischen Linien. \\ \hline
          \textcite{tian_autonomous_2021} & Flottenoptimierung & China & Linie & Autonom & Matem. Optimierungsmodell & — & Kosten/Fahrgast-km / Flottengröße & autonome Flotten mit Unsicherheiten können Betreiberkosten senken und Leistung stabilisieren. \\ \hline
          \textcite{wei_optimizing_2020} & Integration Metro-Bus & China & Netz & Integration & Matem. Optimierungsmodell / Fallstudie & Realdaten & Ø-Wartezeit / Reisezeit & koordinierte Metro-Bus-Planung senkt Nutzerzeiten und verbessert Gesamtleistung. \\ \hline
        \end{tabular}
        }% end resizebox
    \end{table}
\end{landscape}


\begin{landscape}

    \scriptsize
    \setlength{\tabcolsep}{2.2pt}
    \setlength{\arrayrulewidth}{0.1pt}
    \begin{xltabular}{\textwidth}{%
        L{1.8cm}:
        L{1.8cm}:
        L{1.5cm}:
        L{2.3cm}:
        L{2.3cm}:
        L{2.0cm}:
        L{1.5cm}:
        L{2.4cm}:
        L{6.0cm}:
    }
        \caption{Forschungsergebnisse zu on-demand — rural}\label{tab:od-rural}\\ 
        \hline
        \LH{Paper} & \LH{Zielsetzung} & \LH{Region /\\Land} & \LH{Betrachtungs-\\ebene} &
        \LH{Fokus / \\ Anwendungsfeld} & \LH{Methode} & \LH{Daten} & \LH{KPI} & \LH{Zentrale\\Erkenntnisse} \\
        \Xhline{0.6pt}
        \endfirsthead

        \multicolumn{9}{l}{\small\itshape Fortsetzung von Tabelle~\ref{tab:od-rural}.}\\[0.6\baselineskip]
        \hline
        \LH{Paper} & \LH{Zielsetzung} & \LH{Region /\\Land} & \LH{Betrachtungs-\\ebene} &
        \LH{Fokus / \\ Anwendungsfeld} & \LH{Methode} & \LH{Daten} & \LH{KPI} & \LH{Zentrale\\Erkenntnisse} \\
        \Xhline{0.6pt}
        \endhead

        \hline
        \multicolumn{9}{r}{\small\itshape Fortsetzung auf der nächsten Seite}
        \endfoot

        \hline
        \endlastfoot

        \textcite{bauchinger_developing_2021} & Konnektivität verbessern & Europa & Region & Feeder, Multimodal & Fallstudie, Empirie & Mix & Erreichbarkeit, Umsteigezeit & Komplementäre Dienste (DRT, multimodal) verbessern Erreichbarkeit und ÖPNV-Anbindung in ländlich--urbanen Regionen. \\ \hline
        
        \textcite{brake_demand_2004} & DRT-Markt prüfen & UK & Region & Inklusion & Fallstudie & Realdaten & — & Britische DRT-Erfahrungen fördern soziale Inklusion und Intermodalität; Umsetzung wird durch Finanzierungs- und Betriebsfragen begrenzt. \\ \hline
        
        \textcite{calabro_designing_2023} & DRT-Bedingungen vergleichen & Italien & Region & Kleinstädte & Simulation & synthetisch & Ø-Wartezeit, Auslastung & ABM zeigt: DRT ist FRT in Kleinstädten überlegen, wenn Umwege, Warte- und Gehzeiten minimiert und Teilfahrten gebündelt werden. \\ \hline
        
        \textcite{coutinho_impacts_2020} & Linienersatz bewerten & Niederlande & Linie & Linienersatz & Empirie & Realdaten & Fahrgast-km, CO\(_2\) & Im Mokumflex-Pilotprojekt sanken Fahrgast-km, Kosten und Emissionen pro Fahrgast deutlich, jedoch ging die Nachfrage stark zurück. \\ \hline
        
        \textcite{dorso_transforming_2025} & DRT-Effekte bewerten & Italien & Region & Pilotstudie & Empirie, Simulation & Mix & Ø-Wartezeit, Reisezeit & DRT steigerte Erreichbarkeit und senkte Warte- und Fahrzeiten im suburbanen Palermo; wirtschaftliche und rechtliche Rahmenbedingungen bleiben kritisch. \\ \hline
        
        \textcite{daduna_evolution_2020} & Entwicklung skizzieren & ohne Bezug & Region & Autonom/Digital & Literatur-Review & Literatur & — & Autonome Fahrzeuge und Digitalisierung können Kostenstrukturen ländlicher ÖPNV-Angebote grundlegend verändern, ersetzen Linienverkehr jedoch nicht vollständig. \\ \hline
        
        \textcite{daniels_flexible_2012} & Barrieren identifizieren & Australien & Region & Implementierung & Empirie & Umfrage & — & Fünf Barrierefelder (Regulierung, Finanzierung, Betrieb, Einstellungen, Information) hemmen FTS-Einführung in Niedrigdichtegebieten. \\ \hline
        
        \textcite{dytckov_potential_2022} & Ruralen DRT bewerten & Dänemark & Region & Studierende & Simulation & — & Ø-Wartezeit, Auslastung & Studie zeigt Potenziale und Grenzen von DRT im ländlichen Kontext. \\ \hline
        
        \textcite{galarza_montenegro_large_2021} & Feeder-Service optimieren & ohne Bezug & Linie & Feeder & Matem. Optimierungsmodell & — & Kosten/Fahrgast-km, Auslastung & Studie zeigt Potenziale und Grenzen von DRT im ländlichen Kontext. \\ \hline
        
        \textcite{galarza_montenegro_demand-responsive_2024} & Feeder-Service optimieren & ohne Bezug & Linie & Feeder & Matem. Optimierungsmodell & — & Ø-Wartezeit, Kosten/Fahrgast-km & Ansatz verbessert Erreichbarkeit und Servicegrad gegenüber Status quo. \\ \hline

        \textcite{guo_modular_2023} & Fahrplan/Zuteilung optimieren & China & Netz & Studierende & Matem. Optimierungsmodell, Empirie & — & Auslastung, Erreichbarkeit & Ansatz verbessert Erreichbarkeit und Servicegrad gegenüber Status quo. \\ \hline
        
        \textcite{jiang_integrated_2025} & Fahrplan/Zuteilung optimieren & ohne Bezug & Netz & Studierende & Simulation, Matem. Optimierungsmodell & Realdaten & Reisezeit, Flottengröße & Optimierung/Szenarien senken Warte- und Reisezeiten im ländlichen DRT deutlich. \\ \hline

        \textcite{knierim_attitude_2021} & DRT-Konzept bewerten & Deutschland & Linie & Studierende & Empirie & Mix & Auslastung & Ansatz verbessert Erreichbarkeit und Servicegrad gegenüber Status quo. \\ \hline
        
        \textcite{lu_demand-responsive_2023} & Ruralen DRT bewerten & Deutschland & Linie & Studierende & Simulation, Empirie & Realdaten & Flottengröße, Kosten/Fahrgast-km & Studie zeigt Potenziale und Grenzen von DRT im ländlichen Kontext. \\ \hline
        
        \textcite{marti_optimization_2023} & Ruralen DRT bewerten & Schweiz & Region & Flexible Busse & Matem. Optimierungsmodell & — & Auslastung & Studie zeigt Potenziale und Grenzen von DRT im ländlichen Kontext. \\ \hline
        
        \textcite{marti_flexible_2024} & Ruralen DRT bewerten & Spanien & Netz & Schule & — & — & Auslastung & Studie zeigt Potenziale und Grenzen von DRT im ländlichen Kontext. \\ \hline
        
        \textcite{mulley_flexible_2009} & Ruralen DRT bewerten & UK & Region & Flexible Busse & Kartenbasierte Raumanalyse (GIS) & — & Auslastung, Erreichbarkeit & Studie zeigt Potenziale und Grenzen von DRT im ländlichen Kontext. \\ \hline
        
        \textcite{nourbakhsh_structured_2012} & DRT-Konzept bewerten & USA & Linie & Flexible Busse & — & — & Auslastung & Studie zeigt Potenziale und Grenzen von DRT im ländlichen Kontext. \\ \hline
        
        \textcite{papanikolaou_analytical_2021} & DRT-Konzept bewerten & ohne Bezug & Region & Flexible Busse & Analytische Näherung & — & Auslastung & Studie zeigt Potenziale und Grenzen von DRT im ländlichen Kontext. \\ \hline
        
        \textcite{poltimae_search_2022} & Ruralen DRT bewerten & ohne Bezug & Region & Flexible Busse & Literatur-Review & Literatur & — & Studie zeigt Potenziale und Grenzen von DRT im ländlichen Kontext. \\ \hline
        
        \textcite{schluter_impact_2021} & Autonomen DRT planen & Deutschland & Netz & Autonom & — & — & Auslastung & Studie zeigt Potenziale und Grenzen von DRT im ländlichen Kontext. \\ \hline
        
        \textcite{sieber_improved_2020} & Ruralen DRT bewerten & Schweiz & Netz & Autonom & Simulation & — & Auslastung & Ansatz verbessert Erreichbarkeit und Servicegrad gegenüber Status quo. \\ \hline
        
        \textcite{torrisi_evaluation_2025} & DRT-Konzept bewerten & Italien & Region & Flexible Busse & — & — & Auslastung & Studie zeigt Potenziale und Grenzen von DRT im ländlichen Kontext. \\ \hline
        
        \textcite{velaga_potential_2012} & Ruralen DRT bewerten & UK & Netz & Flexible Busse & Literatur-Review & Literatur & — & Ansatz verbessert Erreichbarkeit und Servicegrad gegenüber Status quo. \\ \hline
        
        \textcite{viergutz_demand_2019} & DRT-Konzept bewerten & Deutschland & Netz & Flexible Busse & Literatur-Review & Literatur & Auslastung & Studie zeigt Potenziale und Grenzen von DRT im ländlichen Kontext. \\ \hline
        
        \textcite{wang_multilevel_2014} & DRT-Konzept bewerten & ohne Bezug & Linie & Flexible Busse & — & — & Auslastung & Ansatz verbessert Erreichbarkeit und Servicegrad gegenüber Status quo. \\ \hline
        
        \textcite{wang_planning_2025} & DRT-Konzept bewerten & Frankreich & Netz & Flexible Busse & — & — & Auslastung, Erreichbarkeit & Optimierung/Szenarien senken Warte- und Reisezeiten im ländlichen DRT deutlich. \\ \hline
        
        \textcite{white_roles_2016} & Rollen vergleichen & UK & Region & Kostenvergleich & Literatur-Review & Literatur & Kosten/Fahrgast-km, Auslastung & Konventionelle Interurban-Linien sind oft kosteneffizienter; DRT weist häufig hohe Kosten je Fahrt auf und eignet sich eher für Zielgruppen-/Mindestbedienung. \\ \hline
    \end{xltabular}
\end{landscape}

\begin{landscape}

    \scriptsize
    \setlength{\tabcolsep}{2.2pt}
    \setlength{\arrayrulewidth}{0.1pt}
    \begin{xltabular}{\textwidth}{%
        L{1.8cm}:
        L{1.8cm}:
        L{1.5cm}:
        L{2.3cm}:
        L{2.3cm}:
        L{2.0cm}:
        L{1.5cm}:
        L{2.4cm}:
        L{6.0cm}:
    }
        \caption{Forschungsergebnisse zu on-demand — urban}\label{tab:od-urban}\\ 
        \hline
        \LH{Paper} & \LH{Zielsetzung} & \LH{Region /\\Land} & \LH{Betrachtungs-\\ebene} &
        \LH{Fokus / \\ Anwendungsfeld} & \LH{Methode} & \LH{Daten} & \LH{KPI} & \LH{Zentrale\\Erkenntnisse} \\
        \Xhline{0.6pt}
        \endfirsthead

        \multicolumn{9}{l}{\small\itshape Fortsetzung von Tabelle~\ref{tab:od-urban}.}\\[0.6\baselineskip]
        \hline
        \LH{Paper} & \LH{Zielsetzung} & \LH{Region /\\Land} & \LH{Betrachtungs-\\ebene} &
        \LH{Fokus / \\ Anwendungsfeld} & \LH{Methode} & \LH{Daten} & \LH{KPI} & \LH{Zentrale\\Erkenntnisse} \\
        \Xhline{0.6pt}
        \endhead

        \hline
        \multicolumn{9}{r}{\small\itshape Fortsetzung auf der nächsten Seite}
        \endfoot

        \hline
        \endlastfoot


        \textcite{abe_introducing_2019} & Autonomen DRT planen & UK & Netz & Autonom & Matem. Optimierungsmodell & — & Kosten/Fahrgast-km, Erreichbarkeit & Optimierung/Szenarien senken Warte- und Reisezeiten im urbanen DRT deutlich. \\ \hline
        
        \textcite{alonso-gonzalez_potential_2018} & DRT-Konzept bewerten & ohne Bezug & Linie & Flexible Busse & Matem. Optimierungsmodell, Empirie & — & Auslastung & Ansatz verbessert Erreichbarkeit und Servicegrad gegenüber Status quo. \\ \hline
        
        \textcite{charisis_drt_2018} & First/Last-Mile verbessern & Deutschland & Linie & First/Last-Mile & — & — & Auslastung, Erreichbarkeit & Optimierung/Szenarien senken Warte- und Reisezeiten im urbanen DRT deutlich. \\ \hline
        
        \textcite{diana_methodology_2009} & DRT-Konzept bewerten & Frankreich & Linie & Flexible Busse & Matem. Optimierungsmodell & — & Auslastung & Ansatz verbessert Erreichbarkeit und Servicegrad gegenüber Status quo. \\ \hline
        
        \textcite{giuffrida_addressing_2021} & DRT-Konzept bewerten & Italien & Region & Flexible Busse & Matem. Optimierungsmodell, Kartenbasierte Raumanalyse (GIS) & Realdaten & Auslastung, Erreichbarkeit & Studie zeigt Potenziale und Grenzen von DRT im urbanen Kontext. \\ \hline
        
        \textcite{hazan_-demand_nodate} & Urbanen DRT bewerten & ohne Bezug & Region & Flexible Busse & Matem. Optimierungsmodell & — & Kosten/Fahrgast-km, Auslastung & Ansatz verbessert Erreichbarkeit und Servicegrad gegenüber Status quo. \\ \hline
        
        \textcite{huang_flexible_2020} & DRT-Konzept bewerten & ohne Bezug & Linie & Flexible Busse & Simulation, Matem. Optimierungsmodell & — & Auslastung & Studie zeigt Potenziale und Grenzen von DRT im urbanen Kontext. \\ \hline
        
        \textcite{inturri_taxi_2021} & DRT per Simulation prüfen & Italien & Netz & Flexible Busse & Simulation & — & Auslastung & Studie zeigt Potenziale und Grenzen von DRT im urbanen Kontext. \\ \hline
        
        \textcite{jager_multi-agent_2018} & Autonomen DRT planen & ohne Bezug & Region & Autonom & Matem. Optimierungsmodell & — & Auslastung & Optimierung/Szenarien senken Warte- und Reisezeiten im urbanen DRT deutlich. \\ \hline
        
        \textcite{kim_optimization_2025} & DRT-Konzept bewerten & Schweiz & Region & Flexible Busse & Matem. Optimierungsmodell & — & Auslastung & Studie zeigt Potenziale und Grenzen von DRT im urbanen Kontext. \\ \hline
        
        \textcite{li_efficient_2024} & DRT-Konzept bewerten & Korea & Linie & Flexible Busse & — & — & Auslastung & Studie zeigt Potenziale und Grenzen von DRT im urbanen Kontext. \\ \hline
        
        \textcite{liyanage_agent-based_2020} & DRT per Simulation prüfen & Australien & Region & Flexible Busse & Simulation & — & Auslastung, CO\(_2\) & Optimierung/Szenarien senken Warte- und Reisezeiten im urbanen DRT deutlich. \\ \hline
        
        \textcite{melo_demand-responsive_2024} & First/Last-Mile verbessern & Schweiz & Netz & First/Last-Mile & — & — & Auslastung & Ansatz verbessert Erreichbarkeit und Servicegrad gegenüber Status quo. \\ \hline
        
        \textcite{meshkani_innovative_2024} & First/Last-Mile verbessern & ohne Bezug & Region & First/Last-Mile & Matem. Optimierungsmodell & — & Auslastung & Studie zeigt Potenziale und Grenzen von DRT im urbanen Kontext. \\ \hline
        
        \textcite{mortazavi_integrated_2024} & DRT-Konzept bewerten & Australien & Netz & Flexible Busse & Matem. Optimierungsmodell, Fallstudie & — & Reisezeit, Kosten/Fahrgast-km & Studie zeigt Potenziale und Grenzen von DRT im urbanen Kontext. \\ \hline
        
        \textcite{pal_smartporter_2016} & Urbanen DRT bewerten & ohne Bezug & Netz & Flexible Busse & Kartenbasierte Raumanalyse (GIS) & — & — & Optimierung/Szenarien senken Warte- und Reisezeiten im urbanen DRT deutlich. \\ \hline

        \textcite{queiroz_instance_2024} & DRT-Konzept bewerten & Frankreich & Linie & Flexible Busse & Matem. Optimierungsmodell & — & Auslastung & Studie zeigt Potenziale und Grenzen von DRT im urbanen Kontext. \\ \hline
        
        \textcite{shinoda_is_2004} & DRT per Simulation prüfen & UK & Linie & Flexible Busse & Simulation, Matem. Optimierungsmodell & — & — & Ansatz verbessert Erreichbarkeit und Servicegrad gegenüber Status quo. \\ \hline
        
        \textcite{sun_research_2025} & DRT-Konzept bewerten & China & Region & Flexible Busse & Matem. Optimierungsmodell & — & Flottengröße, Kosten/Fahrgast-km & Studie zeigt Potenziale und Grenzen von DRT im urbanen Kontext. \\ \hline
        
        \textcite{torrisi_evaluation_2025} & Urbanen DRT bewerten & Italien & Region & Flexible Busse & Matem. Optimierungsmodell & — & Auslastung & Studie zeigt Potenziale und Grenzen von DRT im urbanen Kontext. \\ \hline
        
        \textcite{yan_integrating_2019} & Urbanen DRT bewerten & USA & Netz & First/Last-Mile & Matem. Optimierungsmodell & — & Auslastung & Ansatz verbessert Erreichbarkeit und Servicegrad gegenüber Status quo. \\ \hline
        
        \textcite{yeon_real-time_2025} & DRT-Konzept bewerten & Korea & Linie & Flexible Busse & Matem. Optimierungsmodell, Kartenbasierte Raumanalyse (GIS) & — & Auslastung & Studie zeigt Potenziale und Grenzen von DRT im urbanen Kontext. \\ \hline
        
        \textcite{yi_real-time_2025} & Feeder-Service optimieren & USA & Linie & Feeder & Matem. Optimierungsmodell & — & Auslastung & Studie zeigt Potenziale und Grenzen von DRT im urbanen Kontext. \\ \hline
        
        \textcite{zhang_routing_2024} & Fahrplan/Zuteilung optimieren & China & Linie & Flexible Busse & Matem. Optimierungsmodell & — & Auslastung & Ansatz verbessert Erreichbarkeit und Servicegrad gegenüber Status quo. \\ \hline
    \end{xltabular}
\end{landscape}

\begin{landscape}

    \scriptsize
    \setlength{\tabcolsep}{2.2pt}
    \setlength{\arrayrulewidth}{0.1pt}
    \begin{xltabular}{\textwidth}{%
        L{1.8cm}:
        L{1.8cm}:
        L{1.5cm}:
        L{2.3cm}:
        L{2.3cm}:
        L{2.0cm}:
        L{1.5cm}:
        L{2.4cm}:
        L{6.0cm}:
    }
        \caption{Forschungsergebnisse zu ridepooling — rural}\label{tab:rp-rural}\\ 
        \hline
        \LH{Paper} & \LH{Zielsetzung} & \LH{Region /\\Land} & \LH{Betrachtungs-\\ebene} &
        \LH{Fokus / \\ Anwendungsfeld} & \LH{Methode} & \LH{Daten} & \LH{KPI} & \LH{Zentrale\\Erkenntnisse} \\
        \Xhline{0.6pt}
        \endfirsthead

        \multicolumn{9}{l}{\small\itshape Fortsetzung von Tabelle~\ref{tab:rp-rural}.}\\[0.6\baselineskip]
        \hline
        \LH{Paper} & \LH{Zielsetzung} & \LH{Region /\\Land} & \LH{Betrachtungs-\\ebene} &
        \LH{Fokus / \\ Anwendungsfeld} & \LH{Methode} & \LH{Daten} & \LH{KPI} & \LH{Zentrale\\Erkenntnisse} \\
        \Xhline{0.6pt}
        \endhead

        \hline
        \multicolumn{9}{r}{\small\itshape Fortsetzung auf der nächsten Seite}
        \endfoot

        \hline
        \endlastfoot

        \textcite{elting_potential_2021} & Ridepooling-Konzept bewerten & ohne Bezug & Region & Flexible Busse & Matem. Optimierungsmodell, Literatur-Review & Literatur & — & Studie zeigt Potenziale und Grenzen von Ridepooling im ländlichen Kontext. \\ \hline
        
        \textcite{heinitz_flexible_2022} & Rurales Ridepooling bewerten & Deutschland & Linie & Flexible Busse & Simulation, Kartenbasierte Raumanalyse (GIS) & synthetisch & Auslastung & Pooling und Optimierung senken Warte- und Reisezeiten sowie Fahrzeug-km im ländlichen Raum. \\ \hline
        
        \textcite{liu_filtering_2024} & Autonomen DRT planen & Frankreich & Linie & Autonom & Matem. Optimierungsmodell & — & — & Studie zeigt Potenziale und Grenzen von Ridepooling im ländlichen Kontext. \\ \hline
        
        \textcite{patricio_assessing_2025} & Ridepooling simulieren & ohne Bezug & Netz & Flexible Busse & Simulation & — & Kosten/Fahrgast-km, Auslastung & Pooling und Optimierung senken Warte- und Reisezeiten sowie Fahrzeug-km im ländlichen Raum. \\ \hline
        
        \textcite{schaefer_acceptance_2022} & Rurales Ridepooling bewerten & Schweiz & Netz & Flexible Busse & — & — & — & Studie zeigt Potenziale und Grenzen von Ridepooling im ländlichen Kontext. \\ \hline
        
        \textcite{si_vehicle_2024} & Pooling/Zuteilung optimieren & USA & Linie & Pooling & Matem. Optimierungsmodell & — & Auslastung & Studie zeigt Potenziale und Grenzen von Ridepooling im ländlichen Kontext. \\ \hline

        \textcite{sorensen_how_2021} & Rurales Ridepooling bewerten & Deutschland & Netz & Flexible Busse & — & — & — & Studie zeigt Potenziale und Grenzen von Ridepooling im ländlichen Kontext. \\ \hline
        
        \textcite{truden_analysis_2021} & First/Last-Mile verbessern & Österreich & Region & First/Last-Mile & Simulation, Matem. Optimierungsmodell & synthetisch & Kosten/Fahrgast-km, Auslastung & Studie zeigt Potenziale und Grenzen von Ridepooling im ländlichen Kontext. \\ \hline
        
        \textcite{yu_optimal_2021} & Fahrplan/Zuteilung optimieren & USA & Linie & Flexible Busse & Literatur-Review & Literatur & Auslastung & Studie zeigt Potenziale und Grenzen von Ridepooling im ländlichen Kontext. \\ \hline
        
        \textcite{zhou_bus-pooling_2025} & Pooling/Zuteilung optimieren & ohne Bezug & Linie & Pooling & Matem. Optimierungsmodell & Realdaten & Ø-Wartezeit, Kosten/Fahrgast-km & Pooling und Optimierung senken Warte- und Reisezeiten sowie Fahrzeug-km im ländlichen Raum. \\ \hline
        
        \textcite{zwick_ride-pooling_2021} & Ridepooling-Konzept bewerten & Polen & Region & Pooling & Simulation, Literatur-Review & Literatur & — & Studie zeigt Potenziale und Grenzen von Ridepooling im ländlichen Kontext. \\ \hline
    \end{xltabular}
\end{landscape}

\begin{landscape}

    \scriptsize
    \setlength{\tabcolsep}{2.2pt}
    \setlength{\arrayrulewidth}{0.1pt}
    \begin{xltabular}{\textwidth}{%
        L{1.8cm}:
        L{1.8cm}:
        L{1.5cm}:
        L{2.3cm}:
        L{2.3cm}:
        L{2.0cm}:
        L{1.5cm}:
        L{2.4cm}:
        L{6.0cm}:
    }
        \caption{Forschungsergebnisse zu ridepooling — urban}\label{tab:rp-urban}\\ 
        \hline
        \LH{Paper} & \LH{Zielsetzung} & \LH{Region /\\Land} & \LH{Betrachtungs-\\ebene} &
        \LH{Fokus / \\ Anwendungsfeld} & \LH{Methode} & \LH{Daten} & \LH{KPI} & \LH{Zentrale\\Erkenntnisse} \\
        \Xhline{0.6pt}
        \endfirsthead

        \multicolumn{9}{l}{\small\itshape Fortsetzung von Tabelle~\ref{tab:rp-urban}.}\\[0.6\baselineskip]
        \hline
        \LH{Paper} & \LH{Zielsetzung} & \LH{Region /\\Land} & \LH{Betrachtungs-\\ebene} &
        \LH{Fokus / \\ Anwendungsfeld} & \LH{Methode} & \LH{Daten} & \LH{KPI} & \LH{Zentrale\\Erkenntnisse} \\
        \Xhline{0.6pt}
        \endhead

        \hline
        \multicolumn{9}{r}{\small\itshape Fortsetzung auf der nächsten Seite}
        \endfoot

        \hline
        \endlastfoot

        \textcite{agatz_dynamic_2011} & Ridepooling simulieren & Niederlande & Netz & Feeder & Simulation, Matem. Optimierungsmodell & — & Kosten/Fahrgast-km, Auslastung & Pooling und Optimierung reduzieren Warte-/Reisezeiten und Fahrzeug-km im urbanen Kontext deutlich. \\ \hline
        
        \textcite{alonso-gonzalez_value_2020} & Urbanes Ridepooling bewerten & Niederlande & Region & Flexible Busse & Matem. Optimierungsmodell & — & Kosten/Fahrgast-km, Auslastung & Zielkonflikt zwischen Servicequalität und Bündelungsgrad wird sichtbar. \\ \hline
        
        \textcite{alonso-gonzalez_what_2021} & Ridepooling simulieren & ohne Bezug & Region & Pooling & Simulation & — & Kosten/Fahrgast-km, Auslastung & Ridepooling erhöht Abdeckung und Auslastung gegenüber Solo-DRT. \\ \hline
        
        \textcite{alonso-mora_-demand_2017} & Fahrplan/Zuteilung optimieren & USA & Netz & Studierende & Literatur-Review & Literatur & Auslastung & Studie zeigt Potenziale und Grenzen urbanen Ridepoolings. \\ \hline
        
        \textcite{asatryan_ridepooling_2023} & Pooling/Zuteilung optimieren & Deutschland & Haltestelle & Pooling & Simulation & synthetisch & Auslastung & Studie zeigt Potenziale und Grenzen urbanen Ridepoolings. \\ \hline
        
        \textcite{basu_automated_2018} & Ridepooling simulieren & ohne Bezug & Region & Flexible Busse & Matem. Optimierungsmodell & — & Auslastung & Studie zeigt Potenziale und Grenzen urbanen Ridepoolings. \\ \hline
        
        \textcite{bischoff_city-wide_2017} & Ridepooling simulieren & Deutschland & Netz & Pooling & Simulation, Matem. Optimierungsmodell & — & Reisezeit & Pooling und Optimierung reduzieren Warte-/Reisezeiten und Fahrzeug-km im urbanen Kontext deutlich. \\ \hline

        \textcite{chen_exploring_2021} & Ridepooling-Konzept bewerten & ohne Bezug & Region & Flexible Busse & Empirie & Mix & Fahrzeug-km, Kosten/Fahrgast-km & Ridepooling erhöht Abdeckung und Auslastung gegenüber Solo-DRT. \\ \hline
        
        \textcite{engelhardt_speed-up_2020} & Pooling/Zuteilung optimieren & Deutschland & Region & Pooling & Matem. Optimierungsmodell & — & Auslastung & Studie zeigt Potenziale und Grenzen urbanen Ridepoolings. \\ \hline
        
        \textcite{engelhardt_predictive_2023} & Pooling/Zuteilung optimieren & Deutschland & Region & Pooling & — & — & Auslastung & Ridepooling erhöht Abdeckung und Auslastung gegenüber Solo-DRT. \\ \hline
        
        \textcite{fagnant_dynamic_2018} & Autonomen DRT planen & USA & Netz & First/Last-Mile & Simulation & — & Auslastung & Studie zeigt Potenziale und Grenzen urbanen Ridepoolings. \\ \hline
        
        \textcite{fehn_ride-parcel-pooling_2023} & Pooling/Zuteilung optimieren & Deutschland & Region & Pooling & Kartenbasierte Raumanalyse (GIS) & — & Flottengröße, Auslastung & Studie zeigt Potenziale und Grenzen urbanen Ridepoolings. \\ \hline
        
        \textcite{fielbaum_improving_2024} & Pooling/Zuteilung optimieren & Niederlande & Linie & Pooling & Matem. Optimierungsmodell & — & Auslastung & Ridepooling erhöht Abdeckung und Auslastung gegenüber Solo-DRT. \\ \hline
        \midrule
        
        \textcite{garus_exploring_2024} & Urbanes Ridepooling bewerten & Italien & Region & Flexible Busse & Matem. Optimierungsmodell & — & — & Studie zeigt Potenziale und Grenzen urbanen Ridepoolings. \\ \hline
        
        \textcite{jung_dynamic_2016} & Fahrplan/Zuteilung optimieren & USA & Netz & Flexible Busse & Simulation, Matem. Optimierungsmodell & — & Ø-Wartezeit, Auslastung & Ridepooling erhöht Abdeckung und Auslastung gegenüber Solo-DRT. \\ \hline
        
        \textcite{konig_analyzing_2018} & Pooling/Zuteilung optimieren & Deutschland & Netz & Pooling & — & — & CO\(_2\) & Studie zeigt Potenziale und Grenzen urbanen Ridepoolings. \\ \hline
        
        \textcite{leich_should_2019} & Ridepooling-Konzept bewerten & ohne Bezug & Region & Autonom & Simulation, Matem. Optimierungsmodell & Literatur & — & Studie zeigt Potenziale und Grenzen urbanen Ridepoolings. \\ \hline
        
        \textcite{liang_automated_2020} & Ridepooling-Konzept bewerten & Niederlande & Netz & First/Last-Mile & Matem. Optimierungsmodell & — & Reisezeit & Studie zeigt Potenziale und Grenzen urbanen Ridepoolings. \\ \hline
        
        \textcite{liu_filtering_2024} & Autonomen DRT planen & Frankreich & Linie & Autonom & Matem. Optimierungsmodell & — & — & Studie zeigt Potenziale und Grenzen urbanen Ridepoolings. \\ \hline
        
        \textcite{liu_quantifying_2025} & Pooling/Zuteilung optimieren & China & Netz & Pooling & Fallstudie & — & Auslastung & Studie zeigt Potenziale und Grenzen urbanen Ridepoolings. \\ \hline

        \textcite{liu_analysis_2015} & Ridepooling-Konzept bewerten & China & Netz & Flexible Busse & Matem. Optimierungsmodell & — & Auslastung & Studie zeigt Potenziale und Grenzen urbanen Ridepoolings. \\ \hline
        
        \textcite{lobel_detours_2025} & Pooling/Zuteilung optimieren & USA & Netz & Pooling & Matem. Optimierungsmodell & — & — & Studie zeigt Potenziale und Grenzen urbanen Ridepoolings. \\ \hline
        
        \textcite{schatzmann_investigating_2023} & Pooling/Zuteilung optimieren & Deutschland & Netz & Pooling & Matem. Optimierungsmodell, Empirie & Mix & — & Studie zeigt Potenziale und Grenzen urbanen Ridepoolings. \\ \hline
        
        \textcite{shen_integrating_2018} & First/Last-Mile verbessern & USA & Netz & First/Last-Mile & Simulation, Matem. Optimierungsmodell & — & Auslastung & Studie zeigt Potenziale und Grenzen urbanen Ridepoolings. \\ \hline
        
        \textcite{soza-parra_shareability_2024} & Pooling/Zuteilung optimieren & UK & Region & Pooling & Kartenbasierte Raumanalyse (GIS) & — & Auslastung & Studie zeigt Potenziale und Grenzen urbanen Ridepoolings. \\ \hline
        
        \textcite{stiglic_benefits_2015} & Ridepooling-Konzept bewerten & Niederlande & Netz & Pooling & Matem. Optimierungsmodell & — & — & Studie zeigt Potenziale und Grenzen urbanen Ridepoolings. \\ \hline
        
        \textcite{stiglic_enhancing_2018} & Urbanes Ridepooling bewerten & Niederlande & Netz & Studierende & Matem. Optimierungsmodell & — & — & Ridepooling erhöht Abdeckung und Auslastung gegenüber Solo-DRT. \\ \hline

        \textcite{vosooghi_shared_2019} & Autonomen DRT planen & Frankreich & Netz & Autonom & Simulation, Matem. Optimierungsmodell & — & Auslastung & Studie zeigt Potenziale und Grenzen urbanen Ridepoolings. \\ \hline
        
        \textcite{wallar_vehicle_2018} & Urbanes Ridepooling bewerten & ohne Bezug & Linie & Flexible Busse & Matem. Optimierungsmodell & — & Auslastung & Zielkonflikt zwischen Servicequalität und Bündelungsgrad wird sichtbar. \\ \hline
        
        \textcite{wen_transit-oriented_2018} & Autonomen DRT planen & USA & Netz & Autonom & Simulation, Matem. Optimierungsmodell & — & Auslastung & Studie zeigt Potenziale und Grenzen urbanen Ridepoolings. \\ \hline
        
        \textcite{zwick_ride-pooling_2021} & Ridepooling-Konzept bewerten & Deutschland & Region & Pooling & Simulation, Literatur-Review & Literatur & — & Studie zeigt Potenziale und Grenzen urbanen Ridepoolings. \\ \hline
        
        \textcite{zwick_analysis_2020} & Pooling/Zuteilung optimieren & Schweiz & Netz & Pooling & Matem. Optimierungsmodell & synthetisch & — & Ridepooling erhöht Abdeckung und Auslastung gegenüber Solo-DRT. \\ \hline
        
        \textcite{zwick_ride-pooling_2022} & Pooling/Zuteilung optimieren & Deutschland & Region & Pooling & Simulation, Kartenbasierte Raumanalyse (GIS) & — & Flottengröße, Fahrzeug-km & Pooling und Optimierung reduzieren Warte-/Reisezeiten und Fahrzeug-km im urbanen Kontext deutlich. \\ \hline
                
    \end{xltabular}
\end{landscape}
Nachdem die Forschungsgebiete der zentralen Begriffe nun beleuchtet wurden, werden im Folgenden Kapitel Arbeiten aufgezeigt, die mit dem von Schulz \& Vlćek vorgestellten Modell verwandt sind, um das Modell von Schulz \& Vlćek einzuordnen und abzugrenzen.

Es ist zwar nicht Teil des Untersuchungsgebietes dieser Arbeit, da Schulz \& Vlćek einen fest vorgegebenen Fahrplan für ihr Modell nutzen, aber es sei an dieser Stelle auf die Forschung im Bereich für die Vorhersage der Ankunftszeiten, etc. mit Quelle1, Quelle 2 usw. verwiesen

\textbf{DAS HIER IST ÜBERLEITUNG MIT DEM "REINEN" DARP UND VRP ZU DEN SEMI-FLEXIBLEN SYSTEMEN}
Das, in dieser Arbeit zu testende, Modell von Schulz \& Vlćek kombiniert Elemente eines klassischen Linienbus-Services mit der Flexibilität die Routen anhand der gestellten Nachfrage zu optimieren. Die Problemstellung von Schulz \& Vlćek ist mit denen des Dial-a-Ride Problems (DARP) und des Vehicle Routing Problems (VRP) verwandt, \textbf{lässt sich aber nicht klar einer der beiden Problemfamilien zuordnen ???}. Das DARP wird bereits seit Jahrzenten untersucht \parencite{psaraftis_dynamic_1980}, die Forschung zu diesem Problem ist dementsprechend sehr umfassend. Den wohl aktuellsten Überblick zum DARP und seinen Vairanten geben \textcite{molenbruch_typology_2017} und \textcite{ho_survey_2018}. 

Auch das VRP ist ein seit Jahrzehnten erforschtes Problem \parencite{orloff_fundamental_1974}. Das VRP allein ist ein so breit und intensiv erforschtes Problem, dass es dazu über 150 Review-Paper für spezifische Varianten oder Aspekte gibt. Es wird mit dem Searchstring \textbf{XXXX (siehe Anhang XXX)} auf die Literatur dieser Reviews \textbf{und damit indirekt auf die einzelnen Veröffentlichungen ???} verwiesen. Reviews des Problems haben unter anderem \textcite{braekers_vehicle_2016} und \textcite{vidal_concise_2020} gegeben. 

\textbf{Irgendwie noch erwähnen, dass auch die CVRP Variante ein breit erforschtes Thema ist, daher nicht tiefer im Detail betrachtet}

Überleitung: Da Mdoell von Schulz \& Vlćek semiflexibel und nicht so richtig ein reines DARP pder VRP wird anschließend auf die verwandten Problemstellungen verwiesen und eingeordnet.

%%%%%%%%%%%%%%%%%%%%
%%%%%%%%%%%%%%%%%%%%
%%%%%%%%%%%%%%%%%%%%
%%%%%%%%%%%%%%%%%%%%
\newpage
\section{Semi-flexible Systeme - Einordnung des zu betrachtenden Modells}
\label{sec:Einordnung}
\label{sec:2.2}

\textbf{GLIEDERUNG DIESES ABSCHNITTS:}
\begin{enumerate}
    \item Systeme die zwar semi-flexible aber nicht mit BLSP verwandt sind
    \item Verwandte Systeme
\end{enumerate}


\textbf{DIE TABELLE ZU DEN NICHT VERWANDTEN SYSTEMEN EVTL. NICHT AUF EXTRA SEITE}
\begin{landscape}

    \scriptsize
    \setlength{\tabcolsep}{2.2pt}
    \setlength{\arrayrulewidth}{0.1pt}
    \begin{xltabular}{\textwidth}{%
        L{1.8cm}:
        L{1.8cm}:
        L{2.2cm}:
        L{2.3cm}:
        L{1.2cm}:
        L{2.0cm}:
        L{2.0cm}:
        L{2.4cm}:
        L{6.0cm}:
    }
        \caption{Forschungsergebnisse zu nicht BLSP-verwandten Systemen}\label{tab:nicht_verwandt}\\ 
        \hline
        \LH{Paper} & \LH{Zielsetzung} & \LH{Semi-Flexible\\Mechanik} & \LH{Bedienmodus} &
        \LH{Stopps} & \LH{Raumbezug} & \LH{Nachfrage-\\berücksichtigung} & \LH{Unterscheidung \\zum BLSP} & \LH{Zentrale Erkenntnisse} \\
        \Xhline{0.6pt}
        \endfirsthead

        \multicolumn{9}{l}{\small\itshape Fortsetzung von Tabelle~\ref{tab:nicht_verwandt}.}\\[0.6\baselineskip]
        \hline
        \LH{Paper} & \LH{Zielsetzung} & \LH{Semi-Flexible\\Mechanik} & \LH{Bedienmodus} &
        \LH{Stopps} & \LH{Raumbezug} & \LH{Nachfrage-\\berücksichtigung} & \LH{Unterscheidung \\zum BLSP} & \LH{Zentrale Erkenntnisse} \\
        \Xhline{0.6pt}
        \endhead

        \hline
        \multicolumn{9}{r}{\small\itshape Fortsetzung auf der nächsten Seite}
        \endfoot

        \hline
        \endlastfoot

        \textcite{abdelwahed_balancing_2023} & Komfort und Emissionen balancieren & & & & & & & Dynamische Netze verbessern Komfort-Emissions-Trade-off \\ \hline

        \textcite{edward_kim_optimal_2019} & Konzept evaluieren & Zonen-Flex-Route & Headway-basiert & flexibel & Zone & stochastisch/unsicher & reine First/Last-Mile & Ansätze reduzieren Wartezeiten und Betriebskosten bei moderater Flotte deutlich. \\ \hline
        
        \textcite{kim_optimal_2021} & Planung/Dimensionierung & Zonen-Flex-Route & Headway-basiert & flexibel & Zone & stationär & reine First/Last-Mile & Ansätze reduzieren Wartezeiten und Betriebskosten bei moderater Flotte deutlich. \\ \hline
        
        \textcite{kim_conventional_2012} & Konzept evaluieren & Zonen-Flex-Route & Headway-basiert & teils-fix & Zone & mehrperiodig & Zonen/Korridor-basiert & Ansätze reduzieren Wartezeiten und Betriebskosten bei moderater Flotte deutlich. \\ \hline
        
        \textcite{lu_flexible_2016} & Verbesserung der Erreichbarkeit & Flex-Feeder & Headway-basiert & teils-fix & Feeder→Hub & stationär & reine First/Last-Mile & Ansätze reduzieren Wartezeiten und Betriebskosten bei moderater Flotte deutlich. \\ \hline
        
        \textcite{mehran_analytical_2020} & Verbesserung der Servicequalität & Variabler Servicetyp & bedarfsorientiert & fix & Netz & mehrperiodig & Zonen/Korridor-basiert & Ansätze reduzieren Wartezeiten und Betriebskosten bei moderater Flotte deutlich. \\ \hline
        
        \textcite{mishra_optimal_2023} & Planung/Dimensionierung & Zonen-Flex-Route & fester Takt & fix & Zone & stochastisch/unsicher & reine First/Last-Mile & Ansätze reduzieren Wartezeiten und Betriebskosten bei moderater Flotte deutlich. \\ \hline
        
        \textcite{mishra_cost_2024} & Konzept evaluieren & Flex-Feeder & Headway-basiert & teils-fix & Feeder→Hub & mehrperiodig & reine First/Last-Mile & Ansätze reduzieren Wartezeiten und Betriebskosten bei moderater Flotte deutlich. \\ \hline

        \textcite{ng_autonomous_2023} & Konzept evaluieren & Zonen-Flex-Route & bedarfsorientiert & teils-fix & Zone & stationär & Zonen/Korridor-basiert & Semi-flexible Varianten zeigen Vorteile gegenüber starren Konzepten in ausgewählten Szenarien. \\ \hline
        
        \textcite{qiu_demi-flexible_2015} & Konzept evaluieren & Zonen-Flex-Route & bedarfsorientiert & teils-fix & Netz & stochastisch/unsicher & Zonen/Korridor-basiert & Semi-flexible Varianten zeigen Vorteile gegenüber starren Konzepten in ausgewählten Szenarien. \\ \hline
        
        \textcite{sadrani_vehicle_2022} & Minimierung der Wartezeit & Korridor-Checkpoints & hybrid & teils-fix & Korridor & stochastisch/unsicher & Zonen/Korridor-basiert & Ansätze reduzieren Wartezeiten und Betriebskosten bei moderater Flotte deutlich. \\ \hline
        
        \textcite{stiglic_enhancing_2018} & Konzept evaluieren & Ride-Sharing-Kopplung & bedarfsorientiert & teils-fix & Netz & stationär & Zonen/Korridor-basiert & Erreichbarkeit und Servicequalität steigen gegenüber rein festen Linien merklich an. \\ \hline
        
        \textcite{wang_joint_2021} & Konzept evaluieren & Zonen-Flex-Route & bedarfsorientiert & flexibel & Zone & stationär & reine First/Last-Mile & Semi-flexible Varianten zeigen Vorteile gegenüber starren Konzepten in ausgewählten Szenarien. \\ \hline
        
        \textcite{zheng_slack_2018} & Konzept evaluieren & Korridor-Checkpoints & bedarfsorientiert & teils-fix & Netz & stochastisch/unsicher & Zonen/Korridor-basiert & Ansätze reduzieren Wartezeiten und Betriebskosten bei moderater Flotte deutlich. \\ \hline
        
                
    \end{xltabular}
\end{landscape}

Irgendwie Aussage darüber, dass in der Forschung oft gesagt wird, dass durch autonome vehicle erst so richtig ermöglichen demand responsive zu agiere, um fixed-schedule abzulösen (sieh)

Modell von Schulz \& Vlćek ist Kombination aus Linienverkehr \& on-demand, daher: semi-flexible Systeme --> \textbf{HIER EVENTUELL mehrere Quellen die Felxibilitt definieren?}

\textbf{EINORDNUNG unter anderem ÜBER EINEN TABELARISCHEN VERGLEICH IN HINBLICK AUF OPTIMIERUNGSZIELE, aber auch Text: Textlich beschreiben welche aspekte für die abgrenzung des BLSP wichtig sind und dann auf tabelle verweisen, das diese aspekte dahingehen gegenübergestellt sind.}

\begin{landscape}
    \begingroup
    \scriptsize
    
    % ---- Fallbacks (stören nicht, falls schon definiert) ----
    \makeatletter
    \@ifundefined{LH}{\newcommand{\LH}[1]{\textbf{#1}}}{}%
    \@ifundefined{rotH}{\newcommand{\rotH}[1]{\rotatebox{90}{\makecell[l]{#1}}}}{}% linksbündige Rot-Köpfe
    \@ifundefined{yes}{\newcommand{\yes}{\textbf{X}}}{}%
    \@ifundefined{no}{\newcommand{\no}{\textbf{-}}}{}%
    \makeatother
    \providecommand{\Xhline}[1]{\specialrule{#1}{0pt}{0pt}}% falls \Xhline fehlt
    
    \setlength{\tabcolsep}{2pt}
    
    \begin{xltabular}{\textwidth}{%
      L{1.7cm} L{2.5cm} L{1.9cm} L{1.7cm}% 4 Textspalten
      *{5}{C{0.30cm}}% 5 Netz/Bedienlogik
      *{8}{C{0.30cm}}% 8 Nebenbedingungen
      C{0.30cm} C{0.30cm} C{0.30cm} C{0.30cm}% 4 Lösungsmethodik (Exakt, Metaheuristik, Dekomposition, Solver)
      L{1.7cm}% Datengrundlage (Text)
      L{5.6cm}% Zentrale Erkenntnisse (Text)
    }
    
    \caption{Forschungsergebnisse zu BLSP-verwandten Systemen}\label{tab:verwandteSys}\\
    \toprule
    % ---------- Kopf 1 ----------
    \multirow{2}{*}{\LH{Paper}} &
    \multirow{2}{*}{\LH{Zielsetzung}} &
    \multirow{2}{*}{\LH{Optimierungsziel}} &
    \multirow{2}{*}{\LH{Nachfrage-\\dynamik}} &
    \multicolumn{5}{c}{\LH{Netz \& Bedienlogik}} &
    \multicolumn{8}{c}{\LH{Nebenbedingungen}} &
    % (Modellklasse entfällt)
    \multicolumn{4}{c}{\LH{Lösungsmethodik}} &
    \multirow{2}{*}{\LH{Daten-\\grundlage}} &
    \multirow{2}{*}{\LH{Zentrale Erkenntnisse}} \\
    % ---------- Kopf 2 ----------
    & & & &
    \rotH{Feste Stopsequenz} &
    \rotH{Fahrplansegmente} &
    \rotH{Multi-Linien/Netz} &
    \rotH{Transfers modelliert} &
    \rotH{Pickup/Delivery} &
    \rotH{Kapazität} &
    \rotH{Max. Mitfahrzeit} &
    \rotH{Feste Abfahrtszeiten} &
    \rotH{Routen-/Schichtdauer} &
    \rotH{Fahrer/Schichten} &
    \rotH{Pausen/Ruhezeiten} &
    \rotH{Heterogene Flotte} &
    \rotH{Multi-Depot} &
    \rotH{Exakt} &
    \rotH{Metaheuristik} &
    \rotH{Dekomposition} &
    \rotH{Solver} &
    & \\
    \midrule
    \endfirsthead
    
    \multicolumn{23}{l}{\small\itshape Fortsetzung von Tabelle~\ref{tab:verwandteSys}.}\\[0.25\baselineskip]
    \toprule
    \multirow{2}{*}{\LH{Paper}} &
    \multirow{2}{*}{\LH{Zielsetzung}} &
    \multirow{2}{*}{\LH{Optimierungsziel}} &
    \multirow{2}{*}{\LH{Nachfrage-\\dynamik}} &
    \multicolumn{5}{c}{\LH{Netz \& Bedienlogik}} &
    \multicolumn{8}{c}{\LH{Nebenbedingungen}} &
    \multicolumn{4}{c}{\LH{Lösungsmethodik}} &
    \multirow{2}{*}{\LH{Daten-\\grundlage}} &
    \multirow{2}{*}{\LH{Zentrale Erkenntnisse}} \\
    & & & &
    \rotH{Feste Stopsequenz} &
    \rotH{Fahrplansegmente} &
    \rotH{Multi-Linien/Netz} &
    \rotH{Transfers modelliert} &
    \rotH{Pickup/Delivery} &
    \rotH{Kapazität} &
    \rotH{Max. Mitfahrzeit} &
    \rotH{Feste Abfahrtszeiten} &
    \rotH{Routen-/Schichtdauer} &
    \rotH{Fahrer/Schichten} &
    \rotH{Pausen/Ruhezeiten} &
    \rotH{Heterogene Flotte} &
    \rotH{Multi-Depot} &
    \rotH{Exakt} &
    \rotH{Metaheuristik} &
    \rotH{Dekomposition} &
    \rotH{Solver} &
    & \\
    \midrule
    \endhead
    
    \midrule
    \multicolumn{23}{r}{\small\itshape Fortsetzung auf der nächsten Seite} \\
    \endfoot
    
    \bottomrule
    \endlastfoot
    %%%%%%%%%%%%%%%%%%
    % PER NOTIZ HIER %
    %%%%%%%%%%%%%%%%%%
    \textcite{bakas_flexible_2016} & Flex-Dienst im Liniennetz optimieren & min Kosten & Statisch &
    \no & \no & \yes & \yes & \yes &
    \yes & \yes & \yes & \yes & \no & \no &
    \no & \no & \no & \yes & – & – & Synthetische Daten (bis 500 Nutzer) & Tabu-Search reduziert Kosten gegenüber Fixnetz \\ \hline
    
    \textcite{marinelli_integrated_2024} & Integrierten On-Demand-Bus planen & min Reisezeit+Wartezeit & Statisch &
    \no & \no & \yes & \no & \yes &
    \yes & \yes & \yes & \yes & \no & \no &
    \yes & \no & \yes & \no & – & – & Fallstudie/Simulierte Nachfrage & MILP integriert Kapazität, Zeitfenster, Heterogenität \\ \hline
    
    \textcite{melis_integrated_2024} & On-Demand mit Fixnetz integrieren & min Mitfahrzeit & Statisch &
    \no & \no & \yes & \yes & \no &
    \yes & \yes & \yes & \no & \no & \no &
    \no & \no & \no & \yes & – & – & Synthetische/kalibrierte Daten & Integration senkt Nutzer-Mitfahrzeiten signifikant \\ \hline
    
    \textcite{melis_static_2022} & Statisches On-Demand-Routing lösen & min Kosten/Distanz & Statisch &
    \no & \no & \no & \no & \yes &
    \yes & \yes & \yes & \yes & \no & \no &
    \no & \no & \no & \yes & – & – & Synthetische Benchmarks & ALNS liefert hochwertige Lösungen für ODBRP \\ \hline
    
    \textcite{pei_real-time_2019} & Flexible Linienlänge in Echtzeit planen & min Kosten (Nutzer+Betreiber) & Dynamisch &
    \yes & \no & \no & \no & \no &
    \yes & \no & \yes & \no & \no & \no &
    \no & \no & \yes & \no & – & Xpress & Realdaten (Guangzhou, Lin. 15) & Echtzeit-Anpassung reduziert Gesamtzeiten >10\,\% \\ \hline

    %%%%%%%%%%%%%%%
    % DARP - capacity
    %%%%%%%%%%%%%%%
    \textcite{charikar_finite_1998} & Tourlänge approximieren & min Strecke & Statisch & \no & \no & \no & \no & \yes & \yes & \no & \no & \no & \no & \no & \no & \no & \no & \no & – & – & Theorie (keine Daten) & Erster Approximationsalgorithmus mit Garantien \\ \hline

    \textcite{maalouf_new_2014} & Dynamische Zuweisung verbessern & multiobj & Dynamisch & \no & \no & \no & \no & \yes & \yes & \no & \yes & \no & \no & \no & \no & \no & \no & \no & – & – & Synthetische/Simulierte Anfragen & Fuzzy-Regeln reduzieren Distanz und Fahrzeit \\ \hline
    
    \textcite{madsen_heuristic_1995} & DARPTW heuristisch lösen & multiobj & Statisch & \no & \no & \no & \no & \yes & \yes & \no & \yes & \no & \yes & \yes & \no & \no & \no & \no & – & – & Realdaten (Kopenhagen CFFS) & Insertion-Heuristik REBUS performant im Betrieb \\ \hline
    
    \textcite{tang_ant_2021} & ACO für DARPTW entwickeln & min Fahrzeuge+Strecke & Statisch & \no & \no & \no & \no & \yes & \yes & \yes & \yes & \no & \no & \no & \no & \no & \no & \yes & – & – & Benchmark-Instanzen (DARPTW) & ACO erreicht wettbewerbsfähige DARPTW-Lösungen \\ \hline
    
    \textcite{tellez_fleet_2018} & Flottengröße und -mix optimieren & min Kosten & Statisch & \no & \no & \no & \no & \yes & \yes & \yes & \yes & \yes & \no & \no & \yes & \yes & \no & \yes & – & MILP (nicht spezifiziert) & Synthetische Instanzen & Rekonfiguration senkt Kosten signifikant \\ \hline
    
    \textcite{wong_solution_2006} & Mehrdimensionale Kapazität behandeln & min Gesamtkosten & Statisch & \no & \no & \no & \no & \yes & \yes & \yes & \yes & \yes & \no & \no & \no & \no & \no & \yes & – & – & Simulation/Beispieldaten & Tabu-Thresholding verbessert Lösungsqualität \\ \hline
    

    %%%%%%%%%%%%%%%
    % DARP - integrated
    %%%%%%%%%%%%%%%

    \textcite{amor_new_2019} & IDARP mit Fahrplan formulieren & min Kosten & Statisch & \no & \yes & \yes & \yes & \yes & \yes & \yes & \yes & \no & \no & \yes & \yes & \no & \yes & \yes & – & Xpress & Realdaten/Fallstudie & Methoden senken durchschnittliche Wartezeit \\ \hline
    
    \textcite{hall_integrated_2009} & IDARP definieren und lösen & min Kosten & Statisch & \no & \no & \yes & \yes & \yes & \yes & \yes & \yes & \no & \no & \yes & \no & \no & \no & \yes & – & CPLEX & Synthetische Benchmarks & Methoden senken durchschnittliche Wartezeit \\ \hline
    
    \textcite{hassan_integrated_2021} & Fixnetz-Auswahl im IDARP optimieren & min Kosten & Statisch & \no & \yes & \yes & \yes & \yes & \yes & \yes & \yes & \yes & \yes & \yes & \no & \no & \yes & \yes & – & Gurobi & Synthetische Benchmarks & Kosten sinken bei vergleichbarer Servicequalität \\ \hline
    
    \textcite{posada_integrated_2017} & IDARP mit Fahrplan formulieren & min Kosten & Statisch & \no & \yes & \yes & \yes & \yes & \yes & \no & \yes & \yes & \no & \yes & \yes & \no & \yes & \yes & – & Xpress & Realdaten/Fallstudie & Methoden senken durchschnittliche Wartezeit \\ \hline
    
    \textcite{saathoff_practice-oriented_2025} & Praxisnahe IDARP-Lösungen entwickeln & min Wartezeit & Statisch & \no & \no & \yes & \yes & \yes & \yes & \yes & \yes & \no & \no & \yes & \no & \no & \yes & \yes & – & CPLEX & Synthetische Benchmarks & Methoden senken durchschnittliche Wartezeit \\ \hline
    
    \textcite{schenekemberg_hybrid_2025} & Hybridheuristik für integrierten Dienst & min Kosten & Statisch & \no & \yes & \yes & \yes & \yes & \yes & \yes & \yes & \yes & \yes & \yes & \no & \no & \no & \yes & – & Xpress & Realdaten/Fallstudie & Hybridheuristik reduziert Fahrzeugkilometer deutlich \\ \hline
    

    %%%%%%%%%%%%%%%
    % DARP - line-baed
    %%%%%%%%%%%%%%%

    \textcite{lauerbach_complexity_2025} & Wendungen im liDARP zählen & min Wendungen & Statisch & \yes & \no & \yes & \no & \yes & \yes & \yes & \yes & \no & \no & \no & \no & \no & \no & \no & – & – & Theorie (keine Daten) & Wendungszählung weist hohe Komplexität auf \\ \hline
    
    \textcite{reiter_line-based_2024} & liDARP modellieren und lösen & multiobj & Statisch & \yes & \no & \yes & \no & \yes & \yes & \yes & \yes & \no & \no & \no & \no & \no & \yes & \no & – & Gurobi/CPLEX & Synthetische Benchmarks & Event‑basierte MILP respektiert Warte‑ und Reisezeit \\ \hline
    

    %%%%%%%%%%%%%%%
    % DARP - static
    %%%%%%%%%%%%%%%

    \textcite{cordeau_tabu_2003} & DARP effizient lösen & min Kosten & Dynamisch & \no & \yes & \no & \no & \yes & \yes & \yes & \yes & \yes & \yes & \no & \no & \yes & \yes & \yes & CG & – & Synthetische Daten & PD-Ansatz erhöht Flexibilität \\ \hline
    
    \textcite{pfeiffer_new_2022} & DARP effizient lösen & multiobj & Dynamisch & \no & \yes & \no & \no & \yes & \yes & \yes & \yes & \no & \yes & \yes & \yes & \yes & \yes & \yes & – & CPLEX & Nicht spezifiziert & PD-Ansatz erhöht Flexibilität \\ \hline
    
    \textcite{melis_static_2022} & Linienbasierte Bedienung planen & min Kosten & Dynamisch & \yes & \yes & \yes & \no & \yes & \yes & \yes & \yes & \yes & \yes & \yes & \no & \no & \yes & \no & – & – & Synthetische Daten & PD-Ansatz erhöht Flexibilität \\ \hline
    
    \textcite{xiang_fast_2006} & DARP effizient lösen & min Kosten & Dynamisch & \no & \yes & \yes & \no & \yes & \yes & \no & \yes & \yes & \yes & \yes & \no & \no & \yes & \yes & CG & – & Realdaten & PD-Ansatz erhöht Flexibilität \\ \hline
    
    \textcite{zidi_multi-agent_2011} & DARP effizient lösen & multiobj & Dynamisch & \no & \yes & \no & \no & \yes & \yes & \yes & \yes & \no & \yes & \no & \yes & \no & \yes & \yes & – & – & Synthetische Daten & PD-Ansatz erhöht Flexibilität \\ \hline
    
    \textcite{kim_model_2011} & DARP effizient lösen & min Kosten & Dynamisch & \no & \yes & \yes & \no & \yes & \yes & \yes & \yes & \yes & \yes & \no & \yes & \yes & \yes & \yes & – & CPLEX & Realdaten & PD-Ansatz erhöht Flexibilität \\ \hline
    
    \textcite{pfeiffer_alns_2022} & DARP effizient lösen & multiobj & Dynamisch & \no & \yes & \no & \no & \yes & \yes & \yes & \yes & \no & \yes & \yes & \no & \no & \yes & \yes & – & CPLEX & Synthetische Daten & PD-Ansatz erhöht Flexibilität \\ \hline
    
    \textcite{luo_rejected-reinsertion_2007} & DARP effizient lösen & min Kosten & Dynamisch & \no & \yes & \yes & \yes & \yes & \yes & \yes & \yes & \no & \yes & \no & \no & \no & \yes & \yes & CG & – & Realdaten & Integration verbessert Erreichbarkeit/Kosten \\ \hline

    %%%%%%%%%%%%%%%
    % DARP - time windows
    %%%%%%%%%%%%%%%

    \textcite{belhaiza_data_2017} & Fahrtzeit minimieren & min Fahrtzeit & Statisch & \no & \no & \no & \no & \yes & \yes & \yes & \yes & \yes & \no & \no & \no & \no & \no & \yes & -- & eigener & synthetische Daten & Hybrid-VNS verbessert Benchmarks. \\ \hline
    
    \textcite{belhaiza_data-driven_2023} & Mehrziele balancieren & multiobj & Statisch & \no & \no & \no & \no & \yes & \yes & \yes & \yes & \yes & \no & \no & \no & \no & \no & \yes & -- & eigener & Realdaten + Benchmarks & Mehrziel-ALNS liefert starke Lösungen. \\ \hline
    
    \textcite{ben_abdelkrim_mathematical_2023} & Kosten minimieren & min Kosten & Statisch & \no & \no & \no & \no & \yes & \yes & \no & \yes & \no & \no & \no & \no & \no & \yes & \no & -- & LINGO & Realdaten (Fallstudie) & MIP zeigt Fallstudienlösungen. \\ \hline
    
    \textcite{deleplanque_dial--ride_2013} & Kosten minimieren & min Kosten & Statisch & \no & \no & \no & \yes & \yes & \yes & \yes & \yes & \yes & \no & \no & \no & \no & \no & \no & -- & eigener & synthetische Daten & Transfers senken Kosten deutlich. \\ \hline
    
    \textcite{gschwind_effective_2015} & Kosten minimieren & min Kosten & Statisch & \no & \no & \no & \no & \yes & \yes & \yes & \yes & \no & \no & \no & \no & \no & \yes & \no & CG & CPLEX & Benchmarks (Cordeau) & B\&P/CG löst DARP exakt. \\ \hline
    
    \textcite{hame_adaptive_2011} & Unzufriedenheit minimieren & multiobj & Statisch & \no & \no & \no & \no & \yes & \yes & \yes & \yes & \no & \no & \no & \no & \no & \yes & \no & -- & eigener & synthetische Daten & Exakt für Single-Vehicle-DARP. \\ \hline
    
    \textcite{jaw_heuristic_1986} & Mehrziele balancieren & multiobj & Statisch & \no & \no & \no & \no & \yes & \yes & \yes & \yes & \no & \no & \no & \no & \no & \no & \no & -- & eigener & Realdaten & Insertion-Heuristik skaliert gut. \\ \hline

    \textcite{madsen_heuristic_1995} & Kosten \& Unzufriedenheit minimieren & multiobj & Statisch & \no & \no & \no & \no & \yes & \yes & \no & \yes & \no & \no & \no & \no & \no & \no & \no & -- & eigener & Realdaten (CFFS) & Heuristik skaliert für Großinstanzen. \\ \hline
    
    \textcite{psaraftis_exact_1983} & Gesamtbedienzeit minimieren & min Gesamtdauer & Statisch & \no & \no & \no & \no & \yes & \yes & \no & \yes & \no & \no & \no & \no & \no & \yes & \no & -- & eigener & synthetische Daten & Exakter Algorithmus für Single-Vehicle DARPTW. \\ \hline
    
    \textcite{urra_hyperheuristic_2015} & Kosten \& Ride-Time minimieren & multiobj & Statisch & \no & \no & \no & \no & \yes & \yes & \yes & \yes & \no & \no & \no & \no & \no & \no & \yes & -- & eigener & synthetische Benchmarks & Hyperheuristik verbessert Benchmarks konsistent. \\ \hline
    
    \textcite{yi_online_2005} & Distanz minimieren & min Distanz & Dynamisch & \no & \no & \no & \no & \yes & \yes & \no & \yes & \no & \no & \no & \no & \no & \no & \no & -- & eigene Online-Strategien & theoretische Analyse & Strategien mit Wettbewerbsraten analysiert. \\ \hline
    
    \textcite{yi_online_2006} & Bedienrate maximieren & max bediente Anfragen & Dynamisch & \no & \no & \no & \no & \yes & \yes & \no & \yes & \no & \no & \no & \no & \no & \no & \no & -- & eigene Online-Strategien & theoretische Analyse & Garantien für bediente Anfragen gezeigt. \\ \hline
    
    \textcite{yi_online_2009} & Bedienrate maximieren & max bediente Anfragen & Dynamisch & \no & \no & \no & \no & \yes & \yes & \no & \yes & \no & \no & \no & \no & \no & \no & \no & -- & eigene Online-Strategien & theoretische Analyse & Ungleiche Fenster: neue Wettbewerbsraten. \\ \hline
    

    %%%%%%%%%%%%%%%
    % VRP - CVRP
    %%%%%%%%%%%%%%%

    \textcite{anuar_multi-depot_2021} & Kosten unter Unsicherheit & min erwartete Kosten & Stochastisch & \no & \no & \no & \no & \no & \yes & \no & \yes & \no & \no & \no & \no & \yes & \yes & \yes & – & CPLEX & Realdaten + Benchmarks & Stochastik via Straßenkapazität berücksichtigt. \\ \hline

    \textcite{banerjee_single_2023} & Kosten minimieren & min Kosten & Statisch & \no & \no & \no & \no & \no & \yes & \no & \no & \no & \no & \no & \yes & \no & \yes & \no & — & Gurobi & synthetische Benchmarks (Golden) & Heterogene Flotte mit SD+PickUp. \\ \hline
    
    \textcite{bernardo_achieving_2023} & Robustheit erhöhen & min erwartete Kosten & Stochastisch & \no & \no & \no & \no & \no & \yes & \no & \yes & \no & \no & \no & \no & \no & \yes & \no & — & Xpress & Realdaten + Benchmarks & Robuste Lösungen bei unsicherer Nachfrage. \\ \hline
    
    \textcite{birtolo_capacity_2025} & Simulationswerkzeug evaluieren & — & Statisch & \no & \no & \no & \no & \no & \yes & \no & \yes & \no & \no & \no & \no & \yes & \no & \no & — & Xpress & Realdaten + Solomon-Benchmarks & Digitale Simulation für CVRPTW-Netzwerk. \\ \hline
    

    %%%%%%%%%%%%%%%
    % VRP - HVRP
    %%%%%%%%%%%%%%%

    \textcite{amiri_bi-objective_2023} & Kosten \& Emissionen minimieren & multiobj & Statisch & \no & \no & \no & \no & \no & \yes & \no & \yes & \no & \no & \no & \yes & \no & \no & \yes & – & CPLEX & Realdaten + synthetisch & Gemischte Flotte spart Kosten und Emissionen \\ \hline
    
    \textcite{fadda_heterogeneous_2023} & Kosten unter Tiefgangsgrenzen minimieren & min Kosten & Statisch & \no & \no & \no & \no & \no & \yes & \no & \yes & \no & \no & \no & \yes & \yes & \no & \yes & – & eigener & Realdaten + synthetisch & Matheuristik löst maritime HVRP effizient \\ \hline
    
    \textcite{fernando_applying_2022} & Retail-Distribution optimieren & min Kosten & Statisch & \no & \no & \no & \no & \no & \yes & \no & \no & \no & \no & \no & \yes & \yes & \no & \yes & – & OR-Tools & Realdaten & Mehrdepots senken Gesamtkosten im Retail \\ \hline
    
    \textcite{kritikos_heterogeneous_2013} & Kosten mit Überladung minimieren & min Kosten & Statisch & \no & \no & \no & \no & \no & \yes & \no & \yes & \no & \no & \no & \yes & \yes & \no & \yes & – & Xpress & Synthetische Benchmarks & Überladungen über Strafkosten modelliert \\ \hline
    
    \textcite{mancini_real-life_2016} & Mehrperiodische Distribution planen & min Kosten & Statisch & \no & \no & \no & \no & \no & \yes & \no & \yes & \yes & \no & \no & \yes & \yes & \no & \yes & – & Xpress & Realdaten & Mehrperiodenansatz verbessert Ressourcennutzung \\ \hline
    

    %%%%%%%%%%%%%%%
    % VRP - MDVRP
    %%%%%%%%%%%%%%%
    
    \textcite{alinaghian_time-dependent_2022} & Emissionen minimieren & min Emissionen & Zeitabhängig & \no & \no & \no & \no & \no & \yes & \no & \yes & \no & \no & \no & \yes & \yes & \no & \yes & – & CPLEX & Synthetische Daten & Zeitabhängigkeit verbessert Routingqualität \\ \hline
    
    \textcite{azadeh_closeopen_2019} & Gesamtkosten minimieren & min Kosten & Zeitabhängig & \no & \no & \no & \no & \no & \yes & \no & \yes & \no & \no & \no & \yes & \yes & \no & \yes & – & CPLEX & Realdaten + Benchmarks & Zeitabhängigkeit verbessert Routingqualität \\ \hline
    
    \textcite{bettinelli_branch-and-cut-and-price_2011} & Kosten minimieren & min Kosten & Statisch & \no & \no & \no & \no & \no & \yes & \no & \yes & \no & \no & \no & \yes & \yes & \yes & \no & CG & CPLEX & Synthetische Benchmarks & Branch-and-price skaliert effizient \\ \hline
    
    \textcite{dao-tuan_multi-criteria_2018} & Kosten und Emissionen minimieren & multiobj & Stochastisch & \no & \no & \no & \no & \no & \yes & \no & \yes & \no & \no & \no & \yes & \yes & \no & \yes & – & Synthetische Daten & Abwägung Kosten–Emissionen verbessert Planung \\ \hline
    
    \textcite{erdes_multi-depot_2024} & Einsatz von E-Fahrzeugen optimieren & min Kosten & Zeitabhängig & \no & \no & \no & \no & \no & \yes & \no & \yes & \no & \no & \no & \no & \yes & \no & \no & – & CPLEX & Realdaten + Benchmarks & Zeitabhängigkeit verbessert Routingqualität \\ \hline
    

    %%%%%%%%%%%%%%%
    % VRP - MTVRP
    %%%%%%%%%%%%%%%

    \textcite{bernardino_multi-trip_2025} & Kosten und Auslastung minimieren & min Kosten & Zeitabhängig & \no & \no & \no & \no & \no & \yes & \no & \no & \no & \no & \no & \yes & \yes & \no & \yes & – & CPLEX & Realdaten + Benchmarks & Dekomposition und Heuristik liefern starke Lösungen \\ \hline
    
    \textcite{brandao_tabu_1997} & Kosten minimieren & min Kosten & Statisch & \no & \no & \no & \no & \no & \yes & \no & \yes & \no & \no & \no & \yes & \no & \no & \yes & – & eigener & Synthetische Benchmarks & Tabu-Suche löst MTVRP effizient \\ \hline
    
    \textcite{cinar_reduction_2015} & Emissionen reduzieren & min Emissionen & Statisch & \no & \no & \no & \no & \no & \yes & \no & \yes & \yes & \no & \no & \yes & \no & \no & \yes & – & CPLEX & Synthetische Benchmarks & SA senkt Kraftstoffverbrauch signifikant \\ \hline
    
    \textcite{huang_exact_2024} & Gesamtkosten minimieren & min Kosten & Statisch & \no & \no & \no & \no & \yes & \yes & \no & \yes & \no & \no & \no & \no & \no & \yes & \no & DD & CPLEX & Realdaten + Benchmarks & Branch-and-price löst NEMT-MTVRP optimal \\ \hline
    
    \textcite{nguyen_modeling_2022} & Kosten und Fahrzeuge minimieren & min Kosten & Statisch & \no & \no & \no & \no & \no & \yes & \no & \yes & \no & \no & \no & \yes & \yes & \no & \yes & – & eigener & Realdaten + Benchmarks & Heuristiken verbessern Abdeckung bei Multidepots \\ \hline

    %%%%%%%%%%%%%%%
    % VRP - PVRP
    %%%%%%%%%%%%%%%

    \textcite{anityasari_analysing_2025} & Transporteffizienz verbessern & min Kosten & Zeitabhängig & \no & \no & \no & \no & \no & \yes & \no & \no & \no & \no & \no & \no & \no & \no & \no & – & – & Realdaten (Surabaya) & Periodische Planung reduziert Gesamtdistanz \\ \hline
    
    \textcite{rahimi-vahed_fleet-sizing_2015} & Optimale Flottengröße bestimmen & min Kosten & Zeitabhängig & \no & \no & \no & \no & \no & \yes & \no & \no & \no & \no & \no & \no & \yes & \no & \yes & – & eigener & Synthetische Benchmarks & Modularheuristik bestimmt Flottengröße effizient \\ \hline
    
    \textcite{rothenbacher_branch-and-price-and-cut_2019} & Flexible Einsatzpläne optimieren & min Strecke & Zeitabhängig & \no & \no & \no & \no & \no & \yes & \no & \yes & \no & \no & \no & \no & \no & \yes & \no & CG & CPLEX & Synthetische Benchmarks & B\&P\&C löst PVRPTW sehr effizient \\ \hline
    
    \textcite{salamatbakhsh-varjovi_robust_2018} & Robuste Routenplanung entwickeln & min Kosten & Stochastisch & \no & \no & \no & \no & \no & \yes & \no & \yes & \no & \no & \no & \no & \no & \no & \yes & – & eigener; CPLEX & Synthetische Benchmarks & Differential Evolution effizient für robustes PVRPTW \\ \hline
    


    %%%%%%%%%%%%%%%
    % VRP - VRPTW
    %%%%%%%%%%%%%%%
    \textcite{agra_robust_2013} & Robuste Routen planen & min Kosten & Stochastisch & \no & \no & \no & \no & \no & \yes & \no & \yes & \no & \no & \no & \no & \no & \yes & \no & \no & Xpress & Golden & Exakter Ansatz skaliert begrenzt effizient \\ \hline
    
    \textcite{arenas-vasco_effect_2024} & Formulierungen vergleichen & min Kosten & Zeitabhängig & \no & \no & \no & \no & \no & \yes & \no & \no & \no & \no & \no & \yes & \yes & \yes & \no & \no & — & Synthetische Benchmarks & Verbesserte Lösungen gegenüber Benchmarks \\ \hline
    
    \textcite{babaee_tirkolaee_developing_2019} & Stadtreinigung optimieren & min Kosten & Statisch & \no & \no & \no & \no & \no & \yes & \no & \yes & \no & \no & \no & \yes & \no & \yes & \yes & \no & CPLEX & Solomon & Exakter Ansatz skaliert begrenzt effizient \\ \hline
    
    \textcite{barrero_graspvnd_2021} & Heterogene Flotte planen & min Kosten & Statisch & \no & \no & \no & \no & \no & \yes & \no & \yes & \no & \no & \no & \yes & \no & \no & \yes & \no & — & Solomon & GRASP/VND erzielt starke Qualität \\ \hline
    
    \textcite{braysy_well-scalable_2009} & Flottengröße \& Mix planen & min Kosten & Statisch & \no & \no & \no & \no & \no & \yes & \no & \yes & \no & \no & \no & \yes & \no & \no & \yes & \no & — & Solomon \& Homberger & Metaheuristik skaliert sehr gut \\ \hline
    

    %%%%%%%%%%%%%%%
    % VRP - VSP
    %%%%%%%%%%%%%%%
    
    \textcite{carpaneto_branch_1989} & MDVSP kostenminimal lösen & min Kosten & Statisch & \yes & \yes & \yes & \no & \no & \no & \no & \yes & \no & \no & \no & \no & \yes & \yes & \no & — & — & Synthetische Benchmarks & B\&B liefert starke Untergrenzen \\ \hline
    
    \textcite{chau_electric_2024} & E\text{-}Busplanung mit Multi\text{-}Port & min Kosten & Statisch & \yes & \yes & \yes & \no & \no & \no & \no & \yes & \no & \no & \no & \no & \no & \yes & \no & — & CPLEX & Synthetische Benchmarks & Multi\text{-}Port reduziert Lade\text{-}Konflikte \\ \hline
    
    \textcite{gintner_solving_2005} & Praxisfälle groß lösen & min Kosten & Statisch & \yes & \yes & \yes & \no & \no & \no & \no & \yes & \no & \no & \no & \yes & \yes & \no & \no & CG & — & Realdaten & CG skaliert für Praxisfälle \\ \hline
    
    \textcite{gkiotsalitis_exact_2023} & MD\text{-}EBSP exakt lösen & min Kosten & Statisch & \yes & \yes & \yes & \no & \no & \no & \no & \yes & \no & \no & \no & \yes & \yes & \yes & \no & — & — & Synthetische Benchmarks & MILP löst MD\text{-}EBSP bis Optimalität \\ \hline
    
    \textcite{guo_multi-objective_2024} & Emissionen \& Leerfahrten minimieren & multiobj & Statisch & \no & \no & \no & \no & \no & \yes & \no & \yes & \no & \no & \no & \yes & \no & \no & \yes & — & — & Realdaten + Benchmarks & Hybrid\text{-}GWO/GA schlägt Baselines \\ \hline
    
    \textcite{huisman_robust_2004} & Robuste dynamische Umläufe planen & min Kosten & Dynamisch & \yes & \yes & \yes & \no & \no & \no & \no & \yes & \no & \no & \no & \no & \no & \no & \no & — & Realdaten & Robustplanung reduziert Störungen \\ \hline

    \end{xltabular}
    
    \endgroup
\end{landscape}


Hier dann Part mit Aussagen darüber wie sich Verwandschaft zum BLSP definiert und wie sich abgrenzt ????

\textbf{ANTWORTEN VON GPT:}
1. Wodurch definiert sich die Verwandtschaft zum BLSP?
Ein Beitrag ist „BLSP-verwandt“, wenn er wesentliche Strukturmerkmale teilt:
\begin{itemize}
    \item Linienbasiert  Fahrplan: Es existieren vorgegebene Linien, Haltestellen und Zeitpunkte; nicht Tür-zu-Tür.
    \item On-Demand auf Linienebene: Es werden nur Fahrten/Segmente mit positiver, vorab bekannter Nachfrage bedient (Trips/Stopsegmente ohne Nachfrage werden ausgelassen).
    \item Die Planung tour-/fahrplanzeitlich strukturiert (Startzeiten/„Fahrplansegmente“).
    \item Zielgröße „Flotteneinsatz“: Primäres Optimierungsziel ist die Minimierung der benötigten Busse (und implizit Fahrpersonal), nicht z. B. reine Fahrgastzeitminimierung.
    \item ÖPNV-Personenverkehr statt Güter-VRP/DARP-Varianten; BLSP liegt zwischen klassischem Linienverkehr und Ridepooling, bei beibehaltetem Fahrplan (Abgrenzung zum line-based DARP ohne Fahrplan).
\end{itemize}
Je mehr dieser Merkmale erfüllt sind, desto „näher“ am BLSP. Fehlen Linien/Fahrplan (klassisches VRP/DARP) oder handelt es sich um Güterlogistik, ist die Verwandtschaft nur entfernt.

2. Hauptaspekte zur Einordnung (a) und Abgrenzung (b)

\begin{itemize}
    \item Bedienparadigma: Linienbasiert (fixe Reihenfolge) vs. semi-flexibel (Segmente überspringbar) vs. voll flexibel (Ridepooling). BLSP = linienbasiert \& semi-flexibel.
    \item Fahrplanbindung: Zeitplan ist gegeben (BLSP), im Gegensatz zu line-based DARP ohne Fahrplan.
    \item Nachfragewissen \& On-Demand-Regel: Vorab bekannt/angemeldet -> nur nachgefragte Teile werden gefahren.
    \item Netz-Scope: Einzellinie vs. Mehrere Linien mit möglichen Wechseln/Übergängen: Transfers explizit modelliert oder nicht. (BLSP betrachtet mehrere Linien: Transfers werden v. a. in verwandter Literatur diskutiert.)
    \item Zielgrößen: v. a. min \#Busse (BLSP) alternative/ergänzende Ziele wären Wartezeit, Anschlussverluste, Fahrgastzeit, Emissionen.
    \item Restriktionen: Kapazität, Schichten/Pausen (ggf. vorgegeben), Zeitfenster/Fahrplansegmente.
    \item Ressourcen/Flotte: Homogen/heterogen ggf. autonome Busse (kein Fahrpersonalmodell).
    \item Depotstruktur \& Energie: Ein- oder Mehrdepot; (Nicht-)Berücksichtigung von Laden/Tanken. (BLSP: 1 Depot Laden/Tanken nicht modelliert.)
    \item Rechenansatz \& Komplexität: Netzwerkfluss-Formulierung (Setting 1), IP-Modell mit Rücksprungkanten (Setting 2/3), NP-schwer bei Kapazitätsbindung klare „easy/hard“-Grenzen.
\end{itemize}

%%%%%%%%%%%%%%%%%%%%
%%%%%%%%%%%%%%%%%%%%
%%%%%%%%%%%%%%%%%%%%
%%%%%%%%%%%%%%%%%%%%
\section{Offene Forschungsfragen}
\label{sec:2.3}
\label{sec:OffeneForschungsfragen}
\begin{itemize}
    \item welche offenen Punkte aus anderen Papern greifen Schulz \& Vlćek eventuell auf? -> angeblich systematische szenarien zur wirkung uaf flottenbedarf aus \textcite{vansteenwegen_survey_2022} + Einfluss von Fahrerpausen
    \item Kapazitätsfragen, Depotstruktur, Echtzeitfähigkeit
    \item kombination von on-demand und line-based nochmal evtl. aufgreifen nachdem entsprechend mit searchstring bewiesen oder nicht bewiesen ist, dass es diese kombination so noch nicht extensiv gibt
    \item Zukunftsperspektiven: adaptive Fahrpläne, Realtime-Demand
    \item Bewertung der Robustheit und Praktikabilität in Realanwendungen
    
    $\rightarrow$ Enden mit Rechtfertigung dafür, dass es sich lohnt die Kombination, die Vlćek und Schulz gemacht haben, weiter zu untersuchen
\end{itemize}

\textbf{--> Aussage on-demand senkt kosten in rural areas, aber ...}

%Hier Überleitung und EIngrenzung dessen was in der Arbeit behandelt wird