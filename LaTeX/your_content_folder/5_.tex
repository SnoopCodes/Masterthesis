\chapter{Erweiterungsmöglichkeiten \& Diskussion}
%%%%%%%%%%%%%%%%%%%%
%%%%%%%%%%%%%%%%%%%%
%%%%%%%%%%%%%%%%%%%%
%%%%%%%%%%%%%%%%%%%%
\section{Limitierungen des aktuellen Modells}
\begin{itemize}
    \item Nur ein Depot
    \item Statischer Demand
    \item Keine dynamische Tourenbildung, keine Live-Reaktionen
    \item Keine Betriebskostenbetrachtung
\end{itemize}

%%%%%%%%%%%%%%%%%%%%
%%%%%%%%%%%%%%%%%%%%
%%%%%%%%%%%%%%%%%%%%
%%%%%%%%%%%%%%%%%%%%
\section{Mögliche Erweiterungen}
\begin{itemize}
    \item Liniennetz überschneidet sich nicht, daher auch keine Linien übergreifenden Touren möglich
    \item unterscheidlich abzufahrende Stops (Linienzusammensetzung) je Tour
    \item Mehrere Depots: Flexibilität bei der Tourenplanung, bessere Abdeckung
    \item Depotzuordnung optimieren
    \item Zeitfensterbasierte oder dynamische Nachfrage
    \item Realtime-Routing mit Rolling Horizon
    \begin{itemize}
        \item dynamische Änderungen der Fahrzeiten zwischen den Stops sie Abstract von \textcite{lian_-demand_2023} 
    \end{itemize}
    \item Erweiterung um Ladezeiten, Servicelevel-Bedingungen
    \item größerer Datensatz
\end{itemize}

%%%%%%%%%%%%%%%%%%%%
%%%%%%%%%%%%%%%%%%%%
%%%%%%%%%%%%%%%%%%%%
%%%%%%%%%%%%%%%%%%%%
\section{Praxisrelevanz \& Umsetzung}
\begin{itemize}
    \item Welche Erkenntnisse sind direkt anwendbar?
    \item Welche Modellannahmen müssen für reale Implementierung angepasst werden?
    \item Bewertung der Lösung hinsichtlich Kosten, Fahrgastkomfort, Nachhaltigkeit
\end{itemize}
