\begin{table}[p]
\centering
\caption{Forschungsergebnisse zu ridepooling — urban}
\label{tab:ridepooling-urban-5}
\scriptsize
\setlength{\tabcolsep}{2.5pt}
\begin{tabularx}{\textwidth}{
  >{\RaggedRight\arraybackslash}p{2.8cm}
  >{\RaggedRight\arraybackslash}p{2.2cm}
  >{\RaggedRight\arraybackslash}p{1.8cm}
  >{\RaggedRight\arraybackslash}p{1.8cm}
  >{\RaggedRight\arraybackslash}p{2.0cm}
  >{\RaggedRight\arraybackslash}p{2.6cm}
  >{\RaggedRight\arraybackslash}p{1.8cm}
  >{\RaggedRight\arraybackslash}p{2.2cm}
  >{\RaggedRight\arraybackslash}X
}
\toprule
Paper & Zielsetzung & Region/Land & Betrachtungsebene & Fokus/Anwendungsfeld & Methode & Datengrundlage & Kennzahlen (KPI) & Zentrale Erkenntnis \\
\midrule
\textcite{vosooghi\_2019\_sharedautonomousve} & Autonomen DRT planen & Frankreich & Netz & Autonom & Simulation, Matem. Optimierungsmodell & — & Auslastung & Studie zeigt Potenziale und Grenzen urbanen Ridepoolings. \\ \hline
\textcite{wallar\_2018\_vehiclerebalancing} & Urbanes Ridepooling bewerten & ohne Bezug & Linie & Flexible Busse & Matem. Optimierungsmodell & — & Auslastung & Zielkonflikt zwischen Servicequalität und Bündelungsgrad wird sichtbar. \\ \hline
\textcite{wen\_2018\_transitorientedaut} & Autonomen DRT planen & USA & Netz & Autonom & Simulation, Matem. Optimierungsmodell & — & Auslastung & Studie zeigt Potenziale und Grenzen urbanen Ridepoolings. \\ \hline
\textcite{zwick\_2021\_ridepoolingefficie} & Ridepooling-Konzept bewerten & Deutschland & Region & Pooling & Simulation, Literatur-Review & Literatur & — & Studie zeigt Potenziale und Grenzen urbanen Ridepoolings. \\ \hline
\textcite{zwick\_2020\_analysisofridepool} & Pooling/Zuteilung optimieren & Schweiz & Netz & Pooling & Matem. Optimierungsmodell & synthetisch & — & Ridepooling erhöht Abdeckung und Auslastung gegenüber Solo-DRT. \\ \hline
\textcite{zwick\_2022\_ridepoolingdemandp} & Pooling/Zuteilung optimieren & Deutschland & Region & Pooling & Simulation, Kartenbasierte Raumanalyse (GIS) & — & Flottengröße, Fahrzeug-km & Pooling und Optimierung reduzieren Warte-/Reisezeiten und Fahrzeug-km im urbanen Kontext deutlich. \\ \hline
\bottomrule
\end{tabularx}
\end{table}
