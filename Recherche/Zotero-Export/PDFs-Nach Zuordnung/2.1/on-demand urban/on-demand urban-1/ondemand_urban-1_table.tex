\begin{table}[p]
\centering
\caption{Forschungsergebnisse zu on-demand — urban}
\label{tab:ondemand-urban-1}
\scriptsize
\setlength{\tabcolsep}{2.5pt}
\begin{tabularx}{\textwidth}{
  >{\RaggedRight\arraybackslash}p{2.8cm}
  >{\RaggedRight\arraybackslash}p{2.2cm}
  >{\RaggedRight\arraybackslash}p{1.8cm}
  >{\RaggedRight\arraybackslash}p{1.8cm}
  >{\RaggedRight\arraybackslash}p{2.0cm}
  >{\RaggedRight\arraybackslash}p{2.6cm}
  >{\RaggedRight\arraybackslash}p{1.8cm}
  >{\RaggedRight\arraybackslash}p{2.2cm}
  >{\RaggedRight\arraybackslash}X
}
\toprule
Paper & Zielsetzung & Region/Land & Betrachtungsebene & Fokus/Anwendungsfeld & Methode & Datengrundlage & Kennzahlen (KPI) & Zentrale Erkenntnis \\
\midrule
\textcite{abe\_2019\_introducingautonom} & Autonomen DRT planen & UK & Netz & Autonom & Matem. Optimierungsmodell & — & Kosten/Fahrgast-km, Erreichbarkeit & Optimierung/Szenarien senken Warte- und Reisezeiten im urbanen DRT deutlich. \\ \hline
\textcite{alonso\_gonza\_lez\_2018\_thepotentialofdema} & DRT-Konzept bewerten & ohne Bezug & Linie & Flexible Busse & Matem. Optimierungsmodell, Empirie & — & Auslastung & Ansatz verbessert Erreichbarkeit und Servicegrad gegenüber Status quo. \\ \hline
\textcite{charisis\_2018\_drtroutedesignfort} & First/Last-Mile verbessern & Deutschland & Linie & First/Last-Mile & — & — & Auslastung, Erreichbarkeit & Optimierung/Szenarien senken Warte- und Reisezeiten im urbanen DRT deutlich. \\ \hline
\textcite{diana\_2009\_amethodologyforcom} & DRT-Konzept bewerten & Frankreich & Linie & Flexible Busse & Matem. Optimierungsmodell & — & Auslastung & Ansatz verbessert Erreichbarkeit und Servicegrad gegenüber Status quo. \\ \hline
\textcite{giuffrida\_2021\_addressingthepubli} & DRT-Konzept bewerten & Italien & Region & Flexible Busse & Matem. Optimierungsmodell, Kartenbasierte Raumanalyse (GIS) & Realdaten & Auslastung, Erreichbarkeit & Studie zeigt Potenziale und Grenzen von DRT im urbanen Kontext. \\ \hline
\textcite{hazan\_demandtransitcanun} & Urbanen DRT bewerten & ohne Bezug & Region & Flexible Busse & Matem. Optimierungsmodell & — & Kosten/Fahrgast-km, Auslastung & Ansatz verbessert Erreichbarkeit und Servicegrad gegenüber Status quo. \\ \hline
\textcite{huang\_2020\_flexiblerouteoptim} & DRT-Konzept bewerten & ohne Bezug & Linie & Flexible Busse & Simulation, Matem. Optimierungsmodell & — & Auslastung & Studie zeigt Potenziale und Grenzen von DRT im urbanen Kontext. \\ \hline
\textcite{inturri\_2021\_taxivsdemandrespon} & DRT per Simulation prüfen & Italien & Netz & Flexible Busse & Simulation & — & Auslastung & Studie zeigt Potenziale und Grenzen von DRT im urbanen Kontext. \\ \hline
\bottomrule
\end{tabularx}
\end{table}
