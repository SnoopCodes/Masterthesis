\begin{table}[p]
\centering
\caption{Forschungsergebnisse zu on-demand — urban}
\label{tab:ondemand-urban-2}
\scriptsize
\setlength{\tabcolsep}{2.5pt}
\begin{tabularx}{\textwidth}{
  >{\RaggedRight\arraybackslash}p{2.8cm}
  >{\RaggedRight\arraybackslash}p{2.2cm}
  >{\RaggedRight\arraybackslash}p{1.8cm}
  >{\RaggedRight\arraybackslash}p{1.8cm}
  >{\RaggedRight\arraybackslash}p{2.0cm}
  >{\RaggedRight\arraybackslash}p{2.6cm}
  >{\RaggedRight\arraybackslash}p{1.8cm}
  >{\RaggedRight\arraybackslash}p{2.2cm}
  >{\RaggedRight\arraybackslash}X
}
\toprule
Paper & Zielsetzung & Region/Land & Betrachtungsebene & Fokus/Anwendungsfeld & Methode & Datengrundlage & Kennzahlen (KPI) & Zentrale Erkenntnis \\
\midrule
\textcite{ja\_ger\_2018\_multiagentsimulati} & Autonomen DRT planen & ohne Bezug & Region & Autonom & Matem. Optimierungsmodell & — & Auslastung & Optimierung/Szenarien senken Warte- und Reisezeiten im urbanen DRT deutlich. \\ \hline
\textcite{kim\_2025\_optimizationandimp} & DRT-Konzept bewerten & Schweiz & Region & Flexible Busse & Matem. Optimierungsmodell & — & Auslastung & Studie zeigt Potenziale und Grenzen von DRT im urbanen Kontext. \\ \hline
\textcite{li\_2024\_efficientrouteplan} & DRT-Konzept bewerten & Korea & Linie & Flexible Busse & — & — & Auslastung & Studie zeigt Potenziale und Grenzen von DRT im urbanen Kontext. \\ \hline
\textcite{liyanage\_2020\_anagentbasedsimula} & DRT per Simulation prüfen & Australien & Region & Flexible Busse & Simulation & — & Auslastung, CO₂ & Optimierung/Szenarien senken Warte- und Reisezeiten im urbanen DRT deutlich. \\ \hline
\textcite{melo\_2024\_demandresponsivetr} & First/Last-Mile verbessern & Schweiz & Netz & First/Last-Mile & — & — & Auslastung & Ansatz verbessert Erreichbarkeit und Servicegrad gegenüber Status quo. \\ \hline
\textcite{meshkani\_2024\_innovativeondemand} & First/Last-Mile verbessern & ohne Bezug & Region & First/Last-Mile & Matem. Optimierungsmodell & — & Auslastung & Studie zeigt Potenziale und Grenzen von DRT im urbanen Kontext. \\ \hline
\textcite{mortazavi\_2024\_integrateddemandre} & DRT-Konzept bewerten & Australien & Netz & Flexible Busse & Matem. Optimierungsmodell, Fallstudie & — & Reisezeit, Kosten/Fahrgast-km & Studie zeigt Potenziale und Grenzen von DRT im urbanen Kontext. \\ \hline
\textcite{pal\_2016\_smartporteracombin} & Urbanen DRT bewerten & ohne Bezug & Netz & Flexible Busse & Kartenbasierte Raumanalyse (GIS) & — & — & Optimierung/Szenarien senken Warte- und Reisezeiten im urbanen DRT deutlich. \\ \hline
\bottomrule
\end{tabularx}
\end{table}
