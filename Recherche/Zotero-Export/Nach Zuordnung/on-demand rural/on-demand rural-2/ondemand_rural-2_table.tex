\begin{table}[p]
\centering
\caption{Forschungsergebnisse zu on-demand — rural}
\label{tab:ondemand-rural-2}
\scriptsize
\setlength{\tabcolsep}{2.5pt}
\begin{tabularx}{\textwidth}{
  >{\RaggedRight\arraybackslash}p{2.8cm}
  >{\RaggedRight\arraybackslash}p{2.2cm}
  >{\RaggedRight\arraybackslash}p{1.8cm}
  >{\RaggedRight\arraybackslash}p{1.8cm}
  >{\RaggedRight\arraybackslash}p{2.0cm}
  >{\RaggedRight\arraybackslash}p{2.6cm}
  >{\RaggedRight\arraybackslash}p{1.8cm}
  >{\RaggedRight\arraybackslash}p{2.2cm}
  >{\RaggedRight\arraybackslash}X
}
\toprule
Paper & Zielsetzung & Region/Land & Betrachtungsebene & Fokus/Anwendungsfeld & Methode & Datengrundlage & Kennzahlen (KPI) & Zentrale Erkenntnis \\
\midrule

\textcite{} & Ruralen DRT bewerten & Dänemark & Region & Studierende & Simulation & — & Ø-Wartezeit, Auslastung & Studie zeigt Potenziale und Grenzen von DRT im ländlichen Kontext. \\ \hline

\textcite{galarza\_montenegro\_2021\_alargeneighborhood} & Feeder-Service optimieren & ohne Bezug & Linie & Feeder & Matem. Optimierungsmodell & — & Kosten/Fahrgast-km, Auslastung & Studie zeigt Potenziale und Grenzen von DRT im ländlichen Kontext. \\ \hline

\textcite{galarza\_montenegro\_2024\_ademandresponsivef} & Feeder-Service optimieren & ohne Bezug & Linie & Feeder & Matem. Optimierungsmodell & — & Ø-Wartezeit, Kosten/Fahrgast-km & Ansatz verbessert Erreichbarkeit und Servicegrad gegenüber Status quo. \\ \hline

\textcite{guo\_2023\_modularautonomouse} & Fahrplan/Zuteilung optimieren & China & Netz & Studierende & Matem. Optimierungsmodell, Empirie & — & Auslastung, Erreichbarkeit & Ansatz verbessert Erreichbarkeit und Servicegrad gegenüber Status quo. \\ \hline

\textcite{jiang\_2025\_integratedoptimiza} & Fahrplan/Zuteilung optimieren & ohne Bezug & Netz & Studierende & Simulation, Matem. Optimierungsmodell & Realdaten & Reisezeit, Flottengröße & Optimierung/Szenarien senken Warte- und Reisezeiten im ländlichen DRT deutlich. \\ \hline

\textcite{knierim\_2021\_theattitudeofpoten} & DRT-Konzept bewerten & Deutschland & Linie & Studierende & Empirie & Mix & Auslastung & Ansatz verbessert Erreichbarkeit und Servicegrad gegenüber Status quo. \\ \hline

\textcite{lu\_2023\_demandresponsivetr} & Ruralen DRT bewerten & Deutschland & Linie & Studierende & Simulation, Empirie & Realdaten & Flottengröße, Kosten/Fahrgast-km & Studie zeigt Potenziale und Grenzen von DRT im ländlichen Kontext. \\ \hline

\bottomrule
\end{tabularx}
\end{table}
